
% Default to the notebook output style

    


% Inherit from the specified cell style.




    
\documentclass{article}

    
    
    \usepackage{graphicx} % Used to insert images
    \usepackage{adjustbox} % Used to constrain images to a maximum size 
    \usepackage{color} % Allow colors to be defined
    \usepackage{enumerate} % Needed for markdown enumerations to work
    \usepackage{geometry} % Used to adjust the document margins
    \usepackage{amsmath} % Equations
    \usepackage{amssymb} % Equations
    \usepackage[mathletters]{ucs} % Extended unicode (utf-8) support
    \usepackage[utf8x]{inputenc} % Allow utf-8 characters in the tex document
    \usepackage{fancyvrb} % verbatim replacement that allows latex
    \usepackage{grffile} % extends the file name processing of package graphics 
                         % to support a larger range 
    % The hyperref package gives us a pdf with properly built
    % internal navigation ('pdf bookmarks' for the table of contents,
    % internal cross-reference links, web links for URLs, etc.)
    \usepackage{hyperref}
    \usepackage{longtable} % longtable support required by pandoc >1.10
    \usepackage{booktabs}  % table support for pandoc > 1.12.2
    

    
    
    \definecolor{orange}{cmyk}{0,0.4,0.8,0.2}
    \definecolor{darkorange}{rgb}{.71,0.21,0.01}
    \definecolor{darkgreen}{rgb}{.12,.54,.11}
    \definecolor{myteal}{rgb}{.26, .44, .56}
    \definecolor{gray}{gray}{0.45}
    \definecolor{lightgray}{gray}{.95}
    \definecolor{mediumgray}{gray}{.8}
    \definecolor{inputbackground}{rgb}{.95, .95, .85}
    \definecolor{outputbackground}{rgb}{.95, .95, .95}
    \definecolor{traceback}{rgb}{1, .95, .95}
    % ansi colors
    \definecolor{red}{rgb}{.6,0,0}
    \definecolor{green}{rgb}{0,.65,0}
    \definecolor{brown}{rgb}{0.6,0.6,0}
    \definecolor{blue}{rgb}{0,.145,.698}
    \definecolor{purple}{rgb}{.698,.145,.698}
    \definecolor{cyan}{rgb}{0,.698,.698}
    \definecolor{lightgray}{gray}{0.5}
    
    % bright ansi colors
    \definecolor{darkgray}{gray}{0.25}
    \definecolor{lightred}{rgb}{1.0,0.39,0.28}
    \definecolor{lightgreen}{rgb}{0.48,0.99,0.0}
    \definecolor{lightblue}{rgb}{0.53,0.81,0.92}
    \definecolor{lightpurple}{rgb}{0.87,0.63,0.87}
    \definecolor{lightcyan}{rgb}{0.5,1.0,0.83}
    
    % commands and environments needed by pandoc snippets
    % extracted from the output of `pandoc -s`
    \DefineVerbatimEnvironment{Highlighting}{Verbatim}{commandchars=\\\{\}}
    % Add ',fontsize=\small' for more characters per line
    \newenvironment{Shaded}{}{}
    \newcommand{\KeywordTok}[1]{\textcolor[rgb]{0.00,0.44,0.13}{\textbf{{#1}}}}
    \newcommand{\DataTypeTok}[1]{\textcolor[rgb]{0.56,0.13,0.00}{{#1}}}
    \newcommand{\DecValTok}[1]{\textcolor[rgb]{0.25,0.63,0.44}{{#1}}}
    \newcommand{\BaseNTok}[1]{\textcolor[rgb]{0.25,0.63,0.44}{{#1}}}
    \newcommand{\FloatTok}[1]{\textcolor[rgb]{0.25,0.63,0.44}{{#1}}}
    \newcommand{\CharTok}[1]{\textcolor[rgb]{0.25,0.44,0.63}{{#1}}}
    \newcommand{\StringTok}[1]{\textcolor[rgb]{0.25,0.44,0.63}{{#1}}}
    \newcommand{\CommentTok}[1]{\textcolor[rgb]{0.38,0.63,0.69}{\textit{{#1}}}}
    \newcommand{\OtherTok}[1]{\textcolor[rgb]{0.00,0.44,0.13}{{#1}}}
    \newcommand{\AlertTok}[1]{\textcolor[rgb]{1.00,0.00,0.00}{\textbf{{#1}}}}
    \newcommand{\FunctionTok}[1]{\textcolor[rgb]{0.02,0.16,0.49}{{#1}}}
    \newcommand{\RegionMarkerTok}[1]{{#1}}
    \newcommand{\ErrorTok}[1]{\textcolor[rgb]{1.00,0.00,0.00}{\textbf{{#1}}}}
    \newcommand{\NormalTok}[1]{{#1}}
    
    % Define a nice break command that doesn't care if a line doesn't already
    % exist.
    \def\br{\hspace*{\fill} \\* }
    % Math Jax compatability definitions
    \def\gt{>}
    \def\lt{<}
    % Document parameters
    \title{SeniorProjectPDF}
    
    
    

    % Pygments definitions
    
\makeatletter
\def\PY@reset{\let\PY@it=\relax \let\PY@bf=\relax%
    \let\PY@ul=\relax \let\PY@tc=\relax%
    \let\PY@bc=\relax \let\PY@ff=\relax}
\def\PY@tok#1{\csname PY@tok@#1\endcsname}
\def\PY@toks#1+{\ifx\relax#1\empty\else%
    \PY@tok{#1}\expandafter\PY@toks\fi}
\def\PY@do#1{\PY@bc{\PY@tc{\PY@ul{%
    \PY@it{\PY@bf{\PY@ff{#1}}}}}}}
\def\PY#1#2{\PY@reset\PY@toks#1+\relax+\PY@do{#2}}

\expandafter\def\csname PY@tok@gd\endcsname{\def\PY@tc##1{\textcolor[rgb]{0.63,0.00,0.00}{##1}}}
\expandafter\def\csname PY@tok@gu\endcsname{\let\PY@bf=\textbf\def\PY@tc##1{\textcolor[rgb]{0.50,0.00,0.50}{##1}}}
\expandafter\def\csname PY@tok@gt\endcsname{\def\PY@tc##1{\textcolor[rgb]{0.00,0.27,0.87}{##1}}}
\expandafter\def\csname PY@tok@gs\endcsname{\let\PY@bf=\textbf}
\expandafter\def\csname PY@tok@gr\endcsname{\def\PY@tc##1{\textcolor[rgb]{1.00,0.00,0.00}{##1}}}
\expandafter\def\csname PY@tok@cm\endcsname{\let\PY@it=\textit\def\PY@tc##1{\textcolor[rgb]{0.25,0.50,0.50}{##1}}}
\expandafter\def\csname PY@tok@vg\endcsname{\def\PY@tc##1{\textcolor[rgb]{0.10,0.09,0.49}{##1}}}
\expandafter\def\csname PY@tok@m\endcsname{\def\PY@tc##1{\textcolor[rgb]{0.40,0.40,0.40}{##1}}}
\expandafter\def\csname PY@tok@mh\endcsname{\def\PY@tc##1{\textcolor[rgb]{0.40,0.40,0.40}{##1}}}
\expandafter\def\csname PY@tok@go\endcsname{\def\PY@tc##1{\textcolor[rgb]{0.53,0.53,0.53}{##1}}}
\expandafter\def\csname PY@tok@ge\endcsname{\let\PY@it=\textit}
\expandafter\def\csname PY@tok@vc\endcsname{\def\PY@tc##1{\textcolor[rgb]{0.10,0.09,0.49}{##1}}}
\expandafter\def\csname PY@tok@il\endcsname{\def\PY@tc##1{\textcolor[rgb]{0.40,0.40,0.40}{##1}}}
\expandafter\def\csname PY@tok@cs\endcsname{\let\PY@it=\textit\def\PY@tc##1{\textcolor[rgb]{0.25,0.50,0.50}{##1}}}
\expandafter\def\csname PY@tok@cp\endcsname{\def\PY@tc##1{\textcolor[rgb]{0.74,0.48,0.00}{##1}}}
\expandafter\def\csname PY@tok@gi\endcsname{\def\PY@tc##1{\textcolor[rgb]{0.00,0.63,0.00}{##1}}}
\expandafter\def\csname PY@tok@gh\endcsname{\let\PY@bf=\textbf\def\PY@tc##1{\textcolor[rgb]{0.00,0.00,0.50}{##1}}}
\expandafter\def\csname PY@tok@ni\endcsname{\let\PY@bf=\textbf\def\PY@tc##1{\textcolor[rgb]{0.60,0.60,0.60}{##1}}}
\expandafter\def\csname PY@tok@nl\endcsname{\def\PY@tc##1{\textcolor[rgb]{0.63,0.63,0.00}{##1}}}
\expandafter\def\csname PY@tok@nn\endcsname{\let\PY@bf=\textbf\def\PY@tc##1{\textcolor[rgb]{0.00,0.00,1.00}{##1}}}
\expandafter\def\csname PY@tok@no\endcsname{\def\PY@tc##1{\textcolor[rgb]{0.53,0.00,0.00}{##1}}}
\expandafter\def\csname PY@tok@na\endcsname{\def\PY@tc##1{\textcolor[rgb]{0.49,0.56,0.16}{##1}}}
\expandafter\def\csname PY@tok@nb\endcsname{\def\PY@tc##1{\textcolor[rgb]{0.00,0.50,0.00}{##1}}}
\expandafter\def\csname PY@tok@nc\endcsname{\let\PY@bf=\textbf\def\PY@tc##1{\textcolor[rgb]{0.00,0.00,1.00}{##1}}}
\expandafter\def\csname PY@tok@nd\endcsname{\def\PY@tc##1{\textcolor[rgb]{0.67,0.13,1.00}{##1}}}
\expandafter\def\csname PY@tok@ne\endcsname{\let\PY@bf=\textbf\def\PY@tc##1{\textcolor[rgb]{0.82,0.25,0.23}{##1}}}
\expandafter\def\csname PY@tok@nf\endcsname{\def\PY@tc##1{\textcolor[rgb]{0.00,0.00,1.00}{##1}}}
\expandafter\def\csname PY@tok@si\endcsname{\let\PY@bf=\textbf\def\PY@tc##1{\textcolor[rgb]{0.73,0.40,0.53}{##1}}}
\expandafter\def\csname PY@tok@s2\endcsname{\def\PY@tc##1{\textcolor[rgb]{0.73,0.13,0.13}{##1}}}
\expandafter\def\csname PY@tok@vi\endcsname{\def\PY@tc##1{\textcolor[rgb]{0.10,0.09,0.49}{##1}}}
\expandafter\def\csname PY@tok@nt\endcsname{\let\PY@bf=\textbf\def\PY@tc##1{\textcolor[rgb]{0.00,0.50,0.00}{##1}}}
\expandafter\def\csname PY@tok@nv\endcsname{\def\PY@tc##1{\textcolor[rgb]{0.10,0.09,0.49}{##1}}}
\expandafter\def\csname PY@tok@s1\endcsname{\def\PY@tc##1{\textcolor[rgb]{0.73,0.13,0.13}{##1}}}
\expandafter\def\csname PY@tok@sh\endcsname{\def\PY@tc##1{\textcolor[rgb]{0.73,0.13,0.13}{##1}}}
\expandafter\def\csname PY@tok@sc\endcsname{\def\PY@tc##1{\textcolor[rgb]{0.73,0.13,0.13}{##1}}}
\expandafter\def\csname PY@tok@sx\endcsname{\def\PY@tc##1{\textcolor[rgb]{0.00,0.50,0.00}{##1}}}
\expandafter\def\csname PY@tok@bp\endcsname{\def\PY@tc##1{\textcolor[rgb]{0.00,0.50,0.00}{##1}}}
\expandafter\def\csname PY@tok@c1\endcsname{\let\PY@it=\textit\def\PY@tc##1{\textcolor[rgb]{0.25,0.50,0.50}{##1}}}
\expandafter\def\csname PY@tok@kc\endcsname{\let\PY@bf=\textbf\def\PY@tc##1{\textcolor[rgb]{0.00,0.50,0.00}{##1}}}
\expandafter\def\csname PY@tok@c\endcsname{\let\PY@it=\textit\def\PY@tc##1{\textcolor[rgb]{0.25,0.50,0.50}{##1}}}
\expandafter\def\csname PY@tok@mf\endcsname{\def\PY@tc##1{\textcolor[rgb]{0.40,0.40,0.40}{##1}}}
\expandafter\def\csname PY@tok@err\endcsname{\def\PY@bc##1{\setlength{\fboxsep}{0pt}\fcolorbox[rgb]{1.00,0.00,0.00}{1,1,1}{\strut ##1}}}
\expandafter\def\csname PY@tok@kd\endcsname{\let\PY@bf=\textbf\def\PY@tc##1{\textcolor[rgb]{0.00,0.50,0.00}{##1}}}
\expandafter\def\csname PY@tok@ss\endcsname{\def\PY@tc##1{\textcolor[rgb]{0.10,0.09,0.49}{##1}}}
\expandafter\def\csname PY@tok@sr\endcsname{\def\PY@tc##1{\textcolor[rgb]{0.73,0.40,0.53}{##1}}}
\expandafter\def\csname PY@tok@mo\endcsname{\def\PY@tc##1{\textcolor[rgb]{0.40,0.40,0.40}{##1}}}
\expandafter\def\csname PY@tok@kn\endcsname{\let\PY@bf=\textbf\def\PY@tc##1{\textcolor[rgb]{0.00,0.50,0.00}{##1}}}
\expandafter\def\csname PY@tok@mi\endcsname{\def\PY@tc##1{\textcolor[rgb]{0.40,0.40,0.40}{##1}}}
\expandafter\def\csname PY@tok@gp\endcsname{\let\PY@bf=\textbf\def\PY@tc##1{\textcolor[rgb]{0.00,0.00,0.50}{##1}}}
\expandafter\def\csname PY@tok@o\endcsname{\def\PY@tc##1{\textcolor[rgb]{0.40,0.40,0.40}{##1}}}
\expandafter\def\csname PY@tok@kr\endcsname{\let\PY@bf=\textbf\def\PY@tc##1{\textcolor[rgb]{0.00,0.50,0.00}{##1}}}
\expandafter\def\csname PY@tok@s\endcsname{\def\PY@tc##1{\textcolor[rgb]{0.73,0.13,0.13}{##1}}}
\expandafter\def\csname PY@tok@kp\endcsname{\def\PY@tc##1{\textcolor[rgb]{0.00,0.50,0.00}{##1}}}
\expandafter\def\csname PY@tok@w\endcsname{\def\PY@tc##1{\textcolor[rgb]{0.73,0.73,0.73}{##1}}}
\expandafter\def\csname PY@tok@kt\endcsname{\def\PY@tc##1{\textcolor[rgb]{0.69,0.00,0.25}{##1}}}
\expandafter\def\csname PY@tok@ow\endcsname{\let\PY@bf=\textbf\def\PY@tc##1{\textcolor[rgb]{0.67,0.13,1.00}{##1}}}
\expandafter\def\csname PY@tok@sb\endcsname{\def\PY@tc##1{\textcolor[rgb]{0.73,0.13,0.13}{##1}}}
\expandafter\def\csname PY@tok@k\endcsname{\let\PY@bf=\textbf\def\PY@tc##1{\textcolor[rgb]{0.00,0.50,0.00}{##1}}}
\expandafter\def\csname PY@tok@se\endcsname{\let\PY@bf=\textbf\def\PY@tc##1{\textcolor[rgb]{0.73,0.40,0.13}{##1}}}
\expandafter\def\csname PY@tok@sd\endcsname{\let\PY@it=\textit\def\PY@tc##1{\textcolor[rgb]{0.73,0.13,0.13}{##1}}}

\def\PYZbs{\char`\\}
\def\PYZus{\char`\_}
\def\PYZob{\char`\{}
\def\PYZcb{\char`\}}
\def\PYZca{\char`\^}
\def\PYZam{\char`\&}
\def\PYZlt{\char`\<}
\def\PYZgt{\char`\>}
\def\PYZsh{\char`\#}
\def\PYZpc{\char`\%}
\def\PYZdl{\char`\$}
\def\PYZhy{\char`\-}
\def\PYZsq{\char`\'}
\def\PYZdq{\char`\"}
\def\PYZti{\char`\~}
% for compatibility with earlier versions
\def\PYZat{@}
\def\PYZlb{[}
\def\PYZrb{]}
\makeatother


    % Exact colors from NB
    \definecolor{incolor}{rgb}{0.0, 0.0, 0.5}
    \definecolor{outcolor}{rgb}{0.545, 0.0, 0.0}



    
    % Prevent overflowing lines due to hard-to-break entities
    \sloppy 
    % Setup hyperref package
    \hypersetup{
      breaklinks=true,  % so long urls are correctly broken across lines
      colorlinks=true,
      urlcolor=blue,
      linkcolor=darkorange,
      citecolor=darkgreen,
      }
    % Slightly bigger margins than the latex defaults
    
    \geometry{verbose,tmargin=1in,bmargin=1in,lmargin=1in,rmargin=1in}
    
    

    \begin{document}
    
    
    \maketitle
    
    

    

    \section{Jeremy Mulcahey's Senior Project: The IPython Notebook for Data Analysis}



    \subsection{Contents}



    \paragraph{1 Introduction}


    


    \paragraph{2 Starting with Python}


    


    \paragraph{3 Setting up the IPython Notebook}


    


    \paragraph{4 Starting with GitHub}


    


    \paragraph{5 Sharing IPython Notebooks with NBViewer}



    \paragraph{6 R for the IPython Notebook}


    


    \paragraph{7 Analyzing NIST datasets in the IPython Notebook}


    7.3 ANOVA: SiR Dataset

7.4 Univariate Summary Statistics: PiDigits Dataset


    \paragraph{8 Streaming Data in the IPython Notebook}


    


    \paragraph{9 Time Series, More with DataFrames, and Advanced Plotting in the
IPython Notebook}


    9.2 Geiger Counter Data


    \paragraph{10 Formatting and Coverting IPython Notebooks }



    \section{Chapter 1 Introduction}


    This notebook is for anyone that has felt a tinge of excitement by the
mention of terms such as: Data Science, Big Data, Python, etc. The
IPython Notebook makes it easy to add another data analysis tool to your
kit. This document contains installation and set up instructions for
many of the tools that enable you to initially make the most of your own
IPython Notebooks. This document will also provide examples using Python
packages and coding in Python. As you grow more comfortable with Python,
there are many alternatives to the IPython Notebook such as the IPython
console, text editors, and IDEs, but those are not my focus. Dr.~Granger
and his team are striving to make the IPython Notebook capable of
analyzing and presenting data for all situations that will arise.


    \subsection{1.1 Background Requirements}


    Patience and an open mind. Ideally, this guide will be useful for the
full range of statistics students from ``I have never programmed
before'' to ``I know what i'm doing. I just want to know which programs
I need and where to get them.''


    \subsection{1.2 Goal}


    To pass along the struggles, successes, and code I learned while
analyzing data in the IPython Notebook. Some of the packages and
programs I will provide information on are:

Anaconda

Python

IPython Notebook

GitHub \& Sharing IPython Notebooks

Pandas

Numpy

Seaborn

NBViewer

Statsmodels

Requests

JSON

ASCII

SciPy

Rpy2

R in IPython Notebook

urllib2

and more!


    \subsection{1.3 Disclaimer}


    From the moment I started this project with Dr.~Doi, I have had to learn
everything I am sharing in this document. I have no prior experience
with Python, any of its packakges, LaTeX, the IPython Notebook, etc.
Many of the approaches and work arounds are that of a novice. Much of
the code I provide is not the only way to accomplish the task at hand
and most of it might not be the best way either. If you feel a section
of code can be improved, or you find packages that do the same work as
some of my functions, I encourage you to use them or write your own
improvements.

I hope you gain as much using this notebook as I gained writing it. 


    \section{Chapter 2 Starting with Python}



    \subsection{2.1 Why Python?}


    The obvious answer is the IPython Notebook. IPython Notebook is a
one-stop shop for data analysis, widgets, homework, and projects. The
IPython notebook can take the place of an IDE, text editor, and/or
console. Working in the IPython Notebook enables you to write code,
analyze data in Python and R, format using LaTeX and HTML, and produce
graphs - all in the same document! The notebooks can be converted to
HTML, LaTeX, PDF, and more. The notebooks can also be shared, stored,
and backed up using NBViewer and GitHub, which we will discuss later.

Two important points to keep in mind as we introduce Python are:

Python is Executebale Pseudocode,

\& Python is an object oriented language.

If this is the first time you have heard these terms, there is no reason
to be intimiated. Each of these bullet points is effectively
contributing to the same idea, ``{[}T{]}he Python language is easy to
fall in love with.'' (McKinney)

Python is Executeable Pseudocode. This is a spoiling characteristic of
the Python language. As you learn about writing code, or for those with
coding experience, Python will surprise you time and time again as code
you write executes with minimal syntax errors. With little understanding
of programming logic, users can write what they think should work and,
more often than not, it will work.

Python is an object oriented language. Many smarter and more experienced
programmers are working tirelessly to make analyzing data as painless as
possible. A large portion of what we need Python for has already been
coded into modules, libraries, methods, objects, etc. Thanks to these
objects containing their own functions/methods and data,various examples
in this notebook will require very few lines of code to produce a lot of
information.

If you would like a more indepth introduction of the IPython Notebook,
here is what CO-founder Dr.~Granger has to say about it:
http://bit.ly/1rVyrpi


    \subsection{2.2 Installing Python}


    During this project, Dr.~Doi has put together a document for
``Installation and Configuration of Python on PC* (mainly based on
Sheppard's Inrtro to Python for Econometrics, Statisics, and Data
Analysis - IPESDA)''. Most of this section will be directly from his
document. *{[}see IPESDA for install instructions for Mac/Linux{]}

The first and largest step is the installation of Anaconda. Anaconda is
a ``Completely free enterprise-ready Python distribution for large-scale
data processing, predictive analytics, and scientific computing'' with
``195+ of the most popular Python packages for science, math,
engineering, and data analysis''. Thanks to Anaconda, beginning in
Python is a relatively painless process.

    • Download Anaconda (https://store.continuum.io/cshop/anaconda/)

Follow the link above then click on ``Download Anaconda'' in the upper
right.

Click on the picture that matches your operating system.

For this project, everything is coded in Python 2.7, not 3.4.
Additionally, the books and resources Dr.~Doi and I used to learn python
and build this project suggest the use of Python 2.7, for now. So,
download Python 2.7.

Open the installer and follow the steps for the steps for the
default/recommended installation. During installation, be sure to
install in default directory(C:/Anaconda). If Anaconda is not installed
there, be sure that target directory contains no unicode characters or
spaces. Otherwise subsequent steps may not work well. 


    \subsection{2.3 Updating Conda, Anaconda, and Anaconda Packages}


    • After installation is complete, open the command prompt. This can be
done by opening the windows start menu by clicking in the lower left of
your desktop. Just above the icon there will be a search box. Type
letters cmd into the search box and hit enter. A black window will open
on the screen. Type the following lines one at time, hitting enter after
each one and allowing the program to finish before submitting the next
line:

\begin{verbatim}
conda update conda
conda update anaconda
\end{verbatim}

The statements above can be used at any point to ensure that Anaconda is
up to date with the latest packages.

I recommend repeating this update process periodically with any packages
you use. Simply open cmd, type conda update and the name of the package
you want to check for update and hit enter. Do it now.

\begin{verbatim}
conda update pandas
\end{verbatim}

The statement above will update the pandas package and the packages it
is built on (ie Numpy). This is helpful since the individual packages
update more frequently than Anaconda. A few times during this project I
was attempting to access functions in a packages that I found in online
documentation, but they did not exist. Executing these update commands
solved the problem every time.


    \subsection{2.4 Installing and Updating Additional Packages}


    The command prompt (cmd) should still be open at this point. If it is
not, then open it.

Let's install our first package using the command prompt the same way we
used it to update a package. Type the following line and hit enter:

\begin{verbatim}
pip install seaborn
\end{verbatim}

This statement installs the ``Seaborn: statistical data visulation''
package. This will be used in a later section.

This process can be repeated for any packages you come across and want
to try.

\begin{verbatim}
pip install [name of package]
\end{verbatim}

Note: Packages installed using pip have a different update command:

\begin{verbatim}
pip install --upgrade [name of package]
\end{verbatim}

TIP: For pip packages that fail to update, this has worked for me:

\begin{verbatim}
pip uninstall seaborn
pip install seaborn
\end{verbatim}


    \subsection{2.5 Anaconda Add-Ons (Optional)}


    I intend to use these Add-Ons in the future, but I did not use them for
this project. If you want to move forward with the installations, skip
to the next section. Otherwise, here are the descriptions and
instructions from Dr.~Doi's installation guide:

• Get Anaconda Add-Ons (https://store.continuum.io/cshop/anaconda/).

\begin{verbatim}
conda update conda
conda install accelerate
conda install iopro
\end{verbatim}


    \section{Chapter 3Setting up the IPython Notebook}



    \subsection{3.1 Install MathJax}


    Whether or not you intend to use LaTeX, it is recommended that MathJax
is installed before using IPython Notebook. The package that enables us
to use LaTeX code in the IPython Notebook is called MathJax. Since
MathJax is included in the Anaconda package list
(http://docs.continuum.io/anaconda/pkg-docs.html), it can be managed
with conda install and conda update (not pip). Running this conda
install will add MathJax 2.2-0 (or later) as a ``NEW'' package. So,
install it now:

\begin{verbatim}
conda install mathjax
Hit y, then enter.
\end{verbatim}


    \subsection{3.2 Accessing the IPython Notebook}


    Everything, from a Python standpoint, is ready to use. There are several
ways to access the IPython Notebook for regular use. I would use one of
the following three options:

Command Prompt,

Windows start up,

or creating a shortcut.

Command Prompt: Since the Anaconda installation has the IPython Notebook
open in a restrictive directory, the command prompt is the most
versatile way to your notebook kernel. This approach gives you the
option to work out of any directory you choose. I prefer to use
C:\textless{}/b\textgreater{} as my home directory since all of my
notebooks and files are readily accessible from this location. To do
this I type, type cd ../../. Then type ipython notebook.

If you choose to use a different directory, or have more directories to
back out of to reach C:, simply use cd ../ for each directory you need
to back out of, then cd {[}directory name{]} to change to the desired
directory (ie. cd desktop/myfiles/calpoly)

Windows start up: Click the Windows start up menu and type IPython into
the search box. You will see these options:

Select IPython Notebook from the list and it will open in your
Web-browser.

Creating a shortcut: Do eveything from the ``Windows start up''
instructions above, except for selecting ``IPython Notebook from the
list''. Instead, right-click and hold on IPython Notebook and drag it to
your desktop. Release the right-click and select Create shortcuts here.

From there, you can move the shortcut anywhere you desire.


    \subsection{3.3 Your first IPython Notebook}


    Once IPython Notebook opens in the browser, you will see a relatively
blank page with the IP{[}y{]}: Notebook heading.

Click on New Notebook.

Change the title by clicking on the word Untitled by the IP{[}y{]}:
Notebook header.

``Enter a new notebook name:'', hit ok, and you will have your first
IPython notebook.

You're ready to start using Python, but that's only part of our process.
The last big step we need to take is sharing and backing-up your
notebooks on GitHub.

Note:After these installation chapters, if you would like to learn more
about the IPython Notebook from Co-founder Dr.~Granger:
http://bit.ly/Zsoiqc 


    \section{Chapter 4 Starting with GitHub}


    If you haven't heard of GitHub yet, you will. Several people I have
asked about getting a data science job have given me the same order,
``Get a GitHub''. GitHub is the easiest way to share your work with
potential employers, back-up your projects/assignments, and work
simultaneously with other students/colleagues on different sections of
the same project. Aside from how essential GitHub is for aspiring data
scientists and programmers, it's required to use NBViewer (how we
currently share the IPython Notebooks).


    \subsection{4.1 Create an account}


    Creating an account with GitHub is very straightforward. Go to
https://github.com/ and create one now.


    \subsection{4.2 Learn what GitHub is and how to use it}


    GitHub has greatly simplified every process we will need to set up and
share Notebooks. We can work around many of the issues I struggled with
over the past year.

Thoroughly read this GitHub introduction (it's brief) and you will
almost be done with setting up and using your GitHub.

https://guides.github.com/activities/hello-world/


    \subsection{4.3 Installing and understanding GitHub Desktop}


    This miraculous tool is what now makes GitHub so simple that anyone can
use it.

Install GitHub Desktop using this link: https://windows.github.com/

Read this brief tutorial on GitHub desktop:
https://guides.github.com/introduction/getting-your-project-on-github/index.html

If you are new to programming, I highly recommend using the desktop tool
from this point on.


    \subsection{4.4 Using GitHub Desktop}


    Open GitHub desktop.

Locate the folder that stored your IPython Notebook from earlier. If you
are not sure where it is stored, you can perform a windows search to
find it.

Type the name of your notebook into the search bar, or search for
``IPython Notebooks''.

Once you have located the folder containing your notebook, drag it into
GitHub desktop (as done in the previous tutorial).

Now the repository can be viewed online in your GitHub account.

Since the tutorial uses GitHub Desktop for Mac, I'll show you how I use
it for windows.

Return to your IPython Notebook in your web browser. Type print(`Hello
World') into the box and hit Shift+Enter. The notebook will execute this
line of code. Hit the save icon on the left side of the icon bar.

Having completed a change to your Notebook, return to GitHub desktop.
GitHub desktop will now say you have Uncommitted changes. In the summary
box, write what you feel summarizes the change you made. For this case,
type Hello into the Summary box. You can add a detailed description in
the Description box, if you choose.

The right side of the program will show you what has been removed and
added from the file with the Uncommitted Changes. Below is an example of
a change I made to this section.

After reviewing the changes, click Commit to master.

Finally, in the upper right corner of the program, click the Sync
button.

From now on, your notebooks will be automatically updated in GitHub
desktop. Repeat this commit process whenever you want to back them up
online or share the changes you have made. 


    \section{Chapter 5 Sharing IPython Notebooks with NBViewer}


    The hard parts are over.

To share notebooks, we need to use NBViewer. ``IPython Notebook Viewer
is a free webservice that allows you to share static html versions of
hosted notebook files. If a notebook is publicly available, by giving
its url to the Viewer, you should be able to view it.''

To obtain a url for sharing your notebook, go to
http://nbviewer.ipython.org/ Simply type your GitHub username (Section
4.1) into the box provided and hit ``Go!''.

Save the URL from this page. This is the page that allows others to view
your committed IPython Notebooks on GitHub using NBViewer!

 


    \section{Chapter 6R for IPython Notebook}


    Is it time to code yet?

It could be\ldots{} As statistics majors, we have a required course in
R. Since some of you might have prior experience with R, we should set R
up right now. You can familiarize yourself with your IPython Notebook by
running your pre-existing R scripts, or code snippets from your
introductory statistics courses.

NOTE: This is the most important note yet. Some time, some day, you can
expect to see Python 3 and R support built-in to the IPython Notebook
itself. It will be as simple as a dropdown menu in the upper-right
corner of the notebook.

Until then, if you really want R now (like I did), buckle-up! It's going
to be a bumpy ride.


    \subsection{6.1 Installing Rpy2}


    Rpy2 is the package we need to run R through the IPython Notebook.
Unfortunately, it's not as simple as pip install rpy2.

I have it on good authority that this entire section is completely
unnecessary for Mac users. Rpy2 installs and works on Macs with zero
issues. For us Windows users, you are benefiting from many meetings
between Cal Poly's Dr.~Brian Granger and myself, and many failed
attempts to install Rpy2 on Windows. Luckly, there is now a native R
kernel for IPython.

To obtain Rpy2, click on the link. Hit control+f and type rpy2 into the
search bar. http://www.lfd.uci.edu/\textasciitilde{}gohlke/pythonlibs/

Hitting enter a couple times should take you to the Rpy2 section.
Download version 2-2.4.3 for the correct operating system.

Follow the next steps carefully. In my experience, here's where it gets
tricky. The wrong combination of Rpy2 and version of R crashes the
kernel. If you have R installed, I recommend backing up your scripts
before proceeding (and anything else you want to save). If you do not
have R installed, and you are using Windows 8.1, I recommend installing
R version 3.0.2. http://cran.r-project.org/bin/windows/base/old/3.0.2/

For those without R installed:

Follow the link above. Download and install R version 3.0.2.

Run the Rpy2 installation file.

For those with R installed:

Run the Rpy2 installation file.


    \subsection{6.2 Testing the Rpy2 installation}


    Go to your IPython Notebook. In the next cell (the empty one below Hello
World), type these lines in their own cells (hitting shift+enter after
each line):

    \begin{Verbatim}[commandchars=\\\{\}]
{\color{incolor}In [{\color{incolor}1}]:} \PY{k+kn}{import} \PY{n+nn}{rpy2}
\end{Verbatim}

    \begin{Verbatim}[commandchars=\\\{\}]
{\color{incolor}In [{\color{incolor}2}]:} \PY{o}{\PYZpc{}}\PY{k}{load\PYZus{}ext} \PY{n}{rpy2}\PY{o}{.}\PY{n}{ipython}
\end{Verbatim}

    \begin{Verbatim}[commandchars=\\\{\}]
{\color{incolor}In [{\color{incolor}3}]:} \PY{o}{\PYZpc{}}\PY{k}{R} \PY{n}{install}\PY{o}{.}\PY{n}{packages}\PY{p}{(}\PY{l+s}{\PYZdq{}}\PY{l+s}{lattice}\PY{l+s}{\PYZdq{}}\PY{p}{)}
\end{Verbatim}

    Scroll down to USA and select one of the USA portals.

    \begin{Verbatim}[commandchars=\\\{\}]
{\color{incolor}In [{\color{incolor}4}]:} \PY{o}{\PYZpc{}}\PY{k}{R} \PY{n}{library}\PY{p}{(}\PY{n}{lattice}\PY{p}{)}
\end{Verbatim}

            \begin{Verbatim}[commandchars=\\\{\}]
{\color{outcolor}Out[{\color{outcolor}4}]:} <StrVector - Python:0x000000000CDD5E48 / R:0x0000000022F81408>
        [str, str, str, \ldots, str, str, str]
\end{Verbatim}
        
    If your output matches mine, then R is working. It is successfully
downloading and installing packages. Skip ahead to Chapter 7.


    \subsection{6.3 If the Kernel Crashes}


    Go to your windows control panel.

Scroll down to Programs and Features.

Uninstall both R for Windows and Python 2.7 rpy2-2.4.3.

Download a different version of R (ie 3.0.1, 3.0.3, 3.1.0, etc). Install
the new version of R and install Rpy2 again.

Return to the ``Testing Rpy2 Installation'' section above and repeat the
process.

There should be a combination that enables R to work in the IPython
Notebook.

After Rpy2 is working as intended, install any version of R you prefer
and add your backed up files. Rpy2 will work with the version it needs
to and when you work exclusively in R, you can use your preferred
version.


    \subsection{6.4 R installation Final Notes}


    Dr.~Granger and his team are constantly working to improve functionality
in all areas of the IPython Notebook. R is one of those areas.
Originally, getting Rpy2 to work on a PC was almost impossible. Then, it
was very hard. Now, it's a bit touchy. Soon, as previously mentioned, it
will be a drop down option in the notebook itself. 


    \section{Chapter 7Analyzing NIST datasets in the IPython Notebook}


    Before introducing Python packages as an alternative to R and SAS,
Dr.~Doi and I felt it wise to investigate the precision of Python's data
analysis capabilities compared to R's and SAS's.

The NIST (National Institute of Standards and Technology) ``is the
federal technology agency that works with industry to develop and apply
technology, measurements, and standards.''

In this section I will analyze datasets from the NIST's Dataset
Archives. I will compare the NIST's ``Certified Values'' to the values
obtained from analyzing the data in R, SAS, and Python.

This section will provide an introduction to coding in Python, using
statistical packages for analyzing and visualizing data, and establish
ways to extract precise values in R, SAS, and Python.

Note: If you would like to learn more about the Python syntax, visit
Dr.~Granger's notebook (http://bit.ly/1y8h6hS). Otherwise, you can copy
and paste sections of my code and change the arguments as needed.


    \subsection{7.1 Python Packages for data analysis and the Import Cell}


    Packages can be imported for use in any cell at any time. My preference
is to import relevant packages and commands into a common cell at the
beginning of the notebook, or section of the notebook. This will provide
a collection of packages, with their abbreviations, in one convenient
location. In the event that you cannot remember if you have imported a
package, or what you imported it as, you can jump to your import cell.

    Below is the list of packages we will need for this section. Please go
to your cmd and pip install urllib2 before moving on. Then, copy and
execute the cell of packages by pasting them into your notebook cell and
hitting shift+enter.

    \begin{Verbatim}[commandchars=\\\{\}]
{\color{incolor}In [{\color{incolor}2}]:} \PY{k+kn}{import} \PY{n+nn}{urllib2} \PY{k+kn}{as} \PY{n+nn}{ul}
        \PY{k+kn}{import} \PY{n+nn}{pandas} \PY{k+kn}{as} \PY{n+nn}{pd}
        \PY{k+kn}{import} \PY{n+nn}{numpy} \PY{k+kn}{as} \PY{n+nn}{np}
        \PY{k+kn}{import} \PY{n+nn}{matplotlib}
        \PY{k+kn}{import} \PY{n+nn}{scipy} \PY{k+kn}{as} \PY{n+nn}{sp}
        \PY{k+kn}{from} \PY{n+nn}{statsmodels.formula.api} \PY{k+kn}{import} \PY{n}{ols}
        \PY{k+kn}{from} \PY{n+nn}{statsmodels.stats.anova} \PY{k+kn}{import} \PY{n}{anova\PYZus{}lm}
        \PY{k+kn}{import} \PY{n+nn}{matplotlib.pyplot} \PY{k+kn}{as} \PY{n+nn}{plt}
        \PY{k+kn}{from} \PY{n+nn}{IPython.core.display} \PY{k+kn}{import} \PY{n}{Image}
        \PY{k+kn}{from} \PY{n+nn}{matplotlib.gridspec} \PY{k+kn}{import} \PY{n}{GridSpec}
        \PY{k+kn}{import} \PY{n+nn}{seaborn} \PY{k+kn}{as} \PY{n+nn}{sns}
        
        \PY{c}{\PYZsh{}This line allows the graphs to show up in the notebook cells}
        \PY{o}{\PYZpc{}}\PY{k}{matplotlib} \PY{n}{inline}
\end{Verbatim}


    \subsection{7.2 Linear Regression Analysis: Norris Dataset}



    \subsubsection{7.2.1 Object Oriented Language Introduction: Reading in and preparing
data from an ASCII webpage}


    \begin{Verbatim}[commandchars=\\\{\}]
{\color{incolor}In [{\color{incolor}3}]:} \PY{c}{\PYZsh{}create a variable for the web address}
        \PY{n}{url} \PY{o}{=} \PY{l+s}{\PYZsq{}}\PY{l+s}{http://www.itl.nist.gov/div898/strd/lls/data/LINKS/DATA/Norris.dat}\PY{l+s}{\PYZsq{}}
        
        \PY{c}{\PYZsh{}open creates a file object named Norris.dat}
        \PY{c}{\PYZsh{}wb allows us to write to the file object}
        \PY{c}{\PYZsh{}file objects have a built\PYZhy{}in function to write to the file}
        \PY{c}{\PYZsh{}use the urllib2 as ul package to write the website to the file}
        \PY{c}{\PYZsh{}the .read function is built\PYZhy{}in to the ul object}
        \PY{n+nb}{open}\PY{p}{(}\PY{l+s}{\PYZsq{}}\PY{l+s}{data/Norris.dat}\PY{l+s}{\PYZsq{}}\PY{p}{,}\PY{l+s}{\PYZsq{}}\PY{l+s}{wb}\PY{l+s}{\PYZsq{}}\PY{p}{)}\PY{o}{.}\PY{n}{write}\PY{p}{(}\PY{n}{ul}\PY{o}{.}\PY{n}{urlopen}\PY{p}{(}\PY{n}{url}\PY{p}{)}\PY{o}{.}\PY{n}{read}\PY{p}{(}\PY{p}{)}\PY{p}{)}
\end{Verbatim}

    This is our first cell that takes advantage of Python being an object
oriented language. It might take some time to wrap your head around it,
or it might not.

What just happened is\ldots{} as we create objects, which are exactly
what you might intuitively think of as objects (ie: a ball), those
objects have their own functions (methods) and characteristics (data) we
can immediately use.

Let's continute with the ball example. If you have a ball, you do not
determine its color, size, or inertia. You do not show the ball how to
roll. The ball looks how it looks and if you set it on a decline, it
rolls away.

The ball knows what it knows, and you know how to use it to get what you
want. You can roll the ball at pins, throw it, roll it up hill, hit it
with your hand, hit it with a racket, etc.

That is what we are doing here. A file knows how to write to itself, and
the open function knows how to read the content of the page. By passing
arguments (such as the filename and url) we are using what the objects
know to accomplish what we need to accomplish.

    \begin{Verbatim}[commandchars=\\\{\}]
{\color{incolor}In [{\color{incolor}4}]:} \PY{c}{\PYZsh{}creates a np array named NorrisData}
        \PY{c}{\PYZsh{}uses the loadtxt method from the NumPy package (np) to read in }
        \PY{c}{\PYZsh{}the data from line 60 to the end of the file}
        \PY{n}{NorrisData} \PY{o}{=} \PY{n}{np}\PY{o}{.}\PY{n}{loadtxt}\PY{p}{(}\PY{l+s}{\PYZsq{}}\PY{l+s}{data/Norris.dat}\PY{l+s}{\PYZsq{}}\PY{p}{,}\PY{n}{skiprows}\PY{o}{=}\PY{l+m+mi}{60}\PY{p}{)}
\end{Verbatim}

    Since the NIST datasets are provided in ASCII and start on line 60 of
the webpage, I found the best way to import them was to create the file
(which we just completed), import them as a NumPy array, then convert
them to a Pandas DataFrame.

Having data in a Pandas DataFrame provides the greatest amount of
flexbility for analyzing and manipulating data. My goal through-out the
project was to make sure I could get any data I was working with into
this format.

    \begin{Verbatim}[commandchars=\\\{\}]
{\color{incolor}In [{\color{incolor}5}]:} \PY{c}{\PYZsh{}create a DataFrame object using the Pandas package (pd)}
        \PY{c}{\PYZsh{}the columns can be named during the conversion from ndarray to pd.df}
        \PY{n}{NorrisFrame} \PY{o}{=} \PY{n}{pd}\PY{o}{.}\PY{n}{DataFrame}\PY{p}{(}\PY{n}{NorrisData}\PY{p}{,}\PY{n}{columns}\PY{o}{=}\PY{p}{[}\PY{l+s}{\PYZsq{}}\PY{l+s}{y}\PY{l+s}{\PYZsq{}}\PY{p}{,}\PY{l+s}{\PYZsq{}}\PY{l+s}{x}\PY{l+s}{\PYZsq{}}\PY{p}{]}\PY{p}{)}
\end{Verbatim}

    Always check to make sure the data was read in correctly. There are many
ways to do this.

    \begin{Verbatim}[commandchars=\\\{\}]
{\color{incolor}In [{\color{incolor}6}]:} \PY{c}{\PYZsh{}the dataframe object displays its first five observations}
        \PY{n}{NorrisFrame}\PY{o}{.}\PY{n}{head}\PY{p}{(}\PY{p}{)}
\end{Verbatim}

            \begin{Verbatim}[commandchars=\\\{\}]
{\color{outcolor}Out[{\color{outcolor}6}]:}        y      x
        0    0.1    0.2
        1  338.8  337.4
        2  118.1  118.2
        3  888.0  884.6
        4    9.2   10.1
\end{Verbatim}
        
    \begin{Verbatim}[commandchars=\\\{\}]
{\color{incolor}In [{\color{incolor}7}]:} \PY{c}{\PYZsh{}the dataframe object displays its last five observations}
        \PY{n}{NorrisFrame}\PY{o}{.}\PY{n}{tail}\PY{p}{(}\PY{p}{)}
\end{Verbatim}

            \begin{Verbatim}[commandchars=\\\{\}]
{\color{outcolor}Out[{\color{outcolor}7}]:}         y      x
        31  117.6  118.3
        32  228.9  229.2
        33  668.4  669.1
        34  449.2  448.9
        35    0.2    0.5
\end{Verbatim}
        
    It's clear to me that we have all 36 observations (indicies 0 to 35).
The first value matches the first value of the webpage and the last
value macthes the last value of the webpage. If you're uncomfortable
with the indexing, you can check the number of observations with:

    \begin{Verbatim}[commandchars=\\\{\}]
{\color{incolor}In [{\color{incolor}8}]:} \PY{c}{\PYZsh{}number of observations}
        \PY{n+nb}{len}\PY{p}{(}\PY{n}{NorrisFrame}\PY{p}{)}
\end{Verbatim}

            \begin{Verbatim}[commandchars=\\\{\}]
{\color{outcolor}Out[{\color{outcolor}8}]:} 36
\end{Verbatim}
        
    Now is a good time to introduce the dir() function and explain why the
previous two functions required no arguments.

In Python, accessing functions/methods of an object always passes the
object as the first argument. We do not see it, but what the code is
really doing is executing NorrisFrame.head(NorrisFrame), which returns
the head of the NorrisFrame object.

The dir() function is how I knew to use the .head() and .tail()
functions.

To access the full list of data and methods an object has, simply type:
dir(object name)

    \begin{Verbatim}[commandchars=\\\{\}]
{\color{incolor}In [{\color{incolor}10}]:} \PY{c}{\PYZsh{}execute this code in your notebook }
         \PY{c}{\PYZsh{}it is too much output for this document}
         \PY{n+nb}{dir}\PY{p}{(}\PY{n}{NorrisFrame}\PY{p}{)}
\end{Verbatim}

    As you can see, the Pandas DataFrame is a robust and versatile object.

    Before moving to the next section, it's worth summarzing and
acknowledging what we've done. We created a data file from an ASCII
webpage, read the data file into an array, and converted it to a
DataFrame with named columns - in 4 lines of code.


    \subsubsection{7.2.2 NIST Certified Values}


    The Certified values we are trying to match can be found at:
http://www.itl.nist.gov/div898/strd/anova/SiRstv\_cv.html

For easy reference, I have included them:

The Certified values are quite precise. It's important to view them so
we can match the number of decimal places using SAS, R, and Python.

    \begin{Verbatim}[commandchars=\\\{\}]
{\color{incolor}In [{\color{incolor}11}]:} \PY{c}{\PYZsh{}saving the values as strings (ie in quotes \PYZsq{}\PYZsq{}) helps us with future }
         \PY{c}{\PYZsh{}steps in this process, we will need their len() and to compare them digit by }
         \PY{c}{\PYZsh{}digit, this also solves a problem I had with trailing zeros being ignored}
         \PY{n}{B0} \PY{o}{=} \PY{l+s}{\PYZsq{}}\PY{l+s}{\PYZhy{}0.262323073774029}\PY{l+s}{\PYZsq{}}
         \PY{n}{B1} \PY{o}{=} \PY{l+s}{\PYZsq{}}\PY{l+s}{1.00211681802045}\PY{l+s}{\PYZsq{}}
         \PY{n}{STDofEstB0} \PY{o}{=} \PY{l+s}{\PYZsq{}}\PY{l+s}{0.232818234301152}\PY{l+s}{\PYZsq{}}
         \PY{n}{STDofEstB1} \PY{o}{=} \PY{l+s}{\PYZsq{}}\PY{l+s}{0.000429796848199937}\PY{l+s}{\PYZsq{}}
         \PY{n}{resstd} \PY{o}{=} \PY{l+s}{\PYZsq{}}\PY{l+s}{0.884796396144373}\PY{l+s}{\PYZsq{}}
         \PY{n}{Rsq} \PY{o}{=} \PY{l+s}{\PYZsq{}}\PY{l+s}{0.999993745883712}\PY{l+s}{\PYZsq{}}
         \PY{n}{ModSS} \PY{o}{=} \PY{l+s}{\PYZsq{}}\PY{l+s}{4255954.13232369}\PY{l+s}{\PYZsq{}}
         \PY{n}{ModMSE} \PY{o}{=} \PY{l+s}{\PYZsq{}}\PY{l+s}{4255954.13232369}\PY{l+s}{\PYZsq{}}
         \PY{n}{ModSSResid} \PY{o}{=} \PY{l+s}{\PYZsq{}}\PY{l+s}{26.6173985294224}\PY{l+s}{\PYZsq{}}
         \PY{n}{ModMSEResid} \PY{o}{=} \PY{l+s}{\PYZsq{}}\PY{l+s}{0.782864662630069}\PY{l+s}{\PYZsq{}}
         \PY{n}{Fstat} \PY{o}{=} \PY{l+s}{\PYZsq{}}\PY{l+s}{5436385.54079785}\PY{l+s}{\PYZsq{}}
         
         \PY{c}{\PYZsh{}creates an array object named CertVals containing the certified values}
         \PY{n}{CertVals} \PY{o}{=} \PY{n}{np}\PY{o}{.}\PY{n}{array}\PY{p}{(}\PY{p}{[}\PY{n}{B0}\PY{p}{,} \PY{n}{B1}\PY{p}{,} \PY{n}{STDofEstB0}\PY{p}{,} \PY{n}{STDofEstB1}\PY{p}{,}\PY{n}{resstd}\PY{p}{,}\PY{n}{Rsq}\PY{p}{,}\PY{n}{ModSS}\PY{p}{,}
                              \PY{n}{ModMSE}\PY{p}{,} \PY{n}{ModSSResid}\PY{p}{,}\PY{n}{ModMSEResid}\PY{p}{,}\PY{n}{Fstat}\PY{p}{]}\PY{p}{)}
\end{Verbatim}

    We now have the NIST Certified values in an array for our later
comparisons.


    \subsubsection{7.2.3 Linear Regression and ANOVA values in Python}


    The package we will use for data analysis in this notebook is
Statsmodels.

Their help documentation can be found at:
http://statsmodels.sourceforge.net/

    \begin{Verbatim}[commandchars=\\\{\}]
{\color{incolor}In [{\color{incolor}12}]:} \PY{c}{\PYZsh{}we imported ols (Oridnary Least Squares) in our import cell}
         \PY{c}{\PYZsh{}we named our columns x and y, now we need to tell the OLS function}
         \PY{c}{\PYZsh{}what we want to regress. The second argument is our DataFrame.}
         \PY{n}{NorrisLM} \PY{o}{=} \PY{n}{ols}\PY{p}{(}\PY{l+s}{\PYZsq{}}\PY{l+s}{y \PYZti{} x}\PY{l+s}{\PYZsq{}}\PY{p}{,} \PY{n}{NorrisFrame}\PY{p}{)}\PY{o}{.}\PY{n}{fit}\PY{p}{(}\PY{p}{)}
         \PY{k}{print} \PY{n}{NorrisLM}\PY{o}{.}\PY{n}{summary}\PY{p}{(}\PY{p}{)}
\end{Verbatim}

    \begin{Verbatim}[commandchars=\\\{\}]
OLS Regression Results                            
==============================================================================
Dep. Variable:                      y   R-squared:                       1.000
Model:                            OLS   Adj. R-squared:                  1.000
Method:                 Least Squares   F-statistic:                 5.436e+06
Date:                Sun, 12 Oct 2014   Prob (F-statistic):           4.65e-90
Time:                        20:57:41   Log-Likelihood:                -45.647
No. Observations:                  36   AIC:                             95.29
Df Residuals:                      34   BIC:                             98.46
Df Model:                           1                                         
==============================================================================
                 coef    std err          t      P>|t|      [95.0\% Conf. Int.]
------------------------------------------------------------------------------
Intercept     -0.2623      0.233     -1.127      0.268        -0.735     0.211
x              1.0021      0.000   2331.606      0.000         1.001     1.003
==============================================================================
Omnibus:                        2.696   Durbin-Watson:                   1.272
Prob(Omnibus):                  0.260   Jarque-Bera (JB):                1.566
Skew:                          -0.450   Prob(JB):                        0.457
Kurtosis:                       3.485   Cond. No.                         855.
==============================================================================
    \end{Verbatim}

    

    As you can see from the output table, the results provided by OLS are
far from the precision we need. Like R and SAS, we can extract more
decimal places. Some of these values can be extracted directly from the
linear model object and others have to be accessed through the
EstimatedParameters object.

    \begin{Verbatim}[commandchars=\\\{\}]
{\color{incolor}In [{\color{incolor}13}]:} \PY{c}{\PYZsh{}create an Estimated parameters object}
         \PY{n}{NorrisParams} \PY{o}{=} \PY{n}{NorrisLM}\PY{o}{.}\PY{n}{params}
\end{Verbatim}

    \begin{Verbatim}[commandchars=\\\{\}]
{\color{incolor}In [{\color{incolor}14}]:} \PY{c}{\PYZsh{}repr() converts our values to strings without losing truncating them}
         \PY{n}{PB0} \PY{o}{=} \PY{n+nb}{repr}\PY{p}{(}\PY{n}{NorrisParams}\PY{p}{[}\PY{l+m+mi}{0}\PY{p}{]}\PY{p}{)}
         \PY{n}{PB1} \PY{o}{=} \PY{n+nb}{repr}\PY{p}{(}\PY{n}{NorrisParams}\PY{p}{[}\PY{l+m+mi}{1}\PY{p}{]}\PY{p}{)}
         \PY{n}{PSTDofEstB0} \PY{o}{=} \PY{n+nb}{repr}\PY{p}{(}\PY{n}{NorrisLM}\PY{o}{.}\PY{n}{bse}\PY{p}{[}\PY{l+m+mi}{0}\PY{p}{]}\PY{p}{)}
         \PY{n}{PSTDofEstB1} \PY{o}{=} \PY{n+nb}{repr}\PY{p}{(}\PY{n}{NorrisLM}\PY{o}{.}\PY{n}{bse}\PY{p}{[}\PY{l+m+mi}{1}\PY{p}{]}\PY{p}{)}
         \PY{n}{Presstd} \PY{o}{=} \PY{n+nb}{repr}\PY{p}{(}\PY{n}{np}\PY{o}{.}\PY{n}{sqrt}\PY{p}{(}\PY{n}{NorrisLM}\PY{o}{.}\PY{n}{mse\PYZus{}resid}\PY{p}{)}\PY{p}{)}
         \PY{n}{PRsq} \PY{o}{=} \PY{n+nb}{repr}\PY{p}{(}\PY{n}{NorrisLM}\PY{o}{.}\PY{n}{rsquared}\PY{p}{)}
         \PY{n}{PModSS} \PY{o}{=} \PY{n+nb}{repr}\PY{p}{(}\PY{n}{NorrisLM}\PY{o}{.}\PY{n}{ess}\PY{p}{)}
         \PY{n}{PModMSE} \PY{o}{=} \PY{n+nb}{repr}\PY{p}{(}\PY{n}{NorrisLM}\PY{o}{.}\PY{n}{mse\PYZus{}model}\PY{p}{)}
         \PY{n}{PModSSResid} \PY{o}{=} \PY{n+nb}{repr}\PY{p}{(}\PY{n}{NorrisLM}\PY{o}{.}\PY{n}{ssr}\PY{p}{)}
         \PY{n}{PModMSEResid} \PY{o}{=} \PY{n+nb}{repr}\PY{p}{(}\PY{n}{NorrisLM}\PY{o}{.}\PY{n}{mse\PYZus{}resid}\PY{p}{)}
         \PY{n}{PFstat} \PY{o}{=} \PY{n+nb}{repr}\PY{p}{(}\PY{n}{NorrisLM}\PY{o}{.}\PY{n}{fvalue}\PY{p}{)}
         
         \PY{n}{PyVals} \PY{o}{=} \PY{n}{np}\PY{o}{.}\PY{n}{array}\PY{p}{(}\PY{p}{[}\PY{n}{PB0}\PY{p}{,} \PY{n}{PB1}\PY{p}{,} \PY{n}{PSTDofEstB0}\PY{p}{,} \PY{n}{PSTDofEstB1}\PY{p}{,}\PY{n}{Presstd}\PY{p}{,}\PY{n}{PRsq}\PY{p}{,}
                              \PY{n}{PModSS}\PY{p}{,}\PY{n}{PModMSE}\PY{p}{,} \PY{n}{PModSSResid}\PY{p}{,}\PY{n}{PModMSEResid}\PY{p}{,}\PY{n}{PFstat}\PY{p}{]}\PY{p}{)}
\end{Verbatim}

    Unforunately, there's no shortcut or easier explanation to what happened
in the cell above. Determining how to extract those values was the
result of a lot of time, dir() usage, and help documentation. The great
part is, once you've done it, you'll know how to repeat the process (as
we've done here by replicating my extractions).


    \subsubsection{7.2.4 Linear Regression and ANOVA values in R (using Rpy2)}


    Here's our first real look at R in the IPython Notebook. We can use this
notebook to extract the necessary values for our precision comparison
(SAS will be a different story).

    To prepare for working with R, I exported our Pandas DataFrame as a csv
file.

    \begin{Verbatim}[commandchars=\\\{\}]
{\color{incolor}In [{\color{incolor}15}]:} \PY{n}{NorrisFrame}\PY{o}{.}\PY{n}{to\PYZus{}csv}\PY{p}{(}\PY{l+s}{\PYZsq{}}\PY{l+s}{C:/Users/flunk\PYZus{}000/Desktop/CalPoly/IPythonNotebook/SeniorProject/data/NorrisFrame.txt}\PY{l+s}{\PYZsq{}}\PY{p}{)}
\end{Verbatim}

    As i'm sure you noticed during the Rpy2 installation, we need import and
load rpy2 to use it.

    \begin{Verbatim}[commandchars=\\\{\}]
{\color{incolor}In [{\color{incolor}16}]:} \PY{k+kn}{import} \PY{n+nn}{rpy2}
         \PY{o}{\PYZpc{}}\PY{k}{load\PYZus{}ext} \PY{n}{rpy2}\PY{o}{.}\PY{n}{ipython}
\end{Verbatim}

    Since this notebook is about using Python and the IPython Notebook, the
R and SAS data extractions will be brief.

To run R code in your notebook, start the line with \%R. If you want
execute an entire cell of R code, start the cell with \%\%R.

    \begin{Verbatim}[commandchars=\\\{\}]
{\color{incolor}In [{\color{incolor}15}]:} \PY{o}{\PYZpc{}\PYZpc{}}\PY{k}{R}
         \PY{c}{\PYZsh{}read in data}
         \PY{n}{setwd}\PY{p}{(}\PY{l+s}{\PYZdq{}}\PY{l+s}{C:/Users/flunk\PYZus{}000/Desktop/CalPoly/IPythonNotebook/SeniorProject/}\PY{l+s}{\PYZdq{}}\PY{p}{)}\PY{p}{;}
         \PY{n}{Norris} \PY{o}{=} \PY{n}{read}\PY{o}{.}\PY{n}{csv}\PY{p}{(}\PY{l+s}{\PYZsq{}}\PY{l+s}{data/NorrisFrame.txt}\PY{l+s}{\PYZsq{}}\PY{p}{,} \PY{n}{header}\PY{o}{=}\PY{n}{T}\PY{p}{)}\PY{p}{;}
         \PY{n}{NorrisFrame} \PY{o}{=} \PY{k}{as}\PY{o}{.}\PY{n}{data}\PY{o}{.}\PY{n}{frame}\PY{p}{(}\PY{n}{Norris}\PY{p}{)}\PY{p}{;}
\end{Verbatim}

    \begin{Verbatim}[commandchars=\\\{\}]
{\color{incolor}In [{\color{incolor}16}]:} \PY{o}{\PYZpc{}}\PY{k}{R} \PY{n}{head}\PY{p}{(}\PY{n}{NorrisFrame}\PY{p}{)}
\end{Verbatim}

            \begin{Verbatim}[commandchars=\\\{\}]
{\color{outcolor}Out[{\color{outcolor}16}]:}    X      y      x
         0  0    0.1    0.2
         1  1  338.8  337.4
         2  2  118.1  118.2
         3  3  888.0  884.6
         4  4    9.2   10.1
         5  5  228.1  226.5
\end{Verbatim}
        
    \begin{Verbatim}[commandchars=\\\{\}]
{\color{incolor}In [{\color{incolor}17}]:} \PY{o}{\PYZpc{}}\PY{k}{R} \PY{n}{tail}\PY{p}{(}\PY{n}{NorrisFrame}\PY{p}{)}
\end{Verbatim}

            \begin{Verbatim}[commandchars=\\\{\}]
{\color{outcolor}Out[{\color{outcolor}17}]:}     X      y      x
         0  30   10.2   11.1
         1  31  117.6  118.3
         2  32  228.9  229.2
         3  33  668.4  669.1
         4  34  449.2  448.9
         5  35    0.2    0.5
\end{Verbatim}
        
    We have verified the data was read in correctly.

    \begin{Verbatim}[commandchars=\\\{\}]
{\color{incolor}In [{\color{incolor}18}]:} \PY{o}{\PYZpc{}\PYZpc{}}\PY{k}{R}
         \PY{c}{\PYZsh{}analysis in R}
         \PY{n}{NorrisLM} \PY{o}{=} \PY{n}{lm}\PY{p}{(}\PY{n}{y} \PY{o}{\PYZti{}} \PY{n}{x}\PY{p}{,} \PY{n}{data}\PY{o}{=}\PY{n}{NorrisFrame}\PY{p}{)}\PY{p}{;}
         \PY{n}{NorrisResid} \PY{o}{=} \PY{n}{NorrisLM}\PY{err}{\PYZdl{}}\PY{n}{residuals}\PY{p}{;}
         \PY{n}{NorrisCoef} \PY{o}{=} \PY{n}{NorrisLM}\PY{err}{\PYZdl{}}\PY{n}{coefficients}\PY{p}{;} 
         \PY{n}{NorrisANOVA} \PY{o}{=} \PY{n}{aov}\PY{p}{(}\PY{n}{y} \PY{o}{\PYZti{}} \PY{n}{x}\PY{p}{,} \PY{n}{data}\PY{o}{=}\PY{n}{NorrisFrame}\PY{p}{)}\PY{p}{;}
\end{Verbatim}

    \begin{Verbatim}[commandchars=\\\{\}]
{\color{incolor}In [{\color{incolor}19}]:} \PY{o}{\PYZpc{}\PYZpc{}}\PY{k}{R}
         \PY{c}{\PYZsh{}obtaining precise numbers in R}
         \PY{k}{print}\PY{p}{(}\PY{n}{summary}\PY{p}{(}\PY{n}{NorrisANOVA}\PY{p}{)}\PY{p}{,} \PY{n}{digits} \PY{o}{=}\PY{l+m+mi}{20}\PY{p}{)}\PY{p}{;}
         \PY{k}{print}\PY{p}{(}\PY{n}{summary}\PY{p}{(}\PY{n}{NorrisLM}\PY{p}{)}\PY{p}{,} \PY{n}{digits} \PY{o}{=}\PY{l+m+mi}{20}\PY{p}{)}\PY{p}{;}
         \PY{k}{print}\PY{p}{(}\PY{n}{sqrt}\PY{p}{(}\PY{n}{deviance}\PY{p}{(}\PY{n}{NorrisANOVA}\PY{p}{)}\PY{o}{/}\PY{n}{df}\PY{o}{.}\PY{n}{residual}\PY{p}{(}\PY{n}{NorrisANOVA}\PY{p}{)}\PY{p}{)}\PY{p}{,} \PY{n}{digits} \PY{o}{=} \PY{l+m+mi}{20}\PY{p}{)}\PY{p}{;}
\end{Verbatim}

    Currently, these values print to the R kernel in your IPython Notebook
console: 

    There are different tricks you can experiment with in order to print to
the notebook instead of the console, but they're not the focus of this
section (and will soon be unnecessary). If you want to be able to copy
and paste the values we need, copy and paste my R code into R and run
it.

    \begin{Verbatim}[commandchars=\\\{\}]
{\color{incolor}In [{\color{incolor}17}]:} \PY{n}{RB0} \PY{o}{=} \PY{l+s}{\PYZsq{}}\PY{l+s}{\PYZhy{}0.26232307377411718}\PY{l+s}{\PYZsq{}}
         \PY{n}{RB1} \PY{o}{=} \PY{l+s}{\PYZsq{}}\PY{l+s}{1.00211681802045427}\PY{l+s}{\PYZsq{}}
         \PY{n}{RSTDofEstB0} \PY{o}{=} \PY{l+s}{\PYZsq{}}\PY{l+s}{0.23281823430115431}\PY{l+s}{\PYZsq{}}
         \PY{n}{RSTDofEstB1} \PY{o}{=} \PY{l+s}{\PYZsq{}}\PY{l+s}{0.00042979684819994}\PY{l+s}{\PYZsq{}}
         \PY{n}{Rresstd} \PY{o}{=} \PY{l+s}{\PYZsq{}}\PY{l+s}{0.88479639614437943784}\PY{l+s}{\PYZsq{}}
         \PY{n}{RRsq} \PY{o}{=} \PY{l+s}{\PYZsq{}}\PY{l+s}{0.999993745883712}\PY{l+s}{\PYZsq{}}
         \PY{n}{RModSS} \PY{o}{=} \PY{l+s}{\PYZsq{}}\PY{l+s}{4255954.13232369348406792}\PY{l+s}{\PYZsq{}}
         \PY{n}{RModMSE} \PY{o}{=} \PY{l+s}{\PYZsq{}}\PY{l+s}{4255954.13232369348406792}\PY{l+s}{\PYZsq{}}
         \PY{n}{RModSSResid} \PY{o}{=} \PY{l+s}{\PYZsq{}}\PY{l+s}{26.61739852942280038}\PY{l+s}{\PYZsq{}}
         \PY{n}{RModMSEResid} \PY{o}{=} \PY{l+s}{\PYZsq{}}\PY{l+s}{0.78286466263010001665}\PY{l+s}{\PYZsq{}}
         \PY{n}{RFstat} \PY{o}{=} \PY{l+s}{\PYZsq{}}\PY{l+s}{5436385.5407999996096}\PY{l+s}{\PYZsq{}}
         
         \PY{n}{RVals} \PY{o}{=} \PY{n}{np}\PY{o}{.}\PY{n}{array}\PY{p}{(}\PY{p}{[}\PY{n}{RB0}\PY{p}{,} \PY{n}{RB1}\PY{p}{,} \PY{n}{RSTDofEstB0}\PY{p}{,} \PY{n}{RSTDofEstB1}\PY{p}{,}\PY{n}{Rresstd}\PY{p}{,}\PY{n}{RRsq}\PY{p}{,}
                             \PY{n}{RModSS}\PY{p}{,}\PY{n}{RModMSE}\PY{p}{,} \PY{n}{RModSSResid}\PY{p}{,}\PY{n}{RModMSEResid}\PY{p}{,}\PY{n}{RFstat}\PY{p}{]}\PY{p}{)}
\end{Verbatim}


    \subsubsection{7.2.5 Linear Regression and ANOVA values in SAS}


    Here is my code for the SAS analysis and value extractions:

    \begin{Verbatim}[commandchars=\\\{\}]
{\color{incolor}In [{\color{incolor}33}]:} \PY{n}{OPTIONS} \PY{n}{NODATE} \PY{n}{NONUMBER} \PY{n}{CENTER} \PY{n}{LS}\PY{o}{=}\PY{l+m+mi}{160}\PY{p}{;}
         \PY{o}{*}\PY{n}{Removes} \PY{n}{the} \PY{n}{header} \PY{n}{information} \PY{o+ow}{and} \PY{n}{centers} \PY{n}{output}\PY{p}{;}
         \PY{n}{OPTIONS} \PY{n}{FORMDLIM}\PY{o}{=}\PY{l+s}{\PYZdq{}}\PY{l+s}{\PYZti{}}\PY{l+s}{\PYZdq{}}\PY{p}{;}	
         
         \PY{n}{data} \PY{n}{NorrisData}\PY{p}{;}
         \PY{n}{infile} \PY{l+s}{\PYZdq{}}\PY{l+s}{C:/Users/flunk\PYZus{}000/Desktop/CalPoly/IPythonNotebook/SeniorProject/data/NorrisFrameSAS.txt}\PY{l+s}{\PYZdq{}} \PY{n}{DLM}\PY{o}{=}\PY{l+s}{\PYZsq{}}\PY{l+s}{,}\PY{l+s}{\PYZsq{}}\PY{p}{;}
         \PY{n+nb}{input} \PY{n}{subject} \PY{n}{y} \PY{n}{x}\PY{p}{;}
         
         \PY{n}{ODS} \PY{n}{TRACE} \PY{n}{ON}\PY{p}{;}
         \PY{n}{proc} \PY{n}{glm} \PY{n}{data}\PY{o}{=} \PY{n}{NorrisData}\PY{p}{;}
         	\PY{n}{model} \PY{n}{y} \PY{o}{=} \PY{n}{x}\PY{p}{;}
         		\PY{n}{output} \PY{n}{out} \PY{o}{=} \PY{n}{linear\PYZus{}norris}\PY{p}{;}
         		\PY{n}{ODS} \PY{n}{output} \PY{n}{ParameterEstimates} \PY{o}{=} \PY{n}{pe}\PY{p}{;}
         		\PY{n}{ODS} \PY{n}{output} \PY{n}{Overallanova} \PY{o}{=} \PY{n}{anova}\PY{p}{;}
         		\PY{n}{ODS} \PY{n}{output} \PY{n}{fitstatistics} \PY{o}{=} \PY{n}{fs}\PY{p}{;}
         \PY{n}{run}\PY{p}{;}
         \PY{n}{ODS} \PY{n}{Trace} \PY{n}{off}\PY{p}{;}
         
         \PY{n}{proc} \PY{k}{print} \PY{n}{data}\PY{o}{=}\PY{n}{pe}\PY{p}{;}
         	\PY{n}{format} \PY{n}{estimate} \PY{l+m+mf}{20.19}
         		\PY{n}{stderr} \PY{l+m+mf}{20.19}
         		\PY{n}{probt} \PY{n}{pvalue20}\PY{o}{.}\PY{l+m+mi}{19}\PY{p}{;}
         	\PY{n}{run}\PY{p}{;}
             
         \PY{n}{proc} \PY{k}{print} \PY{n}{data}\PY{o}{=}\PY{n}{anova}\PY{p}{;}
         	\PY{n}{format} \PY{n}{SS} \PY{l+m+mf}{20.19}
         		\PY{n}{MS} \PY{l+m+mf}{20.19}
         		\PY{n}{fvalue} \PY{l+m+mf}{20.19}\PY{p}{;}
         	\PY{n}{run}\PY{p}{;}
             
         \PY{n}{proc} \PY{k}{print} \PY{n}{data}\PY{o}{=}\PY{n}{fs}
         	\PY{n}{format} 
         		\PY{n}{rsquare} \PY{l+m+mf}{20.19}
         		\PY{n}{rootmse} \PY{l+m+mf}{20.19}\PY{p}{;}
         	\PY{n}{run}\PY{p}{;}
\end{Verbatim}

    \begin{Verbatim}[commandchars=\\\{\}]
{\color{incolor}In [{\color{incolor}18}]:} \PY{n}{SB0} \PY{o}{=} \PY{l+s}{\PYZsq{}}\PY{l+s}{\PYZhy{}0.26232307377383600}\PY{l+s}{\PYZsq{}}
         \PY{n}{SB1} \PY{o}{=} \PY{l+s}{\PYZsq{}}\PY{l+s}{1.00211681802045000}\PY{l+s}{\PYZsq{}}
         \PY{n}{SSTDofEstB0} \PY{o}{=} \PY{l+s}{\PYZsq{}}\PY{l+s}{0.23281823431377200}\PY{l+s}{\PYZsq{}}
         \PY{n}{SSTDofEstB1} \PY{o}{=} \PY{l+s}{\PYZsq{}}\PY{l+s}{0.0004297968482232346}\PY{l+s}{\PYZsq{}}
         \PY{n}{Sresstd} \PY{o}{=} \PY{l+s}{\PYZsq{}}\PY{l+s}{0.884796396192334000}\PY{l+s}{\PYZsq{}}
         \PY{n}{SRsq} \PY{o}{=} \PY{l+s}{\PYZsq{}}\PY{l+s}{0.9999937458837110000}\PY{l+s}{\PYZsq{}}
         \PY{n}{SModSS} \PY{o}{=} \PY{l+s}{\PYZsq{}}\PY{l+s}{4255954.132323680000}\PY{l+s}{\PYZsq{}}
         \PY{n}{SModMSE} \PY{o}{=} \PY{l+s}{\PYZsq{}}\PY{l+s}{4255954.132323680000}\PY{l+s}{\PYZsq{}}
         \PY{n}{SModSSResid} \PY{o}{=} \PY{l+s}{\PYZsq{}}\PY{l+s}{26.61739853230800000}\PY{l+s}{\PYZsq{}}
         \PY{n}{SModMSEResid} \PY{o}{=} \PY{l+s}{\PYZsq{}}\PY{l+s}{0.782864662714941000}\PY{l+s}{\PYZsq{}}
         \PY{n}{SFstat} \PY{o}{=} \PY{l+s}{\PYZsq{}}\PY{l+s}{5436385.54020847000}\PY{l+s}{\PYZsq{}}
             
         \PY{n}{SASVals} \PY{o}{=} \PY{n}{np}\PY{o}{.}\PY{n}{array}\PY{p}{(}\PY{p}{[}\PY{n}{SB0}\PY{p}{,} \PY{n}{SB1}\PY{p}{,} \PY{n}{SSTDofEstB0}\PY{p}{,} \PY{n}{SSTDofEstB1}\PY{p}{,}\PY{n}{Sresstd}\PY{p}{,}\PY{n}{SRsq}\PY{p}{,}
                             \PY{n}{SModSS}\PY{p}{,}\PY{n}{SModMSE}\PY{p}{,} \PY{n}{SModSSResid}\PY{p}{,}\PY{n}{SModMSEResid}\PY{p}{,}\PY{n}{SFstat}\PY{p}{]}\PY{p}{)}
\end{Verbatim}


    \subsubsection{7.2.6 Testing Python's Precision against NIST, R, and SAS}


    I prefer to write functions for any task I have to repeat (and any task
I can get away with writing a function for). Let's look at the functions
we'll use to compare the precision of our programs.

    \begin{Verbatim}[commandchars=\\\{\}]
{\color{incolor}In [{\color{incolor}19}]:} \PY{c}{\PYZsh{}def allows us to define a function}
         \PY{c}{\PYZsh{}cert\PYZus{}val\PYZus{}lengths is the name of the function}
         \PY{c}{\PYZsh{}it creates an array of the lengths of NIST values for later precision comparisons}
         \PY{c}{\PYZsh{}the len() function determines the length of a string}
         \PY{c}{\PYZsh{}the dtype argument here returns an array of integer values}
         \PY{k}{def} \PY{n+nf}{cert\PYZus{}val\PYZus{}lengths}\PY{p}{(}\PY{n}{CertifiedValuesArray}\PY{p}{)}\PY{p}{:}
         
             \PY{c}{\PYZsh{}creates an array of zeros to store the lengths}
             \PY{n}{CertValLengths} \PY{o}{=} \PY{n}{np}\PY{o}{.}\PY{n}{zeros}\PY{p}{(}\PY{n+nb}{len}\PY{p}{(}\PY{n}{CertifiedValuesArray}\PY{p}{)}\PY{p}{,}\PY{n}{dtype}\PY{o}{=}\PY{n+nb}{int}\PY{p}{)}
             
             \PY{k}{for} \PY{n}{val} \PY{o+ow}{in} \PY{n+nb}{range}\PY{p}{(}\PY{n+nb}{len}\PY{p}{(}\PY{n}{CertifiedValuesArray}\PY{p}{)}\PY{p}{)}\PY{p}{:}
                 \PY{n}{CertValLengths}\PY{p}{[}\PY{n}{val}\PY{p}{]}\PY{o}{=} \PY{n+nb}{len}\PY{p}{(}\PY{n}{CertifiedValuesArray}\PY{p}{[}\PY{n}{val}\PY{p}{]}\PY{p}{)}
             \PY{k}{return} \PY{n}{CertValLengths}
\end{Verbatim}

    \begin{Verbatim}[commandchars=\\\{\}]
{\color{incolor}In [{\color{incolor}20}]:} \PY{c}{\PYZsh{}function requires three values, one I provide (from R, SAS, Py, etc)}
         \PY{c}{\PYZsh{}another from the NIST certified values}
         \PY{c}{\PYZsh{}third is the precision we are looking for}
         \PY{k}{def} \PY{n+nf}{nist\PYZus{}compare}\PY{p}{(}\PY{n}{MyValue}\PY{p}{,} \PY{n}{NISTValue}\PY{p}{,} \PY{n}{CertValLength}\PY{p}{)}\PY{p}{:}
             
             \PY{c}{\PYZsh{}converts the value to a list}
             \PY{n}{MyValueList} \PY{o}{=} \PY{n+nb}{list}\PY{p}{(}\PY{n}{MyValue}\PY{p}{)}
             
             \PY{n}{NISTValueList} \PY{o}{=} \PY{n+nb}{list}\PY{p}{(}\PY{n}{NISTValue}\PY{p}{)}
             
             \PY{n}{counter} \PY{o}{=} \PY{l+m+mi}{0}
             
             \PY{c}{\PYZsh{}checks to see how similar the values are}
             \PY{c}{\PYZsh{}the CertValLength allows us to ignore the extraneous precision}
             \PY{c}{\PYZsh{}added to the arrays by NumPy}
             \PY{k}{for} \PY{n}{val} \PY{o+ow}{in} \PY{n+nb}{range}\PY{p}{(}\PY{n}{CertValLength}\PY{p}{)}\PY{p}{:}
                 \PY{k}{if} \PY{n}{MyValueList}\PY{p}{[}\PY{n}{val}\PY{p}{]} \PY{o}{==} \PY{n}{NISTValueList}\PY{p}{[}\PY{n}{val}\PY{p}{]}\PY{p}{:}
                     \PY{n}{counter}\PY{o}{+}\PY{o}{=}\PY{l+m+mi}{1}
                 \PY{k}{else}\PY{p}{:}
                     \PY{k}{return} \PY{n}{counter}
         
             \PY{c}{\PYZsh{}returns how many values matches}
             \PY{k}{return} \PY{n}{counter}
\end{Verbatim}

    We have a way to compare the values, now let's write a function to
compare the arrays.

    \begin{Verbatim}[commandchars=\\\{\}]
{\color{incolor}In [{\color{incolor}21}]:} \PY{k}{def} \PY{n+nf}{array\PYZus{}compare}\PY{p}{(}\PY{n}{MyArray}\PY{p}{,}\PY{n}{NISTArray}\PY{p}{,}\PY{n}{LabelArray}\PY{p}{)}\PY{p}{:}
                 
             \PY{c}{\PYZsh{}create an empty array for value comparisons}
             \PY{n}{ValMatches} \PY{o}{=} \PY{n}{np}\PY{o}{.}\PY{n}{zeros}\PY{p}{(}\PY{n+nb}{len}\PY{p}{(}\PY{n}{NISTArray}\PY{p}{)}\PY{p}{,}\PY{n}{dtype}\PY{o}{=}\PY{n+nb}{int}\PY{p}{)}
             
             \PY{c}{\PYZsh{}uses our first function to create the lengths of the certified values}
             \PY{n}{CertValLengths} \PY{o}{=} \PY{n}{cert\PYZus{}val\PYZus{}lengths}\PY{p}{(}\PY{n}{NISTArray}\PY{p}{)}
             
             \PY{k}{for} \PY{n}{val} \PY{o+ow}{in} \PY{n+nb}{range}\PY{p}{(}\PY{n+nb}{len}\PY{p}{(}\PY{n}{LabelArray}\PY{p}{)}\PY{p}{)}\PY{p}{:}
                 
                 \PY{c}{\PYZsh{}compares the values using the previous function}
                 \PY{n}{ValMatch} \PY{o}{=} \PY{n}{nist\PYZus{}compare}\PY{p}{(}\PY{n}{MyArray}\PY{p}{[}\PY{n}{val}\PY{p}{]}\PY{p}{,}\PY{n}{NISTArray}\PY{p}{[}\PY{n}{val}\PY{p}{]}\PY{p}{,}\PY{n}{CertValLengths}\PY{p}{[}\PY{n}{val}\PY{p}{]}\PY{p}{)}
                 
                 \PY{c}{\PYZsh{}prints the comparison and uses our pre\PYZhy{}determined precision}
                 \PY{k}{print}\PY{p}{(}\PY{n}{LabelArray}\PY{p}{[}\PY{n}{val}\PY{p}{]}\PY{p}{,} \PY{n}{ValMatch}\PY{p}{,}\PY{l+s}{\PYZsq{}}\PY{l+s}{of}\PY{l+s}{\PYZsq{}}\PY{p}{,}\PY{n}{CertValLengths}\PY{p}{[}\PY{n}{val}\PY{p}{]}\PY{p}{)}
                 
                 \PY{c}{\PYZsh{}stores the values in our empty array}
                 \PY{n}{ValMatches}\PY{p}{[}\PY{n}{val}\PY{p}{]} \PY{o}{=} \PY{n}{ValMatch}
             
             \PY{c}{\PYZsh{}returns the precision we were looking for}
             \PY{k}{return} \PY{n}{ValMatches}    
\end{Verbatim}

    \begin{Verbatim}[commandchars=\\\{\}]
{\color{incolor}In [{\color{incolor}22}]:} \PY{c}{\PYZsh{}create an array of the values we want to compare}
         \PY{n}{NorrisLabels} \PY{o}{=} \PY{n}{np}\PY{o}{.}\PY{n}{array}\PY{p}{(}\PY{p}{[}\PY{l+s}{\PYZsq{}}\PY{l+s}{Beta0:}\PY{l+s}{\PYZsq{}}\PY{p}{,}\PY{l+s}{\PYZsq{}}\PY{l+s}{Beta1:}\PY{l+s}{\PYZsq{}}\PY{p}{,}\PY{l+s}{\PYZsq{}}\PY{l+s}{STDofEstimateB0:}\PY{l+s}{\PYZsq{}}\PY{p}{,}
                                    \PY{l+s}{\PYZsq{}}\PY{l+s}{STDofEstimateB1}\PY{l+s}{\PYZsq{}}\PY{p}{,}\PY{l+s}{\PYZsq{}}\PY{l+s}{resstd:}\PY{l+s}{\PYZsq{}}\PY{p}{,}\PY{l+s}{\PYZsq{}}\PY{l+s}{R\PYZhy{}sq:}\PY{l+s}{\PYZsq{}}\PY{p}{,}
                                    \PY{l+s}{\PYZsq{}}\PY{l+s}{Model SS:}\PY{l+s}{\PYZsq{}}\PY{p}{,}\PY{l+s}{\PYZsq{}}\PY{l+s}{Model MS:}\PY{l+s}{\PYZsq{}}\PY{p}{,}\PY{l+s}{\PYZsq{}}\PY{l+s}{Model SSResid:}\PY{l+s}{\PYZsq{}}\PY{p}{,}
                                    \PY{l+s}{\PYZsq{}}\PY{l+s}{Model MSResid:}\PY{l+s}{\PYZsq{}}\PY{p}{,}\PY{l+s}{\PYZsq{}}\PY{l+s}{F\PYZhy{}stat:}\PY{l+s}{\PYZsq{}}\PY{p}{]}\PY{p}{)}
\end{Verbatim}

    Let's look at how each program's output compared to the NIST Certified
Values.

    \begin{Verbatim}[commandchars=\\\{\}]
{\color{incolor}In [{\color{incolor}23}]:} \PY{n}{R} \PY{o}{=} \PY{n}{array\PYZus{}compare}\PY{p}{(}\PY{n}{RVals}\PY{p}{,}\PY{n}{CertVals}\PY{p}{,}\PY{n}{NorrisLabels}\PY{p}{)}
\end{Verbatim}

    \begin{Verbatim}[commandchars=\\\{\}]
('Beta0:', 15, 'of', 18)
('Beta1:', 16, 'of', 16)
('STDofEstimateB0:', 16, 'of', 17)
('STDofEstimateB1', 18, 'of', 20)
('resstd:', 16, 'of', 17)
('R-sq:', 17, 'of', 17)
('Model SS:', 16, 'of', 16)
('Model MS:', 16, 'of', 16)
('Model SSResid:', 15, 'of', 16)
('Model MSResid:', 14, 'of', 17)
('F-stat:', 13, 'of', 16)
    \end{Verbatim}

    \begin{Verbatim}[commandchars=\\\{\}]
{\color{incolor}In [{\color{incolor}24}]:} \PY{n}{SAS} \PY{o}{=} \PY{n}{array\PYZus{}compare}\PY{p}{(}\PY{n}{SASVals}\PY{p}{,}\PY{n}{CertVals}\PY{p}{,}\PY{n}{NorrisLabels}\PY{p}{)}
\end{Verbatim}

    \begin{Verbatim}[commandchars=\\\{\}]
('Beta0:', 14, 'of', 18)
('Beta1:', 16, 'of', 16)
('STDofEstimateB0:', 12, 'of', 17)
('STDofEstimateB1', 14, 'of', 20)
('resstd:', 12, 'of', 17)
('R-sq:', 16, 'of', 17)
('Model SS:', 15, 'of', 16)
('Model MS:', 15, 'of', 16)
('Model SSResid:', 10, 'of', 16)
('Model MSResid:', 11, 'of', 17)
('F-stat:', 11, 'of', 16)
    \end{Verbatim}

    \begin{Verbatim}[commandchars=\\\{\}]
{\color{incolor}In [{\color{incolor}25}]:} \PY{n}{Py} \PY{o}{=} \PY{n}{array\PYZus{}compare}\PY{p}{(}\PY{n}{PyVals}\PY{p}{,}\PY{n}{CertVals}\PY{p}{,}\PY{n}{NorrisLabels}\PY{p}{)}
\end{Verbatim}

    \begin{Verbatim}[commandchars=\\\{\}]
('Beta0:', 16, 'of', 18)
('Beta1:', 16, 'of', 16)
('STDofEstimateB0:', 16, 'of', 17)
('STDofEstimateB1', 18, 'of', 20)
('resstd:', 16, 'of', 17)
('R-sq:', 16, 'of', 17)
('Model SS:', 16, 'of', 16)
('Model MS:', 16, 'of', 16)
('Model SSResid:', 15, 'of', 16)
('Model MSResid:', 15, 'of', 17)
('F-stat:', 14, 'of', 16)
    \end{Verbatim}

    \begin{Verbatim}[commandchars=\\\{\}]
{\color{incolor}In [{\color{incolor}26}]:} \PY{n}{R}\PY{p}{,}\PY{n}{SAS}\PY{p}{,}\PY{n}{Py}\PY{p}{,}\PY{n}{cert\PYZus{}val\PYZus{}lengths}\PY{p}{(}\PY{n}{CertVals}\PY{p}{)}
\end{Verbatim}

            \begin{Verbatim}[commandchars=\\\{\}]
{\color{outcolor}Out[{\color{outcolor}26}]:} (array([15, 16, 16, 18, 16, 17, 16, 16, 15, 14, 13]),
          array([14, 16, 12, 14, 12, 16, 15, 15, 10, 11, 11]),
          array([16, 16, 16, 18, 16, 16, 16, 16, 15, 15, 14]),
          array([18, 16, 17, 20, 17, 17, 16, 16, 16, 17, 16]))
\end{Verbatim}
        
    We can do as little or as much as we want with this data (such as write
functions to compare these integers and provide a rating, or determine
if there are common values, such as SSE, that were rounded and used in
other calculations yielding less a lower precision).

We can also use this information to pick the program that best matches
the task at hand. For instance, R or Python did a great job of matching
18 values of Standard Deviation of Estimate for Beta1, while SAS ony
matched 14 values. If 18 is your target precision, you could use either
R or Python.

The important part is that upon visual inspection, Python appears to be
the closest to the Certified values for Linear Regression, with R close
behind, and SAS in third place.

    We can use Numpy and matplotlib to verify if my visual inspection is
correct:

    \begin{Verbatim}[commandchars=\\\{\}]
{\color{incolor}In [{\color{incolor}28}]:} \PY{c}{\PYZsh{}open a figure and add the axes}
         \PY{n}{hist} \PY{o}{=} \PY{n}{plt}\PY{o}{.}\PY{n}{figure}\PY{p}{(}\PY{n}{figsize}\PY{o}{=}\PY{p}{(}\PY{l+m+mi}{16}\PY{p}{,}\PY{l+m+mi}{4}\PY{p}{)}\PY{p}{)}
         \PY{n}{gs} \PY{o}{=} \PY{n}{GridSpec}\PY{p}{(}\PY{l+m+mi}{1}\PY{p}{,}\PY{l+m+mi}{2}\PY{p}{)}
         \PY{n}{axis} \PY{o}{=} \PY{n}{hist}\PY{o}{.}\PY{n}{add\PYZus{}subplot}\PY{p}{(}\PY{n}{gs}\PY{p}{[}\PY{l+m+mi}{0}\PY{p}{,}\PY{l+m+mi}{0}\PY{p}{]}\PY{p}{)}
         
         \PY{n}{programs}\PY{o}{=}\PY{l+m+mi}{4}
         \PY{c}{\PYZsh{}create array of bar values}
         \PY{n}{Matches} \PY{o}{=} \PY{p}{[}\PY{n}{np}\PY{o}{.}\PY{n}{sum}\PY{p}{(}\PY{n}{R}\PY{p}{)}\PY{p}{,}\PY{n}{np}\PY{o}{.}\PY{n}{sum}\PY{p}{(}\PY{n}{SAS}\PY{p}{)}\PY{p}{,}\PY{n}{np}\PY{o}{.}\PY{n}{sum}\PY{p}{(}\PY{n}{Py}\PY{p}{)}\PY{p}{,}\PY{n}{np}\PY{o}{.}\PY{n}{sum}\PY{p}{(}\PY{n}{cert\PYZus{}val\PYZus{}lengths}\PY{p}{(}\PY{n}{CertVals}\PY{p}{)}\PY{p}{)}\PY{p}{]}
         
         \PY{c}{\PYZsh{}location of bars on plot}
         \PY{n}{loc} \PY{o}{=} \PY{n}{np}\PY{o}{.}\PY{n}{arange}\PY{p}{(}\PY{n}{programs}\PY{p}{)}
         
         \PY{n}{bars} \PY{o}{=} \PY{n}{axis}\PY{o}{.}\PY{n}{bar}\PY{p}{(}\PY{n}{loc}\PY{p}{,} \PY{n}{Matches}\PY{p}{)}
         
         \PY{n}{axis}\PY{o}{.}\PY{n}{set\PYZus{}ylim}\PY{p}{(}\PY{l+m+mi}{0}\PY{p}{,}\PY{l+m+mi}{200}\PY{p}{)}
         \PY{n}{axis}\PY{o}{.}\PY{n}{set\PYZus{}ylabel}\PY{p}{(}\PY{l+s}{\PYZsq{}}\PY{l+s}{\PYZsh{} of matching values}\PY{l+s}{\PYZsq{}}\PY{p}{)}
         \PY{n}{axis}\PY{o}{.}\PY{n}{set\PYZus{}title}\PY{p}{(}\PY{l+s}{\PYZsq{}}\PY{l+s}{Precision of R,SAS,Py}\PY{l+s}{\PYZsq{}}\PY{p}{)}
         \PY{n}{axis}\PY{o}{.}\PY{n}{set\PYZus{}xticks}\PY{p}{(}\PY{n}{loc}\PY{o}{+}\PY{o}{.}\PY{l+m+mi}{35}\PY{p}{)}
         \PY{n}{XNames} \PY{o}{=} \PY{n}{axis}\PY{o}{.}\PY{n}{set\PYZus{}xticklabels}\PY{p}{(}\PY{p}{[}\PY{l+s}{\PYZsq{}}\PY{l+s}{R}\PY{l+s}{\PYZsq{}}\PY{p}{,} \PY{l+s}{\PYZsq{}}\PY{l+s}{SAS}\PY{l+s}{\PYZsq{}}\PY{p}{,} \PY{l+s}{\PYZsq{}}\PY{l+s}{Python}\PY{l+s}{\PYZsq{}}\PY{p}{,}\PY{l+s}{\PYZsq{}}\PY{l+s}{NIST}\PY{l+s}{\PYZsq{}}\PY{p}{]}\PY{p}{)}
         
         \PY{n}{axis2} \PY{o}{=} \PY{n}{hist}\PY{o}{.}\PY{n}{add\PYZus{}subplot}\PY{p}{(}\PY{n}{gs}\PY{p}{[}\PY{l+m+mi}{0}\PY{p}{,}\PY{l+m+mi}{1}\PY{p}{]}\PY{p}{)}
         \PY{n}{bars2} \PY{o}{=} \PY{n}{axis2}\PY{o}{.}\PY{n}{bar}\PY{p}{(}\PY{n}{loc}\PY{p}{,} \PY{n}{Matches}\PY{p}{)}
         
         \PY{n}{axis2}\PY{o}{.}\PY{n}{set\PYZus{}ylim}\PY{p}{(}\PY{l+m+mi}{145}\PY{p}{,}\PY{l+m+mi}{200}\PY{p}{)}
         \PY{n}{axis2}\PY{o}{.}\PY{n}{set\PYZus{}ylabel}\PY{p}{(}\PY{l+s}{\PYZsq{}}\PY{l+s}{\PYZsh{} of matching values}\PY{l+s}{\PYZsq{}}\PY{p}{)}
         \PY{n}{axis2}\PY{o}{.}\PY{n}{set\PYZus{}title}\PY{p}{(}\PY{l+s}{\PYZsq{}}\PY{l+s}{Precision of R,SAS,Py}\PY{l+s}{\PYZsq{}}\PY{p}{)}
         \PY{n}{axis2}\PY{o}{.}\PY{n}{set\PYZus{}xticks}\PY{p}{(}\PY{n}{loc}\PY{o}{+}\PY{o}{.}\PY{l+m+mi}{35}\PY{p}{)}
         \PY{n}{XNames} \PY{o}{=} \PY{n}{axis2}\PY{o}{.}\PY{n}{set\PYZus{}xticklabels}\PY{p}{(}\PY{p}{[}\PY{l+s}{\PYZsq{}}\PY{l+s}{R}\PY{l+s}{\PYZsq{}}\PY{p}{,} \PY{l+s}{\PYZsq{}}\PY{l+s}{SAS}\PY{l+s}{\PYZsq{}}\PY{p}{,} \PY{l+s}{\PYZsq{}}\PY{l+s}{Python}\PY{l+s}{\PYZsq{}}\PY{p}{,}\PY{l+s}{\PYZsq{}}\PY{l+s}{NIST}\PY{l+s}{\PYZsq{}}\PY{p}{]}\PY{p}{)}
         \PY{n}{Matches}
\end{Verbatim}

            \begin{Verbatim}[commandchars=\\\{\}]
{\color{outcolor}Out[{\color{outcolor}28}]:} [172, 146, 174, 186]
\end{Verbatim}
        
    \begin{center}
    \adjustimage{max size={0.9\linewidth}{0.9\paperheight}}{SeniorProjectPDF_files/SeniorProjectPDF_143_1.png}
    \end{center}
    { \hspace*{\fill} \\}
    
    Over the 11 values of concern, Python was more precise than R by 2
decimal places. SAS was 26,28 decimals less precise than R,Py
(respectively).

    Important Note: This does not mean that SAS miscalculated the values. If
you return to the SAS output, you will see that each SAS value has
trailing zeros instead of additonal digits. SAS simply doesn't provide
the level of precision we sought to completely match the certified
values in this exercise.


    \subsubsection{7.2.7 Plotting Norris data with Mathplotlib}


    The NIST graphics aren't as detailed as I would like, but they give us
something to compare the Python graphics to.

Norris Regression: 

    \begin{Verbatim}[commandchars=\\\{\}]
{\color{incolor}In [{\color{incolor}70}]:} \PY{c}{\PYZsh{}using matplotlib}
         \PY{c}{\PYZsh{}plots the points}
         \PY{n}{plt}\PY{o}{.}\PY{n}{scatter}\PY{p}{(}\PY{n}{NorrisFrame}\PY{p}{[}\PY{l+s}{\PYZsq{}}\PY{l+s}{y}\PY{l+s}{\PYZsq{}}\PY{p}{]}\PY{p}{,}\PY{n}{NorrisFrame}\PY{p}{[}\PY{l+s}{\PYZsq{}}\PY{l+s}{x}\PY{l+s}{\PYZsq{}}\PY{p}{]}\PY{p}{)}
         
         \PY{c}{\PYZsh{}draws the lines}
         \PY{n}{plt}\PY{o}{.}\PY{n}{plot}\PY{p}{(}\PY{n}{NorrisFrame}\PY{p}{[}\PY{l+s}{\PYZsq{}}\PY{l+s}{y}\PY{l+s}{\PYZsq{}}\PY{p}{]}\PY{p}{,}\PY{n}{NorrisFrame}\PY{p}{[}\PY{l+s}{\PYZsq{}}\PY{l+s}{x}\PY{l+s}{\PYZsq{}}\PY{p}{]}\PY{p}{)}
         
         \PY{c}{\PYZsh{}labels the X axis and Y axis}
         \PY{n}{plt}\PY{o}{.}\PY{n}{xlabel}\PY{p}{(}\PY{l+s}{\PYZsq{}}\PY{l+s}{Norris x values}\PY{l+s}{\PYZsq{}}\PY{p}{)}
         \PY{n}{plt}\PY{o}{.}\PY{n}{ylabel}\PY{p}{(}\PY{l+s}{\PYZsq{}}\PY{l+s}{Norris y values}\PY{l+s}{\PYZsq{}}\PY{p}{)}\PY{p}{;}
\end{Verbatim}

    \begin{center}
    \adjustimage{max size={0.9\linewidth}{0.9\paperheight}}{SeniorProjectPDF_files/SeniorProjectPDF_148_0.png}
    \end{center}
    { \hspace*{\fill} \\}
    
    Looks good. Let's check the residual plot:

    \begin{Verbatim}[commandchars=\\\{\}]
{\color{incolor}In [{\color{incolor}71}]:} \PY{c}{\PYZsh{}residual plot}
         \PY{n}{plt}\PY{o}{.}\PY{n}{scatter}\PY{p}{(}\PY{n}{NorrisFrame}\PY{p}{[}\PY{l+s}{\PYZsq{}}\PY{l+s}{x}\PY{l+s}{\PYZsq{}}\PY{p}{]}\PY{p}{,}\PY{n}{NorrisLM}\PY{o}{.}\PY{n}{resid}\PY{p}{)}
         
         \PY{c}{\PYZsh{}adds a horizontal line at y=0}
         \PY{n}{plt}\PY{o}{.}\PY{n}{axhline}\PY{p}{(}\PY{p}{)}
         \PY{n}{plt}\PY{o}{.}\PY{n}{xlabel}\PY{p}{(}\PY{l+s}{\PYZsq{}}\PY{l+s}{Norris x values}\PY{l+s}{\PYZsq{}}\PY{p}{)}
         \PY{n}{plt}\PY{o}{.}\PY{n}{ylabel}\PY{p}{(}\PY{l+s}{\PYZsq{}}\PY{l+s}{Residuals from Linear Regression on NorrisData}\PY{l+s}{\PYZsq{}}\PY{p}{)}
         
         \PY{c}{\PYZsh{}adds a title}
         \PY{n}{plt}\PY{o}{.}\PY{n}{title}\PY{p}{(}\PY{l+s}{\PYZsq{}}\PY{l+s}{Residuals vs. x}\PY{l+s}{\PYZsq{}}\PY{p}{)}\PY{p}{;}
\end{Verbatim}

    \begin{center}
    \adjustimage{max size={0.9\linewidth}{0.9\paperheight}}{SeniorProjectPDF_files/SeniorProjectPDF_150_0.png}
    \end{center}
    { \hspace*{\fill} \\}
    
    As much as we can tell, it appears the graphs are the same. We have
successfully created both of the provided NIST graphics. This is the end
of the Linear Regression Analysis on the Norris dataset.


    \subsection{7.3 ANOVA: SiR Dataset}



    \subsubsection{7.3.1 Reading in and preparing data from an ASCII webpage}


    For a detailed explanation of this process, refer to Section 7.2.1.

    \begin{Verbatim}[commandchars=\\\{\}]
{\color{incolor}In [{\color{incolor}83}]:} \PY{n}{SiRurl} \PY{o}{=} \PY{l+s}{\PYZsq{}}\PY{l+s}{http://www.itl.nist.gov/div898/strd/anova/SiRstv.dat}\PY{l+s}{\PYZsq{}}
         \PY{n+nb}{open}\PY{p}{(}\PY{l+s}{\PYZsq{}}\PY{l+s}{data/SiRstv.dat}\PY{l+s}{\PYZsq{}}\PY{p}{,}\PY{l+s}{\PYZsq{}}\PY{l+s}{wb}\PY{l+s}{\PYZsq{}}\PY{p}{)}\PY{o}{.}\PY{n}{write}\PY{p}{(}\PY{n}{ul}\PY{o}{.}\PY{n}{urlopen}\PY{p}{(}\PY{n}{SiRurl}\PY{p}{)}\PY{o}{.}\PY{n}{read}\PY{p}{(}\PY{p}{)}\PY{p}{)}
\end{Verbatim}

    \begin{Verbatim}[commandchars=\\\{\}]
{\color{incolor}In [{\color{incolor}84}]:} \PY{n}{SiRData} \PY{o}{=} \PY{n}{np}\PY{o}{.}\PY{n}{loadtxt}\PY{p}{(}\PY{l+s}{\PYZsq{}}\PY{l+s}{data/SiRstv.dat}\PY{l+s}{\PYZsq{}}\PY{p}{,}\PY{n}{skiprows}\PY{o}{=}\PY{l+m+mi}{60}\PY{p}{)}
         \PY{n}{SiRFrame} \PY{o}{=} \PY{n}{pd}\PY{o}{.}\PY{n}{DataFrame}\PY{p}{(}\PY{n}{SiRData}\PY{p}{,} \PY{n}{columns}\PY{o}{=}\PY{p}{[}\PY{l+s}{\PYZsq{}}\PY{l+s}{Instrument}\PY{l+s}{\PYZsq{}}\PY{p}{,}\PY{l+s}{\PYZsq{}}\PY{l+s}{Resistance}\PY{l+s}{\PYZsq{}}\PY{p}{]}\PY{p}{)}
\end{Verbatim}

    \begin{Verbatim}[commandchars=\\\{\}]
{\color{incolor}In [{\color{incolor}85}]:} \PY{n}{SiRFrame}\PY{o}{.}\PY{n}{head}\PY{p}{(}\PY{p}{)}
\end{Verbatim}

            \begin{Verbatim}[commandchars=\\\{\}]
{\color{outcolor}Out[{\color{outcolor}85}]:}    Instrument  Resistance
         0           1    196.3052
         1           1    196.1240
         2           1    196.1890
         3           1    196.2569
         4           1    196.3403
\end{Verbatim}
        
    \begin{Verbatim}[commandchars=\\\{\}]
{\color{incolor}In [{\color{incolor}86}]:} \PY{n}{SiRFrame}\PY{o}{.}\PY{n}{tail}\PY{p}{(}\PY{p}{)}
\end{Verbatim}

            \begin{Verbatim}[commandchars=\\\{\}]
{\color{outcolor}Out[{\color{outcolor}86}]:}     Instrument  Resistance
         20           5    196.2119
         21           5    196.1051
         22           5    196.1850
         23           5    196.0052
         24           5    196.2090
\end{Verbatim}
        
    Data was read in correctly.


    \subsubsection{7.3.2 NIST Certified Values}


    For a detailed explanation of this process, refer to Section 7.2.2.

    The Certified values we are trying to match can be found at:
http://www.itl.nist.gov/div898/strd/anova/SiRstv\_cv.html

    \begin{Verbatim}[commandchars=\\\{\}]
{\color{incolor}In [{\color{incolor}87}]:} \PY{n}{Certresstd} \PY{o}{=} \PY{l+s}{\PYZsq{}}\PY{l+s}{0.104076068334656}\PY{l+s}{\PYZsq{}}
         \PY{n}{CertRsq} \PY{o}{=} \PY{l+s}{\PYZsq{}}\PY{l+s}{0.190999039051129}\PY{l+s}{\PYZsq{}}
         \PY{n}{CertModSS} \PY{o}{=} \PY{l+s}{\PYZsq{}}\PY{l+s}{0.0511462616000000}\PY{l+s}{\PYZsq{}}
         \PY{n}{CertModMSE} \PY{o}{=} \PY{l+s}{\PYZsq{}}\PY{l+s}{0.0127865654000000}\PY{l+s}{\PYZsq{}}
         \PY{n}{CertModSSResid} \PY{o}{=} \PY{l+s}{\PYZsq{}}\PY{l+s}{0.216636560000000}\PY{l+s}{\PYZsq{}}
         \PY{n}{CertModMSEResid} \PY{o}{=} \PY{l+s}{\PYZsq{}}\PY{l+s}{0.0108318280000000}\PY{l+s}{\PYZsq{}}
         \PY{n}{CertFstat} \PY{o}{=} \PY{l+s}{\PYZsq{}}\PY{l+s}{1.18046237440255}\PY{l+s}{\PYZsq{}}
         
         \PY{n}{SiRCertVals} \PY{o}{=} \PY{n}{np}\PY{o}{.}\PY{n}{array}\PY{p}{(}\PY{p}{[}\PY{n}{Certresstd}\PY{p}{,}\PY{n}{CertRsq}\PY{p}{,}\PY{n}{CertModSS}\PY{p}{,}\PY{n}{CertModMSE}\PY{p}{,}
                                 \PY{n}{CertModSSResid}\PY{p}{,}\PY{n}{CertModMSEResid}\PY{p}{,}\PY{n}{CertFstat}\PY{p}{]}\PY{p}{)}
\end{Verbatim}


    \subsubsection{7.3.3 ANOVA values in Python}


    For a detailed explanation of this process, refer to Section 7.2.3.

    \begin{Verbatim}[commandchars=\\\{\}]
{\color{incolor}In [{\color{incolor}88}]:} \PY{c}{\PYZsh{}In our formula argument, since we are analyzing Resistence by intstrument}
         \PY{c}{\PYZsh{}we need to treat instrument as a categorical variable with C()}
         \PY{n}{SiRLM} \PY{o}{=} \PY{n}{ols}\PY{p}{(}\PY{l+s}{\PYZsq{}}\PY{l+s}{Resistance \PYZti{} C(Instrument)}\PY{l+s}{\PYZsq{}}\PY{p}{,} \PY{n}{SiRFrame}\PY{p}{)}\PY{o}{.}\PY{n}{fit}\PY{p}{(}\PY{p}{)}
         
         \PY{c}{\PYZsh{}you can chose to use the anova\PYZus{}lm function}
         \PY{c}{\PYZsh{}SiRANOVA = anova\PYZus{}lm(SiRLM)}
\end{Verbatim}

    \begin{Verbatim}[commandchars=\\\{\}]
{\color{incolor}In [{\color{incolor}89}]:} \PY{n}{Pyresstd} \PY{o}{=} \PY{n+nb}{repr}\PY{p}{(}\PY{n}{np}\PY{o}{.}\PY{n}{sqrt}\PY{p}{(}\PY{n}{SiRLM}\PY{o}{.}\PY{n}{mse\PYZus{}resid}\PY{p}{)}\PY{p}{)}
         \PY{n}{PyRsq} \PY{o}{=} \PY{n+nb}{repr}\PY{p}{(}\PY{n}{SiRLM}\PY{o}{.}\PY{n}{rsquared}\PY{p}{)}
         \PY{n}{PyModSS} \PY{o}{=} \PY{n+nb}{repr}\PY{p}{(}\PY{n}{SiRLM}\PY{o}{.}\PY{n}{ess}\PY{p}{)}
         \PY{n}{PyModMSE} \PY{o}{=} \PY{n+nb}{repr}\PY{p}{(}\PY{n}{SiRLM}\PY{o}{.}\PY{n}{mse\PYZus{}model}\PY{p}{)}
         \PY{n}{PyModSSResid} \PY{o}{=} \PY{n+nb}{repr}\PY{p}{(}\PY{n}{SiRLM}\PY{o}{.}\PY{n}{ssr}\PY{p}{)}
         \PY{n}{PyModMSEResid} \PY{o}{=} \PY{n+nb}{repr}\PY{p}{(}\PY{n}{SiRLM}\PY{o}{.}\PY{n}{mse\PYZus{}resid}\PY{p}{)}
         \PY{n}{PyFstat} \PY{o}{=} \PY{n+nb}{repr}\PY{p}{(}\PY{n}{SiRLM}\PY{o}{.}\PY{n}{fvalue}\PY{p}{)}
         
         \PY{n}{SiRPyVals} \PY{o}{=} \PY{n}{np}\PY{o}{.}\PY{n}{array}\PY{p}{(}\PY{p}{[}\PY{n}{Pyresstd}\PY{p}{,} \PY{n}{PyRsq}\PY{p}{,} \PY{n}{PyModSS}\PY{p}{,} \PY{n}{PyModMSE}\PY{p}{,}\PY{n}{PyModSSResid}\PY{p}{,}
                               \PY{n}{PyModMSEResid}\PY{p}{,}\PY{n}{PyFstat}\PY{p}{]}\PY{p}{)}
\end{Verbatim}


    \subsubsection{7.3.4 ANOVA values in R}


    For a detailed explanation of this process, refer to Section 7.2.4.

    \begin{Verbatim}[commandchars=\\\{\}]
{\color{incolor}In [{\color{incolor}90}]:} \PY{n}{SiRFrame}\PY{o}{.}\PY{n}{to\PYZus{}csv}\PY{p}{(}\PY{l+s}{\PYZsq{}}\PY{l+s}{C:/Users/flunk\PYZus{}000/Desktop/CalPoly/IPythonNotebook/SeniorProject/data/SiRFrame.txt}\PY{l+s}{\PYZsq{}}\PY{p}{)}
\end{Verbatim}

    \begin{Verbatim}[commandchars=\\\{\}]
{\color{incolor}In [{\color{incolor}91}]:} \PY{k+kn}{import} \PY{n+nn}{rpy2}
         \PY{o}{\PYZpc{}}\PY{k}{load\PYZus{}ext} \PY{n}{rpy2}\PY{o}{.}\PY{n}{ipython}
\end{Verbatim}

    \begin{Verbatim}[commandchars=\\\{\}]
The rpy2.ipython extension is already loaded. To reload it, use:
  \%reload\_ext rpy2.ipython
    \end{Verbatim}

    \begin{Verbatim}[commandchars=\\\{\}]
{\color{incolor}In [{\color{incolor}92}]:} \PY{o}{\PYZpc{}\PYZpc{}}\PY{k}{R}
         \PY{c}{\PYZsh{}read in data}
         \PY{n}{setwd}\PY{p}{(}\PY{l+s}{\PYZdq{}}\PY{l+s}{C:/Users/flunk\PYZus{}000/Desktop/CalPoly/IPythonNotebook/SeniorProject/}\PY{l+s}{\PYZdq{}}\PY{p}{)}\PY{p}{;}
         \PY{n}{SiR} \PY{o}{=} \PY{n}{read}\PY{o}{.}\PY{n}{csv}\PY{p}{(}\PY{l+s}{\PYZsq{}}\PY{l+s}{data/SiRFrame.txt}\PY{l+s}{\PYZsq{}}\PY{p}{,} \PY{n}{header}\PY{o}{=}\PY{n}{T}\PY{p}{)}\PY{p}{;}
         \PY{n}{SiRFrame} \PY{o}{=} \PY{k}{as}\PY{o}{.}\PY{n}{data}\PY{o}{.}\PY{n}{frame}\PY{p}{(}\PY{n}{SiR}\PY{p}{)}\PY{p}{;}
\end{Verbatim}

    \begin{Verbatim}[commandchars=\\\{\}]
{\color{incolor}In [{\color{incolor}93}]:} \PY{o}{\PYZpc{}}\PY{k}{R} \PY{n}{head}\PY{p}{(}\PY{n}{SiRFrame}\PY{p}{)}
\end{Verbatim}

            \begin{Verbatim}[commandchars=\\\{\}]
{\color{outcolor}Out[{\color{outcolor}93}]:}    X  Instrument  Resistance
         0  0           1    196.3052
         1  1           1    196.1240
         2  2           1    196.1890
         3  3           1    196.2569
         4  4           1    196.3403
         5  5           2    196.3042
\end{Verbatim}
        
    \begin{Verbatim}[commandchars=\\\{\}]
{\color{incolor}In [{\color{incolor}94}]:} \PY{o}{\PYZpc{}}\PY{k}{R} \PY{n}{tail}\PY{p}{(}\PY{n}{SiRFrame}\PY{p}{)}
\end{Verbatim}

            \begin{Verbatim}[commandchars=\\\{\}]
{\color{outcolor}Out[{\color{outcolor}94}]:}     X  Instrument  Resistance
         0  19           4    195.9885
         1  20           5    196.2119
         2  21           5    196.1051
         3  22           5    196.1850
         4  23           5    196.0052
         5  24           5    196.2090
\end{Verbatim}
        
    Data was read in correctly.

    \begin{Verbatim}[commandchars=\\\{\}]
{\color{incolor}In [{\color{incolor}95}]:} \PY{o}{\PYZpc{}\PYZpc{}}\PY{k}{R}
         \PY{c}{\PYZsh{}analysis in R}
         \PY{n}{SiRFrame}\PY{err}{\PYZdl{}}\PY{n}{Instrument} \PY{o}{=} \PY{n}{factor}\PY{p}{(}\PY{n}{SiRFrame}\PY{err}{\PYZdl{}}\PY{n}{Instrument}\PY{p}{)}\PY{p}{;}
         \PY{n}{SiRANOVA} \PY{o}{=} \PY{n}{aov}\PY{p}{(}\PY{n}{Resistance} \PY{o}{\PYZti{}} \PY{n}{Instrument}\PY{p}{,} \PY{n}{data}\PY{o}{=}\PY{n}{SiRFrame}\PY{p}{)}\PY{p}{;}
         \PY{n}{SiRLM} \PY{o}{=} \PY{n}{lm}\PY{p}{(}\PY{n}{Resistance} \PY{o}{\PYZti{}} \PY{n}{Instrument}\PY{p}{,} \PY{n}{data}\PY{o}{=}\PY{n}{SiRFrame}\PY{p}{)}\PY{p}{;}
\end{Verbatim}

    \begin{Verbatim}[commandchars=\\\{\}]
{\color{incolor}In [{\color{incolor}192}]:} \PY{o}{\PYZpc{}\PYZpc{}}\PY{k}{R}
          \PY{c}{\PYZsh{}obtaining precise numbers in R}
          \PY{k}{print}\PY{p}{(}\PY{n}{summary}\PY{p}{(}\PY{n}{SiRANOVA}\PY{p}{)}\PY{p}{,} \PY{n}{digits} \PY{o}{=}\PY{l+m+mi}{17}\PY{p}{)}\PY{p}{;}
          \PY{k}{print}\PY{p}{(}\PY{n}{sqrt}\PY{p}{(}\PY{n}{deviance}\PY{p}{(}\PY{n}{SiRANOVA}\PY{p}{)}\PY{o}{/}\PY{n}{df}\PY{o}{.}\PY{n}{residual}\PY{p}{(}\PY{n}{SiRANOVA}\PY{p}{)}\PY{p}{)}\PY{p}{,} \PY{n}{digits} \PY{o}{=} \PY{l+m+mi}{17}\PY{p}{)}\PY{p}{;}
          \PY{k}{print}\PY{p}{(}\PY{n}{summary}\PY{p}{(}\PY{n}{SiRLM}\PY{p}{)}\PY{p}{,} \PY{n}{digits} \PY{o}{=}\PY{l+m+mi}{17}\PY{p}{)}\PY{p}{;}
\end{Verbatim}

    \begin{Verbatim}[commandchars=\\\{\}]
{\color{incolor}In [{\color{incolor}33}]:} \PY{n}{SiRRresstd} \PY{o}{=} \PY{l+s}{\PYZsq{}}\PY{l+s}{0.10407606833466272189}\PY{l+s}{\PYZsq{}}
         \PY{n}{SiRRRsq} \PY{o}{=} \PY{l+s}{\PYZsq{}}\PY{l+s}{0.19099903905112250446}\PY{l+s}{\PYZsq{}}
         \PY{n}{SiRRModSS} \PY{o}{=} \PY{l+s}{\PYZsq{}}\PY{l+s}{0.051146261600009186588}\PY{l+s}{\PYZsq{}}
         \PY{n}{SiRRModMSE} \PY{o}{=} \PY{l+s}{\PYZsq{}}\PY{l+s}{0.012786565400002294912}\PY{l+s}{\PYZsq{}}
         \PY{n}{SiRRModSSResid} \PY{o}{=} \PY{l+s}{\PYZsq{}}\PY{l+s}{0.216636560000027678097}\PY{l+s}{\PYZsq{}}
         \PY{n}{SiRRModMSEResid} \PY{o}{=} \PY{l+s}{\PYZsq{}}\PY{l+s}{0.010831828000001384252}\PY{l+s}{\PYZsq{}}
         \PY{n}{SiRRFstat} \PY{o}{=} \PY{l+s}{\PYZsq{}}\PY{l+s}{1.1804600000000000648}\PY{l+s}{\PYZsq{}}
         
         \PY{n}{SiRRVals} \PY{o}{=} \PY{n}{np}\PY{o}{.}\PY{n}{array}\PY{p}{(}\PY{p}{[}\PY{n}{SiRRresstd}\PY{p}{,} \PY{n}{SiRRRsq}\PY{p}{,} \PY{n}{SiRRModSS}\PY{p}{,} \PY{n}{SiRRModMSE}\PY{p}{,}
                              \PY{n}{SiRRModSSResid}\PY{p}{,}\PY{n}{SiRRModMSEResid}\PY{p}{,}\PY{n}{SiRRFstat}\PY{p}{]}\PY{p}{)}
\end{Verbatim}


    \subsubsection{7.3.5 ANOVA values in SAS}


    \begin{Verbatim}[commandchars=\\\{\}]
{\color{incolor}In [{\color{incolor}}]:} \PY{n}{OPTIONS} \PY{n}{NODATE} \PY{n}{NONUMBER} \PY{n}{CENTER} \PY{n}{LS}\PY{o}{=}\PY{l+m+mi}{160}\PY{p}{;}
       \PY{o}{*}\PY{n}{Removes} \PY{n}{the} \PY{n}{header} \PY{n}{information} \PY{o+ow}{and} \PY{n}{centers} \PY{n}{output}\PY{p}{;}
       \PY{n}{OPTIONS} \PY{n}{FORMDLIM}\PY{o}{=}\PY{l+s}{\PYZdq{}}\PY{l+s}{\PYZti{}}\PY{l+s}{\PYZdq{}}\PY{p}{;}	
       
       \PY{n}{data} \PY{n}{SiRData}\PY{p}{;}
       \PY{n}{infile} \PY{l+s}{\PYZdq{}}\PY{l+s}{C:/Users/flunk\PYZus{}000/Desktop/CalPoly/IPythonNotebook/SeniorProject/data/SiRFrameSAS.txt}\PY{l+s}{\PYZdq{}} \PY{n}{DLM}\PY{o}{=}\PY{l+s}{\PYZsq{}}\PY{l+s}{,}\PY{l+s}{\PYZsq{}}\PY{p}{;}
       \PY{n+nb}{input} \PY{n}{Instrument} \PY{err}{\PYZdl{}} \PY{n}{Resistance}\PY{p}{;}
       \PY{n}{proc} \PY{k}{print} \PY{n}{data}\PY{o}{=}\PY{n}{SiRData}\PY{p}{;}
       \PY{n}{run}\PY{p}{;}
       
       \PY{n}{proc} \PY{n}{glm} \PY{n}{data}\PY{o}{=} \PY{n}{SiRData}\PY{p}{;}
       	\PY{k}{class} \PY{n+nc}{Instrument}\PY{p}{;}
       	\PY{n}{model} \PY{n}{Resistance} \PY{o}{=} \PY{n}{Instrument}\PY{p}{;}
       		\PY{n}{ODS} \PY{n}{output} \PY{n}{Overallanova} \PY{o}{=} \PY{n}{SiRanova}\PY{p}{;}
       		\PY{n}{ODS} \PY{n}{output} \PY{n}{fitstatistics} \PY{o}{=} \PY{n}{SiRfs}\PY{p}{;}
       \PY{n}{run}\PY{p}{;}
       
       \PY{n}{proc} \PY{k}{print} \PY{n}{data}\PY{o}{=}\PY{n}{SiRanova}\PY{p}{;}
       	\PY{n}{format} \PY{n}{SS} \PY{l+m+mf}{17.16}
       		\PY{n}{MS} \PY{l+m+mf}{17.16}
       		\PY{n}{fvalue} \PY{l+m+mf}{17.16}\PY{p}{;}
       	\PY{n}{run}\PY{p}{;}
           
       \PY{n}{proc} \PY{k}{print} \PY{n}{data}\PY{o}{=}\PY{n}{SiRfs}\PY{p}{;}
       	\PY{n}{format} \PY{n}{rsquare} \PY{l+m+mf}{17.16}
       		\PY{n}{rootmse} \PY{l+m+mf}{17.16}\PY{p}{;}
       	\PY{n}{run}\PY{p}{;}
\end{Verbatim}

    \begin{Verbatim}[commandchars=\\\{\}]
{\color{incolor}In [{\color{incolor}34}]:} \PY{n}{SiRSASresstd} \PY{o}{=} \PY{l+s}{\PYZsq{}}\PY{l+s}{0.1040760683346600}\PY{l+s}{\PYZsq{}}
         \PY{n}{SiRSASRsq} \PY{o}{=} \PY{l+s}{\PYZsq{}}\PY{l+s}{0.1909990390511160}\PY{l+s}{\PYZsq{}}
         \PY{n}{SiRSASModSS} \PY{o}{=} \PY{l+s}{\PYZsq{}}\PY{l+s}{0.0511462615999996}\PY{l+s}{\PYZsq{}}
         \PY{n}{SiRSASModMSE} \PY{o}{=} \PY{l+s}{\PYZsq{}}\PY{l+s}{0.0127865653999999}\PY{l+s}{\PYZsq{}}
         \PY{n}{SiRSASModSSResid} \PY{o}{=} \PY{l+s}{\PYZsq{}}\PY{l+s}{0.2166365600000160}\PY{l+s}{\PYZsq{}}
         \PY{n}{SiRSASModMSEResid} \PY{o}{=} \PY{l+s}{\PYZsq{}}\PY{l+s}{0.0108318280000008}\PY{l+s}{\PYZsq{}}
         \PY{n}{SiRSASFstat} \PY{o}{=} \PY{l+s}{\PYZsq{}}\PY{l+s}{1.180462374402440}\PY{l+s}{\PYZsq{}}
         
         \PY{n}{SiRSASVals} \PY{o}{=} \PY{n}{np}\PY{o}{.}\PY{n}{array}\PY{p}{(}\PY{p}{[}\PY{n}{SiRSASresstd}\PY{p}{,}\PY{n}{SiRSASRsq}\PY{p}{,}\PY{n}{SiRSASModSS}\PY{p}{,}\PY{n}{SiRSASModMSE}
                             \PY{p}{,}\PY{n}{SiRSASModSSResid}\PY{p}{,}\PY{n}{SiRSASModMSEResid}\PY{p}{,}\PY{n}{SiRSASFstat}\PY{p}{]}\PY{p}{)}
\end{Verbatim}


    \subsubsection{7.3.6 Testing Python's Precision against NIST, R, and SAS}


    For a detailed explanation of this process, refer to Section 7.2.6.

    Like 7.2.6, we need a label array to print out the values of interest:

    \begin{Verbatim}[commandchars=\\\{\}]
{\color{incolor}In [{\color{incolor}96}]:} \PY{n}{SiRLabels} \PY{o}{=} \PY{n}{np}\PY{o}{.}\PY{n}{array}\PY{p}{(}\PY{p}{[}\PY{l+s}{\PYZsq{}}\PY{l+s}{resstd:}\PY{l+s}{\PYZsq{}}\PY{p}{,}\PY{l+s}{\PYZsq{}}\PY{l+s}{R\PYZhy{}sq:}\PY{l+s}{\PYZsq{}}\PY{p}{,}\PY{l+s}{\PYZsq{}}\PY{l+s}{Model SS:}\PY{l+s}{\PYZsq{}}\PY{p}{,}\PY{l+s}{\PYZsq{}}\PY{l+s}{Model MS:}\PY{l+s}{\PYZsq{}}\PY{p}{,}
                                   \PY{l+s}{\PYZsq{}}\PY{l+s}{Model SSResid:}\PY{l+s}{\PYZsq{}}\PY{p}{,}\PY{l+s}{\PYZsq{}}\PY{l+s}{Model MSResid:}\PY{l+s}{\PYZsq{}}\PY{p}{,}\PY{l+s}{\PYZsq{}}\PY{l+s}{F\PYZhy{}stat:}\PY{l+s}{\PYZsq{}}\PY{p}{]}\PY{p}{)}
\end{Verbatim}

    \begin{Verbatim}[commandchars=\\\{\}]
{\color{incolor}In [{\color{incolor}97}]:} \PY{n}{SiRR} \PY{o}{=} \PY{n}{array\PYZus{}compare}\PY{p}{(}\PY{n}{SiRRVals}\PY{p}{,}\PY{n}{SiRCertVals}\PY{p}{,}\PY{n}{SiRLabels}\PY{p}{)}
\end{Verbatim}

    \begin{Verbatim}[commandchars=\\\{\}]
('resstd:', 15, 'of', 17)
('R-sq:', 16, 'of', 17)
('Model SS:', 16, 'of', 18)
('Model MS:', 16, 'of', 18)
('Model SSResid:', 15, 'of', 17)
('Model MSResid:', 16, 'of', 18)
('F-stat:', 7, 'of', 16)
    \end{Verbatim}

    \begin{Verbatim}[commandchars=\\\{\}]
{\color{incolor}In [{\color{incolor}98}]:} \PY{n}{SiRSAS} \PY{o}{=} \PY{n}{array\PYZus{}compare}\PY{p}{(}\PY{n}{SiRSASVals}\PY{p}{,}\PY{n}{SiRCertVals}\PY{p}{,}\PY{n}{SiRLabels}\PY{p}{)}
\end{Verbatim}

    \begin{Verbatim}[commandchars=\\\{\}]
('resstd:', 15, 'of', 17)
('R-sq:', 15, 'of', 17)
('Model SS:', 11, 'of', 18)
('Model MS:', 11, 'of', 18)
('Model SSResid:', 15, 'of', 17)
('Model MSResid:', 17, 'of', 18)
('F-stat:', 14, 'of', 16)
    \end{Verbatim}

    \begin{Verbatim}[commandchars=\\\{\}]
{\color{incolor}In [{\color{incolor}99}]:} \PY{n}{SiRPy} \PY{o}{=} \PY{n}{array\PYZus{}compare}\PY{p}{(}\PY{n}{SiRPyVals}\PY{p}{,}\PY{n}{SiRCertVals}\PY{p}{,}\PY{n}{SiRLabels}\PY{p}{)}
\end{Verbatim}

    \begin{Verbatim}[commandchars=\\\{\}]
('resstd:', 15, 'of', 17)
('R-sq:', 15, 'of', 17)
('Model SS:', 11, 'of', 18)
('Model MS:', 11, 'of', 18)
('Model SSResid:', 15, 'of', 17)
('Model MSResid:', 17, 'of', 18)
('F-stat:', 14, 'of', 16)
    \end{Verbatim}

    \begin{Verbatim}[commandchars=\\\{\}]
{\color{incolor}In [{\color{incolor}100}]:} \PY{n}{SiRR}\PY{p}{,} \PY{n}{SiRSAS}\PY{p}{,} \PY{n}{SiRPy}\PY{p}{,} \PY{n}{cert\PYZus{}val\PYZus{}lengths}\PY{p}{(}\PY{n}{SiRCertVals}\PY{p}{)}
\end{Verbatim}

            \begin{Verbatim}[commandchars=\\\{\}]
{\color{outcolor}Out[{\color{outcolor}100}]:} (array([15, 16, 16, 16, 15, 16,  7]),
           array([15, 15, 11, 11, 15, 17, 14]),
           array([15, 15, 11, 11, 15, 17, 14]),
           array([17, 17, 18, 18, 17, 18, 16]))
\end{Verbatim}
        
    With the exception of rounding the F-statistics early, R appears to have
handled this dataset well. Going back and examining the array values,
you can see R has the trailing zeros that match the ceritified values
while Python and SAS did not. Interestingly enough, extracting large
enough R values added additional digits after 4 or 5 trailing zeros. Did
R caluculate those digits or were they randomly produced since we forced
R to provide the extract precision?

    \begin{Verbatim}[commandchars=\\\{\}]
{\color{incolor}In [{\color{incolor}101}]:} \PY{c}{\PYZsh{}open a figure and add the axes}
          \PY{n}{SiRhist} \PY{o}{=} \PY{n}{plt}\PY{o}{.}\PY{n}{figure}\PY{p}{(}\PY{n}{figsize}\PY{o}{=}\PY{p}{(}\PY{l+m+mi}{16}\PY{p}{,}\PY{l+m+mi}{4}\PY{p}{)}\PY{p}{)}
          \PY{n}{gs}\PY{o}{=}\PY{n}{GridSpec}\PY{p}{(}\PY{l+m+mi}{1}\PY{p}{,}\PY{l+m+mi}{2}\PY{p}{)}
          \PY{n}{SiRaxis} \PY{o}{=} \PY{n}{SiRhist}\PY{o}{.}\PY{n}{add\PYZus{}subplot}\PY{p}{(}\PY{n}{gs}\PY{p}{[}\PY{l+m+mi}{0}\PY{p}{,}\PY{l+m+mi}{0}\PY{p}{]}\PY{p}{)}
          
          \PY{n}{SiRprograms}\PY{o}{=}\PY{l+m+mi}{4}
          \PY{c}{\PYZsh{}create array of bar values}
          \PY{n}{SiRMatches} \PY{o}{=} \PY{p}{[}\PY{n}{np}\PY{o}{.}\PY{n}{sum}\PY{p}{(}\PY{n}{SiRR}\PY{p}{)}\PY{p}{,}\PY{n}{np}\PY{o}{.}\PY{n}{sum}\PY{p}{(}\PY{n}{SiRSAS}\PY{p}{)}\PY{p}{,}\PY{n}{np}\PY{o}{.}\PY{n}{sum}\PY{p}{(}\PY{n}{SiRPy}\PY{p}{)}\PY{p}{,}
                        \PY{n}{np}\PY{o}{.}\PY{n}{sum}\PY{p}{(}\PY{n}{cert\PYZus{}val\PYZus{}lengths}\PY{p}{(}\PY{n}{SiRCertVals}\PY{p}{)}\PY{p}{)}\PY{p}{]}
          
          \PY{c}{\PYZsh{}location of bars on plot}
          \PY{n}{SiRloc} \PY{o}{=} \PY{n}{np}\PY{o}{.}\PY{n}{arange}\PY{p}{(}\PY{n}{SiRprograms}\PY{p}{)}
          
          \PY{n}{SiRbars} \PY{o}{=} \PY{n}{SiRaxis}\PY{o}{.}\PY{n}{bar}\PY{p}{(}\PY{n}{SiRloc}\PY{p}{,} \PY{n}{SiRMatches}\PY{p}{)}
          
          \PY{n}{SiRaxis}\PY{o}{.}\PY{n}{set\PYZus{}ylim}\PY{p}{(}\PY{l+m+mi}{0}\PY{p}{,}\PY{l+m+mi}{130}\PY{p}{)}
          \PY{n}{SiRaxis}\PY{o}{.}\PY{n}{set\PYZus{}ylabel}\PY{p}{(}\PY{l+s}{\PYZsq{}}\PY{l+s}{\PYZsh{} of matching values}\PY{l+s}{\PYZsq{}}\PY{p}{)}
          \PY{n}{SiRaxis}\PY{o}{.}\PY{n}{set\PYZus{}title}\PY{p}{(}\PY{l+s}{\PYZsq{}}\PY{l+s}{Precision of R,SAS,Py}\PY{l+s}{\PYZsq{}}\PY{p}{)}
          \PY{n}{SiRaxis}\PY{o}{.}\PY{n}{set\PYZus{}xticks}\PY{p}{(}\PY{n}{SiRloc}\PY{o}{+}\PY{o}{.}\PY{l+m+mi}{35}\PY{p}{)}
          \PY{n}{SiRXNames} \PY{o}{=} \PY{n}{SiRaxis}\PY{o}{.}\PY{n}{set\PYZus{}xticklabels}\PY{p}{(}\PY{p}{[}\PY{l+s}{\PYZsq{}}\PY{l+s}{R}\PY{l+s}{\PYZsq{}}\PY{p}{,} \PY{l+s}{\PYZsq{}}\PY{l+s}{SAS}\PY{l+s}{\PYZsq{}}\PY{p}{,} \PY{l+s}{\PYZsq{}}\PY{l+s}{Python}\PY{l+s}{\PYZsq{}}\PY{p}{,}\PY{l+s}{\PYZsq{}}\PY{l+s}{NIST}\PY{l+s}{\PYZsq{}}\PY{p}{]}\PY{p}{)}
          
          \PY{n}{SiRaxis2} \PY{o}{=} \PY{n}{SiRhist}\PY{o}{.}\PY{n}{add\PYZus{}subplot}\PY{p}{(}\PY{n}{gs}\PY{p}{[}\PY{l+m+mi}{0}\PY{p}{,}\PY{l+m+mi}{1}\PY{p}{]}\PY{p}{)}
          \PY{n}{SiRbars2} \PY{o}{=} \PY{n}{SiRaxis2}\PY{o}{.}\PY{n}{bar}\PY{p}{(}\PY{n}{SiRloc}\PY{p}{,} \PY{n}{SiRMatches}\PY{p}{)}
          \PY{n}{SiRaxis2}\PY{o}{.}\PY{n}{set\PYZus{}ylim}\PY{p}{(}\PY{l+m+mi}{95}\PY{p}{,}\PY{l+m+mi}{125}\PY{p}{)}
          \PY{n}{SiRaxis2}\PY{o}{.}\PY{n}{set\PYZus{}ylabel}\PY{p}{(}\PY{l+s}{\PYZsq{}}\PY{l+s}{\PYZsh{} of matching values}\PY{l+s}{\PYZsq{}}\PY{p}{)}
          \PY{n}{SiRaxis2}\PY{o}{.}\PY{n}{set\PYZus{}title}\PY{p}{(}\PY{l+s}{\PYZsq{}}\PY{l+s}{Precision of R,SAS,Py}\PY{l+s}{\PYZsq{}}\PY{p}{)}
          \PY{n}{SiRaxis2}\PY{o}{.}\PY{n}{set\PYZus{}xticks}\PY{p}{(}\PY{n}{SiRloc}\PY{o}{+}\PY{o}{.}\PY{l+m+mi}{35}\PY{p}{)}
          \PY{n}{SiRXNames} \PY{o}{=} \PY{n}{SiRaxis2}\PY{o}{.}\PY{n}{set\PYZus{}xticklabels}\PY{p}{(}\PY{p}{[}\PY{l+s}{\PYZsq{}}\PY{l+s}{R}\PY{l+s}{\PYZsq{}}\PY{p}{,} \PY{l+s}{\PYZsq{}}\PY{l+s}{SAS}\PY{l+s}{\PYZsq{}}\PY{p}{,} \PY{l+s}{\PYZsq{}}\PY{l+s}{Python}\PY{l+s}{\PYZsq{}}\PY{p}{,}\PY{l+s}{\PYZsq{}}\PY{l+s}{NIST}\PY{l+s}{\PYZsq{}}\PY{p}{]}\PY{p}{)}
          \PY{n}{SiRMatches}
\end{Verbatim}

            \begin{Verbatim}[commandchars=\\\{\}]
{\color{outcolor}Out[{\color{outcolor}101}]:} [101, 98, 98, 121]
\end{Verbatim}
        
    \begin{center}
    \adjustimage{max size={0.9\linewidth}{0.9\paperheight}}{SeniorProjectPDF_files/SeniorProjectPDF_191_1.png}
    \end{center}
    { \hspace*{\fill} \\}
    
    Over the 7 values of concern, R was more precise than SAS and Python by
2 decimal places. The programs appear to have about the same precision
overall, but this is due to big mismatches on some values. R rounded the
F-stat early (7 decimal places before SAS and Python), yet R was 5
decimal places more precise than SAS and Python for calculating Model SS
and Model MSE.


    \subsubsection{7.3.7 Plotting SiR data with Seaborn}


    The NIST graphics aren't as detailed as I would like, but they give us
something to compare the Python graphics to. This time we will Seaborn
to try to match the NIST plots.

SiR ANOVA: 

    \begin{Verbatim}[commandchars=\\\{\}]
{\color{incolor}In [{\color{incolor}102}]:} \PY{c}{\PYZsh{}plot with seaborns linear model plot}
          \PY{c}{\PYZsh{}use the column names are the first and second argument}
          \PY{c}{\PYZsh{}and the data frame as the thid argument}
          \PY{n}{SiRANOVA} \PY{o}{=} \PY{n}{sns}\PY{o}{.}\PY{n}{lmplot}\PY{p}{(}\PY{l+s}{\PYZsq{}}\PY{l+s}{Instrument}\PY{l+s}{\PYZsq{}}\PY{p}{,}\PY{l+s}{\PYZsq{}}\PY{l+s}{Resistance}\PY{l+s}{\PYZsq{}}\PY{p}{,}\PY{n}{data}\PY{o}{=}\PY{n}{SiRFrame}\PY{p}{,}
                     \PY{n}{ci}\PY{o}{=}\PY{n+nb+bp}{False}\PY{p}{,} \PY{n}{hue}\PY{o}{=}\PY{l+s}{\PYZsq{}}\PY{l+s}{Instrument}\PY{l+s}{\PYZsq{}}\PY{p}{,}\PY{n}{fit\PYZus{}reg}\PY{o}{=}\PY{n+nb+bp}{False}\PY{p}{)}
          \PY{n}{SiRANOVA}\PY{o}{.}\PY{n}{set}\PY{p}{(}\PY{n}{title} \PY{o}{=}\PY{l+s}{\PYZsq{}}\PY{l+s}{Resistivity vs Instrument}\PY{l+s}{\PYZsq{}}\PY{p}{)}
          \PY{n}{SiRANOVA}\PY{o}{.}\PY{n}{set}\PY{p}{(}\PY{n}{ylabel} \PY{o}{=} \PY{l+s}{\PYZsq{}}\PY{l+s}{Resistivity,ohm*cm}\PY{l+s}{\PYZsq{}}\PY{p}{)}\PY{p}{;}
\end{Verbatim}

    \begin{center}
    \adjustimage{max size={0.9\linewidth}{0.9\paperheight}}{SeniorProjectPDF_files/SeniorProjectPDF_195_0.png}
    \end{center}
    { \hspace*{\fill} \\}
    
    SiR Residuals vs.~Instrument: 

    \begin{Verbatim}[commandchars=\\\{\}]
{\color{incolor}In [{\color{incolor}101}]:} \PY{n}{SiRPlot} \PY{o}{=} \PY{n}{sns}\PY{o}{.}\PY{n}{regplot}\PY{p}{(}\PY{n}{SiRFrame}\PY{p}{[}\PY{l+s}{\PYZsq{}}\PY{l+s}{Instrument}\PY{l+s}{\PYZsq{}}\PY{p}{]}\PY{p}{,}\PY{n}{SiRLM}\PY{o}{.}\PY{n}{resid}\PY{p}{,} \PY{n}{ci}\PY{o}{=}\PY{n+nb+bp}{False}\PY{p}{)}
          \PY{n}{SiRPlot}\PY{o}{.}\PY{n}{set}\PY{p}{(}\PY{n}{ylabel} \PY{o}{=} \PY{l+s}{\PYZsq{}}\PY{l+s}{Residuals}\PY{l+s}{\PYZsq{}}\PY{p}{)}
          \PY{n}{SiRPlot}\PY{o}{.}\PY{n}{set}\PY{p}{(}\PY{n}{title}\PY{o}{=}\PY{l+s}{\PYZsq{}}\PY{l+s}{Residuals vs Instrument}\PY{l+s}{\PYZsq{}}\PY{p}{)}\PY{p}{;}
\end{Verbatim}

    \begin{center}
    \adjustimage{max size={0.9\linewidth}{0.9\paperheight}}{SeniorProjectPDF_files/SeniorProjectPDF_197_0.png}
    \end{center}
    { \hspace*{\fill} \\}
    
    In sections 7.2 and 7.3 we have determined Python can perform as well as
SAS and R on these NIST data sets.


    \subsection{7.4 Univariate Summary Statistics: PiDigits}



    \subsubsection{7.4.1 Reading in and preparing data from an ASCII webpage}


    For a detailed explanation of this process, refer to Secton 7.2.1.

    \begin{Verbatim}[commandchars=\\\{\}]
{\color{incolor}In [{\color{incolor}103}]:} \PY{n}{PiUrl} \PY{o}{=} \PY{l+s}{\PYZsq{}}\PY{l+s}{http://www.itl.nist.gov/div898/strd/univ/data/PiDigits.dat}\PY{l+s}{\PYZsq{}}
          \PY{n+nb}{open}\PY{p}{(}\PY{l+s}{\PYZsq{}}\PY{l+s}{data/Pi.dat}\PY{l+s}{\PYZsq{}}\PY{p}{,}\PY{l+s}{\PYZsq{}}\PY{l+s}{wb}\PY{l+s}{\PYZsq{}}\PY{p}{)}\PY{o}{.}\PY{n}{write}\PY{p}{(}\PY{n}{ul}\PY{o}{.}\PY{n}{urlopen}\PY{p}{(}\PY{n}{PiUrl}\PY{p}{)}\PY{o}{.}\PY{n}{read}\PY{p}{(}\PY{p}{)}\PY{p}{)}
\end{Verbatim}

    \begin{Verbatim}[commandchars=\\\{\}]
{\color{incolor}In [{\color{incolor}104}]:} \PY{n}{PiData} \PY{o}{=} \PY{n}{np}\PY{o}{.}\PY{n}{loadtxt}\PY{p}{(}\PY{l+s}{\PYZsq{}}\PY{l+s}{data/Pi.dat}\PY{l+s}{\PYZsq{}}\PY{p}{,}\PY{n}{skiprows}\PY{o}{=}\PY{l+m+mi}{60}\PY{p}{,}\PY{n}{dtype}\PY{o}{=}\PY{n+nb}{int}\PY{p}{)}
          \PY{n}{PiFrame} \PY{o}{=} \PY{n}{pd}\PY{o}{.}\PY{n}{DataFrame}\PY{p}{(}\PY{n}{PiData}\PY{p}{)}
\end{Verbatim}

    \begin{Verbatim}[commandchars=\\\{\}]
{\color{incolor}In [{\color{incolor}105}]:} \PY{n}{PiFrame}\PY{o}{.}\PY{n}{head}\PY{p}{(}\PY{p}{)}
\end{Verbatim}

            \begin{Verbatim}[commandchars=\\\{\}]
{\color{outcolor}Out[{\color{outcolor}105}]:}    0
          0  3
          1  1
          2  4
          3  1
          4  5
\end{Verbatim}
        
    \begin{Verbatim}[commandchars=\\\{\}]
{\color{incolor}In [{\color{incolor}106}]:} \PY{n}{PiFrame}\PY{o}{.}\PY{n}{tail}\PY{p}{(}\PY{p}{)}
\end{Verbatim}

            \begin{Verbatim}[commandchars=\\\{\}]
{\color{outcolor}Out[{\color{outcolor}106}]:}       0
          4995  6
          4996  0
          4997  4
          4998  7
          4999  2
\end{Verbatim}
        
    Data looks good. I think we can all agree that Pi starts with
31415\ldots{}


    \subsubsection{7.4.2 NIST Certified Values}


    For a detailed explanation of this process, refer to Section 7.2.2.

    The Certified values we are trying to match can be found at:
http://www.itl.nist.gov/div898/strd/univ/certvalues/pidigits.html

    \begin{Verbatim}[commandchars=\\\{\}]
{\color{incolor}In [{\color{incolor}98}]:} \PY{n}{PiMean} \PY{o}{=} \PY{l+s}{\PYZsq{}}\PY{l+s}{4.53480000000000}\PY{l+s}{\PYZsq{}}
         \PY{n}{PiSigma} \PY{o}{=} \PY{l+s}{\PYZsq{}}\PY{l+s}{2.86733906028871}\PY{l+s}{\PYZsq{}}
         
         \PY{n}{PiCertVals} \PY{o}{=} \PY{n}{np}\PY{o}{.}\PY{n}{array}\PY{p}{(}\PY{p}{[}\PY{n}{PiMean}\PY{p}{,}\PY{n}{PiSigma}\PY{p}{]}\PY{p}{)}
\end{Verbatim}


    \subsubsection{7.4.3 Univariate Summary Statistics in Python}


    NIST only provides the mean and standard deviation for comparison. I'm
going to show you a couple things that might be helpful as well as
extract the values for comparison.

    \begin{Verbatim}[commandchars=\\\{\}]
{\color{incolor}In [{\color{incolor}99}]:} \PY{c}{\PYZsh{}dont forget to check dir(PiFrame) to see what a DataFrame can offer}
         \PY{n}{PiFrame}\PY{o}{.}\PY{n}{describe}\PY{p}{(}\PY{p}{)}
\end{Verbatim}

            \begin{Verbatim}[commandchars=\\\{\}]
{\color{outcolor}Out[{\color{outcolor}99}]:}                  0
         count  5000.000000
         mean      4.534800
         std       2.867339
         min       0.000000
         25\%       2.000000
         50\%       5.000000
         75\%       7.000000
         max       9.000000
\end{Verbatim}
        
    \begin{Verbatim}[commandchars=\\\{\}]
{\color{incolor}In [{\color{incolor}100}]:} \PY{n}{PiFrame}\PY{o}{.}\PY{n}{hist}\PY{p}{(}\PY{p}{)}\PY{p}{;}
\end{Verbatim}

    \begin{center}
    \adjustimage{max size={0.9\linewidth}{0.9\paperheight}}{SeniorProjectPDF_files/SeniorProjectPDF_214_0.png}
    \end{center}
    { \hspace*{\fill} \\}
    
    Back to the desired values. As you can see, the DataFrame has the values
we want already built in. Since we desire a more precise value for this
exercise, the simplest way is to use NumPy.

    \begin{Verbatim}[commandchars=\\\{\}]
{\color{incolor}In [{\color{incolor}399}]:} \PY{n}{PyPiMean} \PY{o}{=} \PY{n+nb}{repr}\PY{p}{(}\PY{n}{np}\PY{o}{.}\PY{n}{mean}\PY{p}{(}\PY{n}{PiData}\PY{p}{)}\PY{p}{)}
          \PY{n}{PyPistd} \PY{o}{=} \PY{n+nb}{repr}\PY{p}{(}\PY{n}{np}\PY{o}{.}\PY{n}{std}\PY{p}{(}\PY{n}{PiData}\PY{p}{)}\PY{p}{)}
          
          \PY{n}{PyPiVals} \PY{o}{=} \PY{n}{np}\PY{o}{.}\PY{n}{array}\PY{p}{(}\PY{p}{[}\PY{n}{PyPiMean}\PY{p}{,} \PY{n}{PyPistd}\PY{p}{]}\PY{p}{)}
\end{Verbatim}


    \subsubsection{7.4.4 Testing Python's Precision against NIST values}


    For a detailed explanation of this process, refer to Section 7.2.6.

    \begin{Verbatim}[commandchars=\\\{\}]
{\color{incolor}In [{\color{incolor}394}]:} \PY{c}{\PYZsh{}specify titles for the output values}
          \PY{n}{PiLabelArray} \PY{o}{=} \PY{n}{np}\PY{o}{.}\PY{n}{array}\PY{p}{(}\PY{p}{[}\PY{l+s}{\PYZsq{}}\PY{l+s}{Mean:}\PY{l+s}{\PYZsq{}}\PY{p}{,}\PY{l+s}{\PYZsq{}}\PY{l+s}{Sigma:}\PY{l+s}{\PYZsq{}}\PY{p}{]}\PY{p}{)}
\end{Verbatim}

    \begin{Verbatim}[commandchars=\\\{\}]
{\color{incolor}In [{\color{incolor}395}]:} \PY{n}{PiPy} \PY{o}{=} \PY{n}{array\PYZus{}compare}\PY{p}{(}\PY{n}{PyPiVals}\PY{p}{,}\PY{n}{PiCertVals}\PY{p}{,}\PY{n}{PiLabelArray}\PY{p}{)}
\end{Verbatim}

    \begin{Verbatim}[commandchars=\\\{\}]
('Mean:', 5, 'of', 16)
('Sigma:', 5, 'of', 16)
    \end{Verbatim}

    That's unforunate\ldots{} Let's look at the values and see if we feel
our previous comparison method might be a bit misleading in this
instance.

    \begin{Verbatim}[commandchars=\\\{\}]
{\color{incolor}In [{\color{incolor}401}]:} \PY{k}{print}\PY{p}{(}\PY{n}{PyPiVals}\PY{p}{[}\PY{l+m+mi}{0}\PY{p}{]}\PY{p}{)}
          \PY{k}{print}\PY{p}{(}\PY{n}{PiCertVals}\PY{p}{[}\PY{l+m+mi}{0}\PY{p}{]}\PY{p}{)}
\end{Verbatim}

    \begin{Verbatim}[commandchars=\\\{\}]
4.5347999999999997
4.53480000000000
    \end{Verbatim}

    These means are essentially the same. Not nearly as bad as matching 5 of
16 digits would lead us to believe.

    \begin{Verbatim}[commandchars=\\\{\}]
{\color{incolor}In [{\color{incolor}403}]:} \PY{k}{print}\PY{p}{(}\PY{n}{PyPiVals}\PY{p}{[}\PY{l+m+mi}{1}\PY{p}{]}\PY{p}{)}
          \PY{k}{print}\PY{p}{(}\PY{n}{PiCertVals}\PY{p}{[}\PY{l+m+mi}{1}\PY{p}{]}\PY{p}{)}
\end{Verbatim}

    \begin{Verbatim}[commandchars=\\\{\}]
2.8670523120445486
2.86733906028871
    \end{Verbatim}

    The NumPy standard deviation is way off. Lucky for us, we still have a
Pandas DataFrame all set to go. Here's a work around:

    \begin{Verbatim}[commandchars=\\\{\}]
{\color{incolor}In [{\color{incolor}416}]:} \PY{n}{PyMean} \PY{o}{=} \PY{n}{PiFrame}\PY{o}{.}\PY{n}{mean}\PY{p}{(}\PY{p}{)}
          \PY{k}{print} \PY{l+s}{\PYZsq{}}\PY{l+s+si}{\PYZpc{}17.14f}\PY{l+s}{\PYZsq{}} \PY{o}{\PYZpc{}} \PY{p}{(}\PY{n}{PyMean}\PY{p}{)}
          \PY{k}{print}\PY{p}{(}\PY{n}{PiCertVals}\PY{p}{[}\PY{l+m+mi}{0}\PY{p}{]}\PY{p}{)}
          
          \PY{n}{PySTD} \PY{o}{=} \PY{n}{PiFrame}\PY{o}{.}\PY{n}{std}\PY{p}{(}\PY{p}{)}
          \PY{k}{print} \PY{l+s}{\PYZsq{}}\PY{l+s+si}{\PYZpc{}17.14f}\PY{l+s}{\PYZsq{}} \PY{o}{\PYZpc{}} \PY{p}{(}\PY{n}{PySTD}\PY{p}{)}
          \PY{k}{print}\PY{p}{(}\PY{n}{PiCertVals}\PY{p}{[}\PY{l+m+mi}{1}\PY{p}{]}\PY{p}{)}
\end{Verbatim}

    \begin{Verbatim}[commandchars=\\\{\}]
4.53480000000000
4.53480000000000
 2.86733906028871
2.86733906028871
    \end{Verbatim}

    Success! This method might be a few more lines of code, but it appears
to be a more precise approach. 


    \section{Chapter 8 Streaming Data in the IPython Notebook}


    An interest in working with Dynamic data is what brought Dr.~Doi and I
together on this project. Dynamic data is data that is always changing.
Some data sets might change only a few times in a 5 minute interval,
others might change 100 times a second. The fascinating thing about
dynamic data is any time it is analyzed, the analysis is potentionally
behind the data. Consider LADWP's (Los Angeles Department of Water and
Power) water data. In 2013, the United States Census Bureau estimated
the population of the city of Los Angeles was about 3.9 millon people.
With 3.9 million people, it seems impossible that the water consumption
in Los Angeles could ever stop. This implys that as we analyze water
consumption in LA, we are immediately missing new data. In the future, I
intend to use dynamic environmental data streams to provide people with
accurate, analyzed in real time, information that enables them to make
choices that best support sustainability in their area.


    \subsection{8.1 Python Packages for Streaming and the Import Cell}


    \begin{Verbatim}[commandchars=\\\{\}]
{\color{incolor}In [{\color{incolor}4}]:} \PY{k+kn}{import} \PY{n+nn}{requests}
        \PY{k+kn}{import} \PY{n+nn}{json}
        \PY{k+kn}{import} \PY{n+nn}{pandas} \PY{k+kn}{as} \PY{n+nn}{pd}
        \PY{k+kn}{from} \PY{n+nn}{mpl\PYZus{}toolkits.basemap} \PY{k+kn}{import} \PY{n}{Basemap}
        \PY{k+kn}{import} \PY{n+nn}{numpy} \PY{k+kn}{as} \PY{n+nn}{np}
        \PY{k+kn}{import} \PY{n+nn}{matplotlib.pyplot} \PY{k+kn}{as} \PY{n+nn}{plt}
        \PY{o}{\PYZpc{}}\PY{k}{matplotlib} \PY{n}{inline}
\end{Verbatim}


    \subsection{8.2 Streaming Data from USA.gov}


    USA.gov describes the data as, ``We provide a raw pub/sub feed of data
created any time anyone clicks on a 1.USA.gov URL. The pub/sub endpoint
responds to http requests for any 1.USA.gov URL and returns a stream of
JSON entries, one per line, that represent real-time clicks.''

A few years ago USA.gov ``held a nationwide 1.USA.gov Hack Day\ldots{}
to encourage people to explore the 1.USA.gov data.'' The projects and
code resulting from this are more sophicated than what we're doing here
and can be found at:
http://www.usa.gov/About/developer-resources/1usagov.shtml

    \begin{Verbatim}[commandchars=\\\{\}]
{\color{incolor}In [{\color{incolor}5}]:} \PY{n}{url} \PY{o}{=} \PY{l+s}{\PYZdq{}}\PY{l+s}{http://developer.usa.gov/1usagov}\PY{l+s}{\PYZdq{}}
\end{Verbatim}

    \begin{Verbatim}[commandchars=\\\{\}]
{\color{incolor}In [{\color{incolor}87}]:} \PY{c}{\PYZsh{}url argument is the live datastream}
         \PY{n}{r} \PY{o}{=} \PY{n}{requests}\PY{o}{.}\PY{n}{get}\PY{p}{(}\PY{n}{url}\PY{p}{,} \PY{n}{stream}\PY{o}{=}\PY{n+nb+bp}{True}\PY{p}{)}
         
         \PY{c}{\PYZsh{}after grabbing n data values, the datastream stops}
         \PY{n}{n} \PY{o}{=} \PY{l+m+mi}{500}
         \PY{n}{data} \PY{o}{=} \PY{p}{[}\PY{p}{]}
         
         \PY{c}{\PYZsh{}looks at each line of the request individually and adds it to the list \PYZdq{}data\PYZdq{}}
         \PY{k}{for} \PY{n}{i}\PY{p}{,} \PY{n}{line} \PY{o+ow}{in} \PY{n+nb}{enumerate}\PY{p}{(}\PY{n}{r}\PY{o}{.}\PY{n}{iter\PYZus{}lines}\PY{p}{(}\PY{p}{)}\PY{p}{)}\PY{p}{:}
             \PY{n}{data}\PY{o}{.}\PY{n}{append}\PY{p}{(}\PY{n}{line}\PY{p}{)}
             
             \PY{c}{\PYZsh{}this is a dirty little trick that should be avoided in larger functions and }
             \PY{c}{\PYZsh{}programs, but works great for this quick line fectching function}
             \PY{k}{if} \PY{n}{i} \PY{o}{\PYZgt{}} \PY{n}{n}\PY{p}{:}
                 \PY{k}{break}
\end{Verbatim}

    \begin{Verbatim}[commandchars=\\\{\}]
{\color{incolor}In [{\color{incolor}88}]:} \PY{c}{\PYZsh{}load the json lines from the list}
         \PY{n}{jdata} \PY{o}{=} \PY{p}{[}\PY{n}{json}\PY{o}{.}\PY{n}{loads}\PY{p}{(}\PY{n}{item}\PY{p}{)} \PY{k}{for} \PY{n}{item} \PY{o+ow}{in} \PY{n}{data}\PY{p}{[}\PY{l+m+mi}{1}\PY{p}{:}\PY{p}{]}\PY{p}{]}
\end{Verbatim}

    \begin{Verbatim}[commandchars=\\\{\}]
{\color{incolor}In [{\color{incolor}89}]:} \PY{c}{\PYZsh{}create a DataFrame}
         \PY{n}{USAGovFrame} \PY{o}{=} \PY{n}{pd}\PY{o}{.}\PY{n}{DataFrame}\PY{p}{(}\PY{n}{jdata}\PY{p}{)}
\end{Verbatim}

    We built a DataFrame from USAGov's live stream. Let's take a peek at it
and see what we ended up with.

    \begin{Verbatim}[commandchars=\\\{\}]
{\color{incolor}In [{\color{incolor}90}]:} \PY{n}{USAGovFrame}\PY{o}{.}\PY{n}{head}\PY{p}{(}\PY{p}{)}
\end{Verbatim}

            \begin{Verbatim}[commandchars=\\\{\}]
{\color{outcolor}Out[{\color{outcolor}90}]:}    \_heartbeat\_                            \_id  \textbackslash{}
         0          NaN  543ae232-002b9-0416e-cf1cf10a   
         1          NaN  543ae232-003ac-038db-301cf10a   
         2          NaN  543ae233-00395-07c0e-361cf10a   
         3          NaN  543ae234-00165-07334-261cf10a   
         4          NaN  543ae234-00364-06587-2a1cf10a   
         
                                                            a               al   c  \textbackslash{}
         0  Mozilla/5.0 (Windows NT 6.1; WOW64) AppleWebKi\ldots  es-419,es;q=0.8  MX   
         1  Mozilla/5.0 (iPhone; CPU iPhone OS 7\_1\_2 like \ldots            en-us  US   
         2  Mozilla/5.0 (iPhone; CPU iPhone OS 8\_0\_2 like \ldots            fr-fr  FR   
         3  Mozilla/5.0 (Windows NT 6.1; WOW64) AppleWebKi\ldots   en-US,en;q=0.8  US   
         4  Mozilla/5.0 (Macintosh; Intel Mac OS X 10\_9\_5)\ldots            es-es  ES   
         
            ckw           cy   dp        g   gr \ldots           hc         hh   kw  \textbackslash{}
         0  NaN          NaN  NaN    15r91  NaN \ldots   1365701422       j.mp  NaN   
         1  NaN       Ponder  NaN  1v02m1T   TX \ldots   1413143626     ift.tt  NaN   
         2  NaN          NaN  NaN   ZzdKp8  NaN \ldots   1413082406  1.usa.gov  NaN   
         3  NaN  Spirit Lake  NaN  1qfk3VH   IA \ldots   1413123177  1.usa.gov  NaN   
         4  NaN         Vigo  NaN    K6Cor   58 \ldots   1394114329  1.usa.gov  NaN   
         
                       l                   ll nk                       r           t  \textbackslash{}
         0     pontifier      [19.43, -99.13]  0                  direct  1413145138   
         1         ifttt  [33.1884, -97.2889]  0                  direct  1413145138   
         2  tweetdeckapi        [48.86, 2.35]  0  http://t.co/tZd7gGVAXC  1413145139   
         3     theusnavy  [43.4216, -95.0932]  1                  direct  1413145140   
         4     anonymous   [42.2328, -8.7226]  0                  direct  1413145140   
         
                         tz                                                  u  
         0                                                 http://www.nsa.gov/  
         1  America/Chicago  http://alerts.weather.gov/cap/wwacapget.php?x=\ldots  
         2     Europe/Paris  http://earthobservatory.nasa.gov/IOTD/view.php\ldots  
         3  America/Chicago  http://www.navy.mil/submit/display.asp?story\_i\ldots  
         4    Europe/Madrid  http://www.fda.gov/ForConsumers/ByAudience/For\ldots  
         
         [5 rows x 21 columns]
\end{Verbatim}
        
    After looking at the descriptions of the variables provided from
1USA.gov, I don't know what I would do with several of them. Let's
reduce the DataFrame to variables we are interested in playing with.

    \begin{Verbatim}[commandchars=\\\{\}]
{\color{incolor}In [{\color{incolor}91}]:} \PY{c}{\PYZsh{}drop all the columns that I don\PYZsq{}t know anything about}
         \PY{c}{\PYZsh{}ckw and dp aren\PYZsq{}t consistently collected, especially with smaller sample sizes}
         \PY{c}{\PYZsh{}this will check for them and drop them if they are present}
         \PY{k}{if} \PY{l+s}{\PYZsq{}}\PY{l+s}{ckw}\PY{l+s}{\PYZsq{}} \PY{o+ow}{and} \PY{l+s}{\PYZsq{}}\PY{l+s}{dp}\PY{l+s}{\PYZsq{}} \PY{o+ow}{in} \PY{n}{USAGovFrame}\PY{o}{.}\PY{n}{columns}\PY{p}{:}
             \PY{n}{USAGovFrame}\PY{o}{.}\PY{n}{drop}\PY{p}{(}\PY{p}{[}\PY{l+s}{\PYZsq{}}\PY{l+s}{\PYZus{}heartbeat\PYZus{}}\PY{l+s}{\PYZsq{}}\PY{p}{,}\PY{l+s}{\PYZsq{}}\PY{l+s}{\PYZus{}id}\PY{l+s}{\PYZsq{}}\PY{p}{,}\PY{l+s}{\PYZsq{}}\PY{l+s}{al}\PY{l+s}{\PYZsq{}}\PY{p}{,}\PY{l+s}{\PYZsq{}}\PY{l+s}{ckw}\PY{l+s}{\PYZsq{}}\PY{p}{,}\PY{l+s}{\PYZsq{}}\PY{l+s}{nk}\PY{l+s}{\PYZsq{}}\PY{p}{,}\PY{l+s}{\PYZsq{}}\PY{l+s}{g}\PY{l+s}{\PYZsq{}}\PY{p}{,}\PY{l+s}{\PYZsq{}}\PY{l+s}{h}\PY{l+s}{\PYZsq{}}\PY{p}{,}\PY{l+s}{\PYZsq{}}\PY{l+s}{kw}\PY{l+s}{\PYZsq{}}\PY{p}{,}\PY{l+s}{\PYZsq{}}\PY{l+s}{hc}\PY{l+s}{\PYZsq{}}\PY{p}{,}\PY{l+s}{\PYZsq{}}\PY{l+s}{dp}\PY{l+s}{\PYZsq{}}\PY{p}{]}\PY{p}{,}\PY{n}{inplace}\PY{o}{=}\PY{n+nb+bp}{True}\PY{p}{,} \PY{n}{axis}\PY{o}{=}\PY{l+m+mi}{1}\PY{p}{)}
         \PY{k}{else}\PY{p}{:}
             \PY{n}{USAGovFrame}\PY{o}{.}\PY{n}{drop}\PY{p}{(}\PY{p}{[}\PY{l+s}{\PYZsq{}}\PY{l+s}{\PYZus{}heartbeat\PYZus{}}\PY{l+s}{\PYZsq{}}\PY{p}{,}\PY{l+s}{\PYZsq{}}\PY{l+s}{\PYZus{}id}\PY{l+s}{\PYZsq{}}\PY{p}{,}\PY{l+s}{\PYZsq{}}\PY{l+s}{al}\PY{l+s}{\PYZsq{}}\PY{p}{,}\PY{l+s}{\PYZsq{}}\PY{l+s}{nk}\PY{l+s}{\PYZsq{}}\PY{p}{,}\PY{l+s}{\PYZsq{}}\PY{l+s}{g}\PY{l+s}{\PYZsq{}}\PY{p}{,}\PY{l+s}{\PYZsq{}}\PY{l+s}{h}\PY{l+s}{\PYZsq{}}\PY{p}{,}\PY{l+s}{\PYZsq{}}\PY{l+s}{kw}\PY{l+s}{\PYZsq{}}\PY{p}{,}\PY{l+s}{\PYZsq{}}\PY{l+s}{hc}\PY{l+s}{\PYZsq{}}\PY{p}{]}\PY{p}{,}\PY{n}{inplace}\PY{o}{=}\PY{n+nb+bp}{True}\PY{p}{,} \PY{n}{axis}\PY{o}{=}\PY{l+m+mi}{1}\PY{p}{)}
\end{Verbatim}

    \begin{Verbatim}[commandchars=\\\{\}]
{\color{incolor}In [{\color{incolor}92}]:} \PY{c}{\PYZsh{}add user friendly names for the remaining variables}
         \PY{n}{USAGovFrame}\PY{o}{.}\PY{n}{columns}\PY{o}{=}\PY{p}{[}\PY{l+s}{\PYZsq{}}\PY{l+s}{User\PYZus{}Agent}\PY{l+s}{\PYZsq{}}\PY{p}{,}\PY{l+s}{\PYZsq{}}\PY{l+s}{Country\PYZus{}Code}\PY{l+s}{\PYZsq{}}\PY{p}{,}\PY{l+s}{\PYZsq{}}\PY{l+s}{Geo\PYZus{}city\PYZus{}name}\PY{l+s}{\PYZsq{}}\PY{p}{,}
                              \PY{l+s}{\PYZsq{}}\PY{l+s}{Geo\PYZus{}Region}\PY{l+s}{\PYZsq{}}\PY{p}{,}\PY{l+s}{\PYZsq{}}\PY{l+s}{Short\PYZus{}url\PYZus{}Cname}\PY{l+s}{\PYZsq{}}\PY{p}{,}\PY{l+s}{\PYZsq{}}\PY{l+s}{Encoding\PYZus{}user\PYZus{}login}\PY{l+s}{\PYZsq{}}\PY{p}{,}
                              \PY{l+s}{\PYZsq{}}\PY{l+s}{[Latitude,Longitude]}\PY{l+s}{\PYZsq{}}\PY{p}{,}\PY{l+s}{\PYZsq{}}\PY{l+s}{Referring\PYZus{}URL}\PY{l+s}{\PYZsq{}}\PY{p}{,}\PY{l+s}{\PYZsq{}}\PY{l+s}{Timestamp}\PY{l+s}{\PYZsq{}}\PY{p}{,}
                              \PY{l+s}{\PYZsq{}}\PY{l+s}{Timezone}\PY{l+s}{\PYZsq{}}\PY{p}{,}\PY{l+s}{\PYZsq{}}\PY{l+s}{Long\PYZus{}URL}\PY{l+s}{\PYZsq{}}\PY{p}{]}
\end{Verbatim}

    \begin{Verbatim}[commandchars=\\\{\}]
{\color{incolor}In [{\color{incolor}93}]:} \PY{c}{\PYZsh{}look at the new DataFrame}
         \PY{n}{USAGovFrame}\PY{o}{.}\PY{n}{head}\PY{p}{(}\PY{p}{)}
\end{Verbatim}

            \begin{Verbatim}[commandchars=\\\{\}]
{\color{outcolor}Out[{\color{outcolor}93}]:}                                           User\_Agent Country\_Code  \textbackslash{}
         0  Mozilla/5.0 (Windows NT 6.1; WOW64) AppleWebKi\ldots           MX   
         1  Mozilla/5.0 (iPhone; CPU iPhone OS 7\_1\_2 like \ldots           US   
         2  Mozilla/5.0 (iPhone; CPU iPhone OS 8\_0\_2 like \ldots           FR   
         3  Mozilla/5.0 (Windows NT 6.1; WOW64) AppleWebKi\ldots           US   
         4  Mozilla/5.0 (Macintosh; Intel Mac OS X 10\_9\_5)\ldots           ES   
         
           Geo\_city\_name Geo\_Region Short\_url\_Cname Encoding\_user\_login  \textbackslash{}
         0           NaN        NaN            j.mp           pontifier   
         1        Ponder         TX          ift.tt               ifttt   
         2           NaN        NaN       1.usa.gov        tweetdeckapi   
         3   Spirit Lake         IA       1.usa.gov           theusnavy   
         4          Vigo         58       1.usa.gov           anonymous   
         
           [Latitude,Longitude]           Referring\_URL   Timestamp         Timezone  \textbackslash{}
         0      [19.43, -99.13]                  direct  1413145138                    
         1  [33.1884, -97.2889]                  direct  1413145138  America/Chicago   
         2        [48.86, 2.35]  http://t.co/tZd7gGVAXC  1413145139     Europe/Paris   
         3  [43.4216, -95.0932]                  direct  1413145140  America/Chicago   
         4   [42.2328, -8.7226]                  direct  1413145140    Europe/Madrid   
         
                                                     Long\_URL  
         0                                http://www.nsa.gov/  
         1  http://alerts.weather.gov/cap/wwacapget.php?x=\ldots  
         2  http://earthobservatory.nasa.gov/IOTD/view.php\ldots  
         3  http://www.navy.mil/submit/display.asp?story\_i\ldots  
         4  http://www.fda.gov/ForConsumers/ByAudience/For\ldots  
\end{Verbatim}
        

    \subsection{8.3 Plotting on a world map}


    With the DataFrame we have, let's plot all the coordinates on a world
map to get an idea of who just visited .gov websites when we excuted the
code above.

    First, we need to learn about the coordinates in the Latitude/Longitude
column.

    \begin{Verbatim}[commandchars=\\\{\}]
{\color{incolor}In [{\color{incolor}101}]:} \PY{c}{\PYZsh{}check the type of data at the first index}
          \PY{n+nb}{type}\PY{p}{(}\PY{n}{USAGovFrame}\PY{p}{[}\PY{l+s}{\PYZsq{}}\PY{l+s}{[Latitude,Longitude]}\PY{l+s}{\PYZsq{}}\PY{p}{]}\PY{p}{[}\PY{l+m+mi}{0}\PY{p}{]}\PY{p}{)}
\end{Verbatim}

            \begin{Verbatim}[commandchars=\\\{\}]
{\color{outcolor}Out[{\color{outcolor}101}]:} list
\end{Verbatim}
        
    I tried to zip and unpack this data the short way, but I received a
``too many values'' error. As you might have figured out by now, it's
time for another function to help us with this process.

    The goal is to pass the column of the DataFrame to a function that will
separate the latitude and the longitude into their own variables in
order to plot them as x and y variables on a world map.

    \begin{Verbatim}[commandchars=\\\{\}]
{\color{incolor}In [{\color{incolor}94}]:} \PY{c}{\PYZsh{}function for unpacking a list in each row of a column into two separate lists}
         \PY{k}{def} \PY{n+nf}{lat\PYZus{}lon}\PY{p}{(}\PY{n}{column}\PY{p}{)}\PY{p}{:}
             
             \PY{c}{\PYZsh{}create two empty lists}
             \PY{n}{lat}\PY{o}{=}\PY{p}{[}\PY{p}{]}
             \PY{n}{lon}\PY{o}{=}\PY{p}{[}\PY{p}{]}
             
             \PY{c}{\PYZsh{}at each row in the column, add first index to lat, second to lon}
             \PY{k}{for} \PY{n}{i} \PY{o+ow}{in} \PY{n}{column}\PY{p}{:}
                 \PY{n}{lat}\PY{o}{.}\PY{n}{append}\PY{p}{(}\PY{n}{i}\PY{p}{[}\PY{l+m+mi}{0}\PY{p}{]}\PY{p}{)}
                 \PY{n}{lon}\PY{o}{.}\PY{n}{append}\PY{p}{(}\PY{n}{i}\PY{p}{[}\PY{l+m+mi}{1}\PY{p}{]}\PY{p}{)}
                 
             \PY{c}{\PYZsh{}return the new lists}
             \PY{k}{return} \PY{n}{lat}\PY{p}{,}\PY{n}{lon}
\end{Verbatim}

    \begin{Verbatim}[commandchars=\\\{\}]
{\color{incolor}In [{\color{incolor}97}]:} \PY{c}{\PYZsh{}open a larger figure so we have a better since of the global web traffic}
         \PY{n}{plt}\PY{o}{.}\PY{n}{figure}\PY{p}{(}\PY{n}{figsize}\PY{o}{=}\PY{p}{(}\PY{l+m+mi}{20}\PY{p}{,}\PY{l+m+mi}{10}\PY{p}{)}\PY{p}{)}
         
         \PY{c}{\PYZsh{} lon\PYZus{}0 is central longitude of projection.}
         \PY{c}{\PYZsh{} resolution = \PYZsq{}c\PYZsq{} means use crude resolution coastlines.}
         \PY{n+nb}{map} \PY{o}{=} \PY{n}{Basemap}\PY{p}{(}\PY{n}{projection}\PY{o}{=}\PY{l+s}{\PYZsq{}}\PY{l+s}{kav7}\PY{l+s}{\PYZsq{}}\PY{p}{,}\PY{n}{lon\PYZus{}0}\PY{o}{=}\PY{l+m+mi}{0}\PY{p}{,}\PY{n}{resolution}\PY{o}{=}\PY{l+s}{\PYZsq{}}\PY{l+s}{c}\PY{l+s}{\PYZsq{}}\PY{p}{)}
         
         \PY{c}{\PYZsh{}bluemarble is a built in function to basemap }
         \PY{n+nb}{map}\PY{o}{.}\PY{n}{bluemarble}\PY{p}{(}\PY{p}{)}
         
         \PY{c}{\PYZsh{}drop the NaN values from the data frame}
         \PY{n}{coords} \PY{o}{=} \PY{n}{USAGovFrame}\PY{p}{[}\PY{l+s}{\PYZsq{}}\PY{l+s}{[Latitude,Longitude]}\PY{l+s}{\PYZsq{}}\PY{p}{]}\PY{o}{.}\PY{n}{dropna}\PY{p}{(}\PY{p}{)}
         
         \PY{c}{\PYZsh{}run the function from the cell above}
         \PY{n}{lat}\PY{p}{,} \PY{n}{lon} \PY{o}{=} \PY{n}{lat\PYZus{}lon}\PY{p}{(}\PY{n}{coords}\PY{p}{)}
         
         \PY{c}{\PYZsh{}plot the points on the bluemarble basemap}
         \PY{c}{\PYZsh{} \PYZsq{}.\PYZsq{} is the marker type, c is the color}
         \PY{n}{x}\PY{p}{,}\PY{n}{y} \PY{o}{=} \PY{n+nb}{map}\PY{p}{(}\PY{n}{lon}\PY{p}{,} \PY{n}{lat}\PY{p}{)}
         \PY{n+nb}{map}\PY{o}{.}\PY{n}{plot}\PY{p}{(}\PY{n}{x}\PY{p}{,}\PY{n}{y}\PY{p}{,}\PY{l+s}{\PYZsq{}}\PY{l+s}{.}\PY{l+s}{\PYZsq{}}\PY{p}{,} \PY{n}{c}\PY{o}{=}\PY{l+s}{\PYZsq{}}\PY{l+s}{red}\PY{l+s}{\PYZsq{}}\PY{p}{,}\PY{n}{markersize}\PY{o}{=}\PY{l+m+mi}{12}\PY{p}{)}
         \PY{n}{plt}\PY{o}{.}\PY{n}{title}\PY{p}{(}\PY{l+s}{\PYZdq{}}\PY{l+s}{USA.Gov Site Traffic}\PY{l+s}{\PYZdq{}}\PY{p}{)}\PY{p}{;}
\end{Verbatim}

    \begin{center}
    \adjustimage{max size={0.9\linewidth}{0.9\paperheight}}{SeniorProjectPDF_files/SeniorProjectPDF_251_0.png}
    \end{center}
    { \hspace*{\fill} \\}
    
    


    \section{Chapter 9 Time Series, More with Data Frames, and Advanced Plotting in
the IPython Notebook}



    \subsection{9.1 Pendulum Data}


    \begin{Verbatim}[commandchars=\\\{\}]
{\color{incolor}In [{\color{incolor}103}]:} \PY{k+kn}{import} \PY{n+nn}{numpy} \PY{k+kn}{as} \PY{n+nn}{np}
          \PY{k+kn}{import} \PY{n+nn}{pandas} \PY{k+kn}{as} \PY{n+nn}{pd}
          \PY{k+kn}{import} \PY{n+nn}{math}
          \PY{o}{\PYZpc{}}\PY{k}{matplotlib} \PY{n}{inline}
          \PY{k+kn}{import} \PY{n+nn}{matplotlib}
          \PY{k+kn}{import} \PY{n+nn}{matplotlib.pyplot} \PY{k+kn}{as} \PY{n+nn}{plt}
          \PY{k+kn}{import} \PY{n+nn}{pylab} \PY{k+kn}{as} \PY{n+nn}{pl}
\end{Verbatim}

    With a pendulum and a web camera, Dr.~Hughes used the Matlab routine
pendulum\_data.m to obtain the pendulum dataset we will be working with.
Additionally, the matlab routine also provided us with the center of the
least-squares best fit circle for the data (1152.57606607623,
394.773399557239), which we will need in the following section.


    \subsubsection{9.1.1 Converting x and y values to angular position (Θ)}


    \begin{Verbatim}[commandchars=\\\{\}]
{\color{incolor}In [{\color{incolor}104}]:} \PY{c}{\PYZsh{}read the csv file}
          \PY{n}{PendData} \PY{o}{=} \PY{n}{np}\PY{o}{.}\PY{n}{loadtxt}\PY{p}{(}\PY{l+s}{\PYZsq{}}\PY{l+s}{data/pend\PYZus{}data.csv}\PY{l+s}{\PYZsq{}}\PY{p}{,} \PY{n}{skiprows}\PY{o}{=}\PY{l+m+mi}{1}\PY{p}{,} \PY{n}{delimiter}\PY{o}{=}\PY{l+s}{\PYZsq{}}\PY{l+s}{,}\PY{l+s}{\PYZsq{}}\PY{p}{)}
          
          \PY{c}{\PYZsh{}convert the numpy structure into a pandas DataFrame}
          \PY{n}{PendFrame} \PY{o}{=} \PY{n}{pd}\PY{o}{.}\PY{n}{DataFrame}\PY{p}{(}\PY{n}{PendData}\PY{p}{,}\PY{n}{columns}\PY{o}{=}\PY{p}{[}\PY{l+s}{\PYZsq{}}\PY{l+s}{time(sec)}\PY{l+s}{\PYZsq{}}\PY{p}{,}\PY{l+s}{\PYZsq{}}\PY{l+s}{x(pixels)}\PY{l+s}{\PYZsq{}}\PY{p}{,}\PY{l+s}{\PYZsq{}}\PY{l+s}{y(pixels)}\PY{l+s}{\PYZsq{}}\PY{p}{]}\PY{p}{)}
          \PY{n}{PendFrame}\PY{o}{.}\PY{n}{head}\PY{p}{(}\PY{p}{)}
\end{Verbatim}

            \begin{Verbatim}[commandchars=\\\{\}]
{\color{outcolor}Out[{\color{outcolor}104}]:}    time(sec)  x(pixels)  y(pixels)
          0    0.00000    347.397     11.801
          1    0.15452    722.760     60.761
          2    0.26375    748.770     66.665
          3    0.32808    598.070     28.983
          4    0.40682    384.457     11.128
\end{Verbatim}
        
    Using the provided data set and center of the circle, let's make some
changes to our data frame.

    First, we need to convert x and y to Θ to obtain a time series in this
format:

Where: 

    Let's write a fucntion that uses this equation to create a new column in
the data frame that contains the computed Θ values.

    \begin{Verbatim}[commandchars=\\\{\}]
{\color{incolor}In [{\color{incolor}105}]:} \PY{c}{\PYZsh{}function takes y and x, then returns the solution to the above equation}
          \PY{k}{def} \PY{n+nf}{calc\PYZus{}theta}\PY{p}{(}\PY{n}{y}\PY{p}{,}\PY{n}{x}\PY{p}{)}\PY{p}{:}
              \PY{k}{return} \PY{p}{(}\PY{p}{(}\PY{l+m+mi}{180}\PY{o}{/}\PY{n}{math}\PY{o}{.}\PY{n}{pi}\PY{p}{)}\PY{o}{*}\PY{n}{math}\PY{o}{.}\PY{n}{atan2}\PY{p}{(}\PY{p}{(}\PY{n}{y}\PY{o}{\PYZhy{}}\PY{l+m+mf}{1152.57606607263}\PY{p}{)}\PY{p}{,}\PY{p}{(}\PY{n}{x}\PY{o}{\PYZhy{}}\PY{l+m+mf}{394.773399557239}\PY{p}{)}\PY{p}{)}\PY{p}{)}\PY{o}{+}\PY{l+m+mi}{90}
          
          \PY{c}{\PYZsh{}the left side creates a new column in the data frame names \PYZdq{}theta\PYZdq{}}
          \PY{c}{\PYZsh{}the right side uses information from the existing columns and the theta function above}
          \PY{c}{\PYZsh{}to create the \PYZdq{}theta\PYZdq{} column}
          \PY{n}{PendFrame}\PY{p}{[}\PY{l+s}{\PYZsq{}}\PY{l+s}{theta}\PY{l+s}{\PYZsq{}}\PY{p}{]}\PY{o}{=} \PY{n}{PendFrame}\PY{o}{.}\PY{n}{apply}\PY{p}{(}\PY{k}{lambda} \PY{n}{row}\PY{p}{:} \PY{n}{calc\PYZus{}theta}\PY{p}{(}\PY{n}{row}\PY{p}{[}\PY{l+s}{\PYZsq{}}\PY{l+s}{y(pixels)}\PY{l+s}{\PYZsq{}}\PY{p}{]}\PY{p}{,}\PY{n}{row}\PY{p}{[}\PY{l+s}{\PYZsq{}}\PY{l+s}{x(pixels)}\PY{l+s}{\PYZsq{}}\PY{p}{]}\PY{p}{)}\PY{p}{,} \PY{n}{axis}\PY{o}{=}\PY{l+m+mi}{1}\PY{p}{)}
          \PY{n}{PendFrame}\PY{o}{.}\PY{n}{head}\PY{p}{(}\PY{p}{)}
\end{Verbatim}

            \begin{Verbatim}[commandchars=\\\{\}]
{\color{outcolor}Out[{\color{outcolor}105}]:}    time(sec)  x(pixels)  y(pixels)      theta
          0    0.00000    347.397     11.801  -2.378128
          1    0.15452    722.760     60.761  16.720526
          2    0.26375    748.770     66.665  18.055472
          3    0.32808    598.070     28.983  10.255820
          4    0.40682    384.457     11.128  -0.517825
\end{Verbatim}
        

    \subsubsection{9.1.2 Graphing the Time series: Angular position vs.~Time}


    Now that the data frame has a time column and a Θ column, let's graph
the time series: 

    \begin{Verbatim}[commandchars=\\\{\}]
{\color{incolor}In [{\color{incolor}106}]:} \PY{c}{\PYZsh{}open a figure window}
          \PY{n}{plt}\PY{o}{.}\PY{n}{figure}\PY{p}{(}\PY{n}{figsize}\PY{o}{=}\PY{p}{(}\PY{l+m+mi}{18}\PY{p}{,}\PY{l+m+mi}{6}\PY{p}{)}\PY{p}{)}
          
          \PY{c}{\PYZsh{}plot the scatterplot first to keep the markers in the foreground}
          \PY{c}{\PYZsh{}s is the size of the markers, and black is the color of the markers}
          \PY{n}{plt}\PY{o}{.}\PY{n}{scatter}\PY{p}{(}\PY{n}{PendFrame}\PY{p}{[}\PY{l+s}{\PYZsq{}}\PY{l+s}{time(sec)}\PY{l+s}{\PYZsq{}}\PY{p}{]}\PY{p}{,}\PY{n}{PendFrame}\PY{p}{[}\PY{l+s}{\PYZsq{}}\PY{l+s}{theta}\PY{l+s}{\PYZsq{}}\PY{p}{]}\PY{p}{,} \PY{n}{s}\PY{o}{=}\PY{l+m+mi}{8}\PY{p}{,} \PY{n}{c}\PY{o}{=}\PY{l+s}{\PYZsq{}}\PY{l+s}{black}\PY{l+s}{\PYZsq{}}\PY{p}{)}
          
          \PY{c}{\PYZsh{}built\PYZhy{}in DataFrame function to plot time as x and theta as y, with custom y limits, line width, and color}
          \PY{n}{TSPlot} \PY{o}{=} \PY{n}{PendFrame}\PY{o}{.}\PY{n}{plot}\PY{p}{(}\PY{n}{x}\PY{o}{=}\PY{l+s}{\PYZsq{}}\PY{l+s}{time(sec)}\PY{l+s}{\PYZsq{}}\PY{p}{,}\PY{n}{y}\PY{o}{=}\PY{l+s}{\PYZsq{}}\PY{l+s}{theta}\PY{l+s}{\PYZsq{}}\PY{p}{,}\PY{n}{ylim}\PY{o}{=}\PY{p}{(}\PY{o}{\PYZhy{}}\PY{l+m+mi}{22}\PY{p}{,}\PY{l+m+mi}{20}\PY{p}{)}\PY{p}{,}\PY{n}{linewidth}\PY{o}{=}\PY{o}{.}\PY{l+m+mi}{7}\PY{p}{,} \PY{n}{c}\PY{o}{=}\PY{l+s}{\PYZsq{}}\PY{l+s}{green}\PY{l+s}{\PYZsq{}}\PY{p}{)}
          
          \PY{c}{\PYZsh{}change background of the plot to white}
          \PY{n}{TSPlot}\PY{o}{.}\PY{n}{set\PYZus{}axis\PYZus{}bgcolor}\PY{p}{(}\PY{l+s}{\PYZsq{}}\PY{l+s}{white}\PY{l+s}{\PYZsq{}}\PY{p}{)}
          
          \PY{c}{\PYZsh{}set x and y labels, title, and adjust thier sizes according}
          \PY{n}{TSPlot}\PY{o}{.}\PY{n}{set\PYZus{}ylabel}\PY{p}{(}\PY{l+s}{\PYZsq{}}\PY{l+s}{pendulum angular position (degrees)}\PY{l+s}{\PYZsq{}}\PY{p}{,} \PY{n}{fontsize}\PY{o}{=}\PY{l+m+mi}{16}\PY{p}{)}
          \PY{n}{TSPlot}\PY{o}{.}\PY{n}{set\PYZus{}xlabel}\PY{p}{(}\PY{l+s}{\PYZsq{}}\PY{l+s}{time (seconds)}\PY{l+s}{\PYZsq{}}\PY{p}{,}\PY{n}{fontsize}\PY{o}{=}\PY{l+m+mi}{16}\PY{p}{)}
          \PY{n}{TSPlot}\PY{o}{.}\PY{n}{set\PYZus{}title}\PY{p}{(}\PY{l+s}{\PYZsq{}}\PY{l+s}{Pendulum Position Time Series}\PY{l+s}{\PYZsq{}}\PY{p}{,}\PY{n}{fontsize}\PY{o}{=}\PY{l+m+mi}{20}\PY{p}{)}
          
          \PY{c}{\PYZsh{}increase the font size of the x and y ticks}
          \PY{n}{plt}\PY{o}{.}\PY{n}{tick\PYZus{}params}\PY{p}{(}\PY{n}{axis}\PY{o}{=}\PY{l+s}{\PYZsq{}}\PY{l+s}{both}\PY{l+s}{\PYZsq{}}\PY{p}{,} \PY{n}{labelsize}\PY{o}{=}\PY{l+m+mi}{15}\PY{p}{)}
\end{Verbatim}

    \begin{center}
    \adjustimage{max size={0.9\linewidth}{0.9\paperheight}}{SeniorProjectPDF_files/SeniorProjectPDF_265_0.png}
    \end{center}
    { \hspace*{\fill} \\}
    

    \subsubsection{9.1.3 Summary Statistics}


    Now that we have a visualization of our dataset, let's look at some
other useful information.

    \begin{Verbatim}[commandchars=\\\{\}]
{\color{incolor}In [{\color{incolor}107}]:} \PY{n}{PendFrame}\PY{o}{.}\PY{n}{describe}\PY{p}{(}\PY{p}{)}
\end{Verbatim}

            \begin{Verbatim}[commandchars=\\\{\}]
{\color{outcolor}Out[{\color{outcolor}107}]:}         time(sec)   x(pixels)   y(pixels)       theta
          count  512.000000  512.000000  512.000000  512.000000
          mean    18.088852  389.999988   24.999988   -0.243205
          std     10.262232  184.034433   14.569337    9.307719
          min      0.000000    2.140000    9.884000  -20.105156
          25\%      9.346825  235.100000   13.858000   -8.029286
          50\%     18.119500  394.293500   20.843000   -0.024068
          75\%     26.951250  551.912500   30.556750    7.897347
          max     35.785000  770.170000   79.954000   19.179884
\end{Verbatim}
        
    From this table we can see that our dataset took measurement of the
pendulums position for a total of 35.785 seconds (since the timespan is
equal to the max-min of the time column). What we do not know is whether
or not the data points are evenly spaced in time. We can look at the
graph above and note there are about 6-7 peaks per 5 second interval. We
can also look at the spacing between the quantiles, the first 25\% of
the data values taking place in a 9.346825 second span, with the
following 25\% taking place in an 8.772675 second span (Q2-Q1), the
following 25\% taking place in an 8.83175 second span (Q3-Q2), and the
final 25\% taking place in an 8.83375 second span. Overall, it looks
pretty close.

    We can apply a smoother to obtain equally spaced ti's, but let's write a
function to assess the time spacing of the data.

    \begin{Verbatim}[commandchars=\\\{\}]
{\color{incolor}In [{\color{incolor}108}]:} \PY{c}{\PYZsh{}this function will create an array with the difference between }
          \PY{c}{\PYZsh{}each point so we can examine if the time intervals are consistent}
          \PY{c}{\PYZsh{}throughout the data, returns a pandas series of differences}
          \PY{k}{def} \PY{n+nf}{spacing\PYZus{}check}\PY{p}{(}\PY{n}{vector}\PY{p}{)}\PY{p}{:}
              
              \PY{c}{\PYZsh{}create a series with one less value than the vector (since the first entry}
              \PY{c}{\PYZsh{}is 2nd\PYZhy{}1st of the vector) to store the difference values}
              \PY{n}{diff} \PY{o}{=} \PY{n}{np}\PY{o}{.}\PY{n}{zeros}\PY{p}{(}\PY{n+nb}{len}\PY{p}{(}\PY{n}{vector}\PY{p}{)}\PY{o}{\PYZhy{}}\PY{l+m+mi}{1}\PY{p}{,}\PY{n}{dtype}\PY{o}{=}\PY{n+nb}{float}\PY{p}{)}
              \PY{n}{difference} \PY{o}{=} \PY{n}{pd}\PY{o}{.}\PY{n}{Series}\PY{p}{(}\PY{n}{diff}\PY{p}{)}
              
              \PY{k}{for} \PY{n}{time} \PY{o+ow}{in} \PY{n+nb}{range}\PY{p}{(}\PY{n+nb}{len}\PY{p}{(}\PY{n}{vector}\PY{p}{)}\PY{o}{\PYZhy{}}\PY{l+m+mi}{1}\PY{p}{)}\PY{p}{:}
                  \PY{n}{difference}\PY{p}{[}\PY{n}{time}\PY{p}{]}\PY{o}{=}\PY{n}{vector}\PY{p}{[}\PY{n}{time}\PY{o}{+}\PY{l+m+mi}{1}\PY{p}{]}\PY{o}{\PYZhy{}}\PY{n}{vector}\PY{p}{[}\PY{n}{time}\PY{p}{]}
              \PY{k}{return} \PY{n}{difference}    
\end{Verbatim}

    Let's create a series of differences using our function.

    \begin{Verbatim}[commandchars=\\\{\}]
{\color{incolor}In [{\color{incolor}109}]:} \PY{n}{TimeSpans} \PY{o}{=} \PY{n}{spacing\PYZus{}check}\PY{p}{(}\PY{n}{PendFrame}\PY{p}{[}\PY{l+s}{\PYZsq{}}\PY{l+s}{time(sec)}\PY{l+s}{\PYZsq{}}\PY{p}{]}\PY{p}{)}
\end{Verbatim}

    There are a couple ways to test that our function is working properly.

    \begin{Verbatim}[commandchars=\\\{\}]
{\color{incolor}In [{\color{incolor}110}]:} \PY{c}{\PYZsh{}this should be equal to 511, since we start by subtracting the second value from the first}
          \PY{k}{print}\PY{p}{(}\PY{n+nb}{len}\PY{p}{(}\PY{n}{TimeSpans}\PY{p}{)}\PY{p}{)}\PY{o}{==}\PY{l+m+mi}{511}
          
          \PY{c}{\PYZsh{}subtracting the first value of our time column from the second value should equal the first value of our difference array}
          \PY{p}{(}\PY{n}{PendFrame}\PY{p}{[}\PY{l+s}{\PYZsq{}}\PY{l+s}{time(sec)}\PY{l+s}{\PYZsq{}}\PY{p}{]}\PY{p}{[}\PY{l+m+mi}{1}\PY{p}{]}\PY{o}{\PYZhy{}}\PY{n}{PendFrame}\PY{p}{[}\PY{l+s}{\PYZsq{}}\PY{l+s}{time(sec)}\PY{l+s}{\PYZsq{}}\PY{p}{]}\PY{p}{[}\PY{l+m+mi}{0}\PY{p}{]}\PY{p}{)}\PY{o}{==}\PY{n}{TimeSpans}\PY{p}{[}\PY{l+m+mi}{0}\PY{p}{]}
\end{Verbatim}

    \begin{Verbatim}[commandchars=\\\{\}]
True
    \end{Verbatim}

            \begin{Verbatim}[commandchars=\\\{\}]
{\color{outcolor}Out[{\color{outcolor}110}]:} True
\end{Verbatim}
        
    Everything appears to be in order. Let's look at a plot of the
difference to see if there is any increasing or decreasing trend in the
time spacing of the data points.

    \begin{Verbatim}[commandchars=\\\{\}]
{\color{incolor}In [{\color{incolor}111}]:} \PY{c}{\PYZsh{}the first argument is a slice that takes the last 511 time }
          \PY{c}{\PYZsh{}values so we can compare them to our 511 differences}
          \PY{c}{\PYZsh{}to see if there is a change over time}
          \PY{n}{fig} \PY{o}{=} \PY{n}{plt}\PY{o}{.}\PY{n}{figure}\PY{p}{(}\PY{n}{figsize}\PY{o}{=}\PY{p}{(}\PY{l+m+mi}{18}\PY{p}{,}\PY{l+m+mi}{6}\PY{p}{)}\PY{p}{)}
          \PY{n}{TimeSpacingPlot} \PY{o}{=} \PY{n}{fig}\PY{o}{.}\PY{n}{add\PYZus{}subplot}\PY{p}{(}\PY{l+m+mi}{1}\PY{p}{,}\PY{l+m+mi}{1}\PY{p}{,}\PY{l+m+mi}{1}\PY{p}{)}
          \PY{n}{plt}\PY{o}{.}\PY{n}{scatter}\PY{p}{(}\PY{n}{x}\PY{o}{=}\PY{n}{PendFrame}\PY{o}{.}\PY{n}{ix}\PY{p}{[}\PY{l+m+mi}{1}\PY{p}{:}\PY{l+m+mi}{511}\PY{p}{,}\PY{l+s}{\PYZsq{}}\PY{l+s}{time(sec)}\PY{l+s}{\PYZsq{}}\PY{p}{]}\PY{p}{,}\PY{n}{y}\PY{o}{=}\PY{n}{TimeSpans}\PY{p}{)}
          
          \PY{c}{\PYZsh{}set x and y labels, title, and adjust thier sizes according}
          \PY{n}{TimeSpacingPlot}\PY{o}{.}\PY{n}{set\PYZus{}ylabel}\PY{p}{(}\PY{l+s}{\PYZsq{}}\PY{l+s}{difference between measurements (seconds)}\PY{l+s}{\PYZsq{}}\PY{p}{,} \PY{n}{fontsize}\PY{o}{=}\PY{l+m+mi}{16}\PY{p}{)}
          \PY{n}{TimeSpacingPlot}\PY{o}{.}\PY{n}{set\PYZus{}xlabel}\PY{p}{(}\PY{l+s}{\PYZsq{}}\PY{l+s}{time (seconds)}\PY{l+s}{\PYZsq{}}\PY{p}{,}\PY{n}{fontsize}\PY{o}{=}\PY{l+m+mi}{16}\PY{p}{)}
          \PY{n}{TimeSpacingPlot}\PY{o}{.}\PY{n}{set\PYZus{}title}\PY{p}{(}\PY{l+s}{\PYZsq{}}\PY{l+s}{Difference in time measurements as time progresses}\PY{l+s}{\PYZsq{}}\PY{p}{,}\PY{n}{fontsize}\PY{o}{=}\PY{l+m+mi}{20}\PY{p}{)}
          \PY{n}{TimeSpacingPlot}\PY{o}{.}\PY{n}{set\PYZus{}ylim}\PY{p}{(}\PY{o}{.}\PY{l+m+mo}{06}\PY{p}{,}\PY{o}{.}\PY{l+m+mi}{157}\PY{p}{)}
          \PY{n}{TimeSpacingPlot}\PY{o}{.}\PY{n}{set\PYZus{}xlim}\PY{p}{(}\PY{l+m+mi}{0}\PY{p}{,}\PY{l+m+mf}{35.9}\PY{p}{)}
          \PY{n}{plt}\PY{o}{.}\PY{n}{tick\PYZus{}params}\PY{p}{(}\PY{n}{axis}\PY{o}{=}\PY{l+s}{\PYZsq{}}\PY{l+s}{both}\PY{l+s}{\PYZsq{}}\PY{p}{,} \PY{n}{labelsize}\PY{o}{=}\PY{l+m+mi}{15}\PY{p}{)}
\end{Verbatim}

    \begin{center}
    \adjustimage{max size={0.9\linewidth}{0.9\paperheight}}{SeniorProjectPDF_files/SeniorProjectPDF_277_0.png}
    \end{center}
    { \hspace*{\fill} \\}
    

    \subsection{9.1.4 Modeling a Sine Wave}


    In section 9.1.2, we graphed the pendulum data and saw a damped sine
wave in the output. To build a model for this, we need a sinusoid with
an exponential decay term.

Where: t = Time, in seconds A = Peak Amplitude f = Frequency, in Hz Φ =
Phase Shift B = Vertical shift

    Now, we need to write a python function. This is one of the easier
functions we have worked with. Simply translate the above function to
code using the math package where necessary (such as math.pi).

    \begin{Verbatim}[commandchars=\\\{\}]
{\color{incolor}In [{\color{incolor}114}]:} \PY{c}{\PYZsh{}function for sinwave}
          \PY{k}{def} \PY{n+nf}{sin\PYZus{}wave}\PY{p}{(}\PY{n}{time}\PY{p}{,}\PY{n}{amp}\PY{p}{,}\PY{n}{freq}\PY{p}{,}\PY{n}{phi}\PY{p}{,}\PY{n}{vertShift}\PY{p}{,}\PY{n}{damp}\PY{p}{)}\PY{p}{:}
              \PY{k}{return} \PY{p}{(}\PY{p}{(}\PY{n}{math}\PY{o}{.}\PY{n}{exp}\PY{p}{(}\PY{o}{\PYZhy{}}\PY{l+m+mi}{1}\PY{o}{*}\PY{n}{damp}\PY{o}{*}\PY{n}{time}\PY{p}{)}\PY{p}{)}\PY{o}{*}\PY{p}{(}\PY{n}{amp}\PY{o}{*}\PY{n}{math}\PY{o}{.}\PY{n}{sin}\PY{p}{(}\PY{l+m+mi}{2}\PY{o}{*}\PY{n}{math}\PY{o}{.}\PY{n}{pi}\PY{o}{*}\PY{n}{freq}\PY{o}{*}\PY{n}{time}\PY{o}{\PYZhy{}}\PY{n}{phi}\PY{p}{)}\PY{o}{+}\PY{n}{vertShift}\PY{p}{)}\PY{p}{)}
\end{Verbatim}

    With this function, we can change the individual values of our Parameter
estimates to find the ``best'' estimate by minimizing SSE. This is
measured by the SSE printed below the next box of code.

    \begin{Verbatim}[commandchars=\\\{\}]
{\color{incolor}In [{\color{incolor}115}]:} \PY{c}{\PYZsh{}changing these values and excuting this code after each change }
          \PY{c}{\PYZsh{}will alter the value. }
          \PY{n}{amp} \PY{o}{=} \PY{l+m+mf}{19.649}
          \PY{n}{freq} \PY{o}{=} \PY{l+m+mf}{1.26082}
          \PY{n}{phi} \PY{o}{=} \PY{o}{.}\PY{l+m+mi}{18205}
          \PY{n}{vertShift} \PY{o}{=} \PY{o}{\PYZhy{}}\PY{o}{.}\PY{l+m+mi}{4705}
          \PY{n}{damp} \PY{o}{=} \PY{o}{.}\PY{l+m+mo}{026}\PY{l+m+mi}{95}
          
          \PY{c}{\PYZsh{}creates a new column in the DataFrame and populates it with sine wave values}
          \PY{n}{PendFrame}\PY{p}{[}\PY{l+s}{\PYZsq{}}\PY{l+s}{SineValues}\PY{l+s}{\PYZsq{}}\PY{p}{]}\PY{o}{=} \PY{n}{PendFrame}\PY{o}{.}\PY{n}{apply}\PY{p}{(}\PY{k}{lambda} \PY{n}{row}\PY{p}{:} \PY{n}{sin\PYZus{}wave}\PY{p}{(}\PY{n}{row}\PY{p}{[}\PY{l+s}{\PYZsq{}}\PY{l+s}{time(sec)}\PY{l+s}{\PYZsq{}}\PY{p}{]}\PY{p}{,}\PY{n}{amp}\PY{p}{,}\PY{n}{freq}\PY{p}{,}\PY{n}{phi}\PY{p}{,}\PY{n}{vertShift}\PY{p}{,}\PY{n}{damp}\PY{p}{)}\PY{p}{,} \PY{n}{axis}\PY{o}{=}\PY{l+m+mi}{1}\PY{p}{)}
          
          \PY{c}{\PYZsh{}check sum of squares to see if error is decreasing}
          \PY{n}{SSE} \PY{o}{=} \PY{n}{np}\PY{o}{.}\PY{n}{sum}\PY{p}{(}\PY{p}{(}\PY{n}{PendFrame}\PY{p}{[}\PY{l+s}{\PYZsq{}}\PY{l+s}{theta}\PY{l+s}{\PYZsq{}}\PY{p}{]}\PY{o}{\PYZhy{}}\PY{n}{PendFrame}\PY{p}{[}\PY{l+s}{\PYZsq{}}\PY{l+s}{SineValues}\PY{l+s}{\PYZsq{}}\PY{p}{]}\PY{p}{)}\PY{o}{*}\PY{o}{*}\PY{l+m+mi}{2}\PY{p}{)}
          \PY{k}{print}\PY{p}{(}\PY{l+s}{\PYZdq{}}\PY{l+s}{SSE:}\PY{l+s}{\PYZdq{}}\PY{p}{,}\PY{n}{SSE}\PY{p}{)}
\end{Verbatim}

    \begin{Verbatim}[commandchars=\\\{\}]
('SSE:', 983.71108227071329)
    \end{Verbatim}

    As you can see above, my estimated Parameters are fairly precise.
Playing around with them might yield an even lower SSE, but it shouldn't
be much different from my printed value.

    Let's plot the model we just made against the graph from Section 9.1.2
and see how they compare.

    \begin{Verbatim}[commandchars=\\\{\}]
{\color{incolor}In [{\color{incolor}119}]:} \PY{c}{\PYZsh{}open a figure window and color it white}
          \PY{n}{plt}\PY{o}{.}\PY{n}{figure}\PY{p}{(}\PY{n}{figsize}\PY{o}{=}\PY{p}{(}\PY{l+m+mi}{18}\PY{p}{,}\PY{l+m+mi}{6}\PY{p}{)}\PY{p}{)}
          
          \PY{c}{\PYZsh{}original points}
          \PY{n}{plt}\PY{o}{.}\PY{n}{scatter}\PY{p}{(}\PY{n}{PendFrame}\PY{p}{[}\PY{l+s}{\PYZsq{}}\PY{l+s}{time(sec)}\PY{l+s}{\PYZsq{}}\PY{p}{]}\PY{p}{,}\PY{n}{PendFrame}\PY{p}{[}\PY{l+s}{\PYZsq{}}\PY{l+s}{theta}\PY{l+s}{\PYZsq{}}\PY{p}{]}\PY{p}{,} \PY{n}{s}\PY{o}{=}\PY{l+m+mi}{8}\PY{p}{,} \PY{n}{c}\PY{o}{=}\PY{l+s}{\PYZsq{}}\PY{l+s}{black}\PY{l+s}{\PYZsq{}}\PY{p}{)}
          
          \PY{c}{\PYZsh{}original line}
          \PY{n}{TSPlot} \PY{o}{=} \PY{n}{PendFrame}\PY{o}{.}\PY{n}{plot}\PY{p}{(}\PY{n}{x}\PY{o}{=}\PY{l+s}{\PYZsq{}}\PY{l+s}{time(sec)}\PY{l+s}{\PYZsq{}}\PY{p}{,}\PY{n}{y}\PY{o}{=}\PY{l+s}{\PYZsq{}}\PY{l+s}{theta}\PY{l+s}{\PYZsq{}}\PY{p}{,}\PY{n}{ylim}\PY{o}{=}\PY{p}{(}\PY{o}{\PYZhy{}}\PY{l+m+mi}{22}\PY{p}{,}\PY{l+m+mi}{20}\PY{p}{)}\PY{p}{,}\PY{n}{linewidth}\PY{o}{=}\PY{o}{.}\PY{l+m+mi}{7}\PY{p}{,} \PY{n}{c}\PY{o}{=}\PY{l+s}{\PYZsq{}}\PY{l+s}{red}\PY{l+s}{\PYZsq{}}\PY{p}{)}
          
          \PY{c}{\PYZsh{}my model line}
          \PY{n}{PendFrame}\PY{o}{.}\PY{n}{plot}\PY{p}{(}\PY{n}{x}\PY{o}{=}\PY{l+s}{\PYZsq{}}\PY{l+s}{time(sec)}\PY{l+s}{\PYZsq{}}\PY{p}{,}\PY{n}{y}\PY{o}{=}\PY{l+s}{\PYZsq{}}\PY{l+s}{SineValues}\PY{l+s}{\PYZsq{}}\PY{p}{,}\PY{n}{ylim}\PY{o}{=}\PY{p}{(}\PY{o}{\PYZhy{}}\PY{l+m+mi}{22}\PY{p}{,}\PY{l+m+mi}{22}\PY{p}{)}\PY{p}{,}\PY{n}{linewidth}\PY{o}{=}\PY{o}{.}\PY{l+m+mi}{7}\PY{p}{,} \PY{n}{c}\PY{o}{=}\PY{l+s}{\PYZsq{}}\PY{l+s}{blue}\PY{l+s}{\PYZsq{}}\PY{p}{)}
          
          \PY{c}{\PYZsh{}change background to white}
          \PY{n}{TSPlot}\PY{o}{.}\PY{n}{set\PYZus{}axis\PYZus{}bgcolor}\PY{p}{(}\PY{l+s}{\PYZsq{}}\PY{l+s}{white}\PY{l+s}{\PYZsq{}}\PY{p}{)}
          
          \PY{c}{\PYZsh{}set x and y labels, title, and adjust thier sizes according}
          \PY{n}{TSPlot}\PY{o}{.}\PY{n}{set\PYZus{}ylabel}\PY{p}{(}\PY{l+s}{\PYZsq{}}\PY{l+s}{pendulum angular position (degrees)}\PY{l+s}{\PYZsq{}}\PY{p}{,} \PY{n}{fontsize}\PY{o}{=}\PY{l+m+mi}{16}\PY{p}{)}
          \PY{n}{TSPlot}\PY{o}{.}\PY{n}{set\PYZus{}xlabel}\PY{p}{(}\PY{l+s}{\PYZsq{}}\PY{l+s}{time (seconds)}\PY{l+s}{\PYZsq{}}\PY{p}{,}\PY{n}{fontsize}\PY{o}{=}\PY{l+m+mi}{16}\PY{p}{)}
          \PY{n}{TSPlot}\PY{o}{.}\PY{n}{set\PYZus{}title}\PY{p}{(}\PY{l+s}{\PYZsq{}}\PY{l+s}{Pendulum Position Time Series: Model Fit and Observed Fit}\PY{l+s}{\PYZsq{}}\PY{p}{,}\PY{n}{fontsize}\PY{o}{=}\PY{l+m+mi}{20}\PY{p}{)}
          \PY{n}{TSPlot}\PY{o}{.}\PY{n}{legend}\PY{p}{(}\PY{p}{[}\PY{l+s}{\PYZsq{}}\PY{l+s}{Fit Data}\PY{l+s}{\PYZsq{}}\PY{p}{,}\PY{l+s}{\PYZsq{}}\PY{l+s}{Approximation}\PY{l+s}{\PYZsq{}}\PY{p}{,}\PY{l+s}{\PYZsq{}}\PY{l+s}{Data Points (time,theta)}\PY{l+s}{\PYZsq{}}\PY{p}{]}\PY{p}{,}\PY{l+m+mi}{1}\PY{p}{)}
          \PY{n}{plt}\PY{o}{.}\PY{n}{tick\PYZus{}params}\PY{p}{(}\PY{n}{axis}\PY{o}{=}\PY{l+s}{\PYZsq{}}\PY{l+s}{both}\PY{l+s}{\PYZsq{}}\PY{p}{,} \PY{n}{labelsize}\PY{o}{=}\PY{l+m+mi}{15}\PY{p}{)}
\end{Verbatim}

    \begin{center}
    \adjustimage{max size={0.9\linewidth}{0.9\paperheight}}{SeniorProjectPDF_files/SeniorProjectPDF_286_0.png}
    \end{center}
    { \hspace*{\fill} \\}
    
    Not too shabby. They appear to match the least toward the end of the
time period.

    We can look at the residuals to see where the model matches and fails to
match the pendulum data.

    \begin{Verbatim}[commandchars=\\\{\}]
{\color{incolor}In [{\color{incolor}121}]:} \PY{c}{\PYZsh{}residuals are the observed values at ti \PYZhy{} the values from model at ti}
          \PY{k}{def} \PY{n+nf}{calc\PYZus{}resid}\PY{p}{(}\PY{n}{obs}\PY{p}{,}\PY{n}{pred}\PY{p}{)}\PY{p}{:}
              \PY{k}{return} \PY{n}{obs}\PY{o}{\PYZhy{}}\PY{n}{pred}
          
          \PY{c}{\PYZsh{}creates a new column in the DataFrame and populates it with residuals}
          \PY{n}{PendFrame}\PY{p}{[}\PY{l+s}{\PYZsq{}}\PY{l+s}{Residuals}\PY{l+s}{\PYZsq{}}\PY{p}{]}\PY{o}{=} \PY{n}{PendFrame}\PY{o}{.}\PY{n}{apply}\PY{p}{(}\PY{k}{lambda} \PY{n}{row}\PY{p}{:} \PY{n}{calc\PYZus{}resid}\PY{p}{(}\PY{n}{row}\PY{p}{[}\PY{l+s}{\PYZsq{}}\PY{l+s}{theta}\PY{l+s}{\PYZsq{}}\PY{p}{]}\PY{p}{,}\PY{n}{row}\PY{p}{[}\PY{l+s}{\PYZsq{}}\PY{l+s}{SineValues}\PY{l+s}{\PYZsq{}}\PY{p}{]}\PY{p}{)}\PY{p}{,} \PY{n}{axis}\PY{o}{=}\PY{l+m+mi}{1}\PY{p}{)}
\end{Verbatim}

    \begin{Verbatim}[commandchars=\\\{\}]
{\color{incolor}In [{\color{incolor}122}]:} \PY{c}{\PYZsh{}open a figure window and color it white}
          \PY{n}{plt}\PY{o}{.}\PY{n}{figure}\PY{p}{(}\PY{n}{figsize}\PY{o}{=}\PY{p}{(}\PY{l+m+mi}{18}\PY{p}{,}\PY{l+m+mi}{6}\PY{p}{)}\PY{p}{)}
          
          \PY{c}{\PYZsh{}plot the scatterplot first to keep the markers in the foreground}
          \PY{n}{plt}\PY{o}{.}\PY{n}{scatter}\PY{p}{(}\PY{n}{PendFrame}\PY{p}{[}\PY{l+s}{\PYZsq{}}\PY{l+s}{time(sec)}\PY{l+s}{\PYZsq{}}\PY{p}{]}\PY{p}{,}\PY{n}{PendFrame}\PY{p}{[}\PY{l+s}{\PYZsq{}}\PY{l+s}{Residuals}\PY{l+s}{\PYZsq{}}\PY{p}{]}\PY{p}{,} \PY{n}{s}\PY{o}{=}\PY{l+m+mi}{8}\PY{p}{,} \PY{n}{c}\PY{o}{=}\PY{l+s}{\PYZsq{}}\PY{l+s}{black}\PY{l+s}{\PYZsq{}}\PY{p}{)}
          
          \PY{c}{\PYZsh{}built\PYZhy{}in DataFrame function to plot time as x and theta as y, with custom y limits, line width, and color}
          \PY{n}{TSPlot} \PY{o}{=} \PY{n}{PendFrame}\PY{o}{.}\PY{n}{plot}\PY{p}{(}\PY{n}{x}\PY{o}{=}\PY{l+s}{\PYZsq{}}\PY{l+s}{time(sec)}\PY{l+s}{\PYZsq{}}\PY{p}{,}\PY{n}{y}\PY{o}{=}\PY{l+s}{\PYZsq{}}\PY{l+s}{Residuals}\PY{l+s}{\PYZsq{}}\PY{p}{,}\PY{n}{linewidth}\PY{o}{=}\PY{o}{.}\PY{l+m+mi}{7}\PY{p}{,} \PY{n}{c}\PY{o}{=}\PY{l+s}{\PYZsq{}}\PY{l+s}{green}\PY{l+s}{\PYZsq{}}\PY{p}{)}
          
          \PY{c}{\PYZsh{}set x and y labels, title, and adjust thier sizes according}
          \PY{n}{TSPlot}\PY{o}{.}\PY{n}{set\PYZus{}ylabel}\PY{p}{(}\PY{l+s}{\PYZsq{}}\PY{l+s}{model residuals (degrees)}\PY{l+s}{\PYZsq{}}\PY{p}{,} \PY{n}{fontsize}\PY{o}{=}\PY{l+m+mi}{16}\PY{p}{)}
          \PY{n}{TSPlot}\PY{o}{.}\PY{n}{set\PYZus{}xlabel}\PY{p}{(}\PY{l+s}{\PYZsq{}}\PY{l+s}{time (seconds)}\PY{l+s}{\PYZsq{}}\PY{p}{,}\PY{n}{fontsize}\PY{o}{=}\PY{l+m+mi}{16}\PY{p}{)}
          \PY{n}{TSPlot}\PY{o}{.}\PY{n}{set\PYZus{}title}\PY{p}{(}\PY{l+s}{\PYZsq{}}\PY{l+s}{Residuals as time progresses}\PY{l+s}{\PYZsq{}}\PY{p}{,}\PY{n}{fontsize}\PY{o}{=}\PY{l+m+mi}{20}\PY{p}{)}
          \PY{n}{plt}\PY{o}{.}\PY{n}{tick\PYZus{}params}\PY{p}{(}\PY{n}{axis}\PY{o}{=}\PY{l+s}{\PYZsq{}}\PY{l+s}{both}\PY{l+s}{\PYZsq{}}\PY{p}{,} \PY{n}{labelsize}\PY{o}{=}\PY{l+m+mi}{15}\PY{p}{)}
\end{Verbatim}

    \begin{center}
    \adjustimage{max size={0.9\linewidth}{0.9\paperheight}}{SeniorProjectPDF_files/SeniorProjectPDF_290_0.png}
    \end{center}
    { \hspace*{\fill} \\}
    
    My visual inspection was confirmed by the residual plot. Additionally,
we can see other areas where the model fails to capture the exact
movement of the pendulum (such as the first 3 seconds).


    \subsection{9.2 Geiger Counter Data}


    This is another data set provided by Dr.~Hughes. It is the result of
putting radioactive material next to a geiger counter. The data was
collected using Matlab. The variables are the number of detections in a
5 second period, for over 22 hours, and the timestamp.

    \begin{Verbatim}[commandchars=\\\{\}]
{\color{incolor}In [{\color{incolor}129}]:} \PY{k+kn}{import} \PY{n+nn}{matplotlib.dates} \PY{k+kn}{as} \PY{n+nn}{mdates}
          \PY{k+kn}{from} \PY{n+nn}{datetime} \PY{k+kn}{import} \PY{n}{datetime}
\end{Verbatim}


    \subsubsection{9.2.1 Converting time and date stamps for plotting}


    \begin{Verbatim}[commandchars=\\\{\}]
{\color{incolor}In [{\color{incolor}130}]:} \PY{n}{gcFrame} \PY{o}{=} \PY{n}{pd}\PY{o}{.}\PY{n}{read\PYZus{}table}\PY{p}{(}\PY{l+s}{\PYZdq{}}\PY{l+s}{data/source\PYZus{}radioactivity.txt}\PY{l+s}{\PYZdq{}}\PY{p}{,}\PY{n}{names}\PY{o}{=}\PY{p}{[}\PY{l+s}{\PYZsq{}}\PY{l+s}{Date/Time}\PY{l+s}{\PYZsq{}}\PY{p}{,}\PY{l+s}{\PYZsq{}}\PY{l+s}{Detections}\PY{l+s}{\PYZsq{}}\PY{p}{]}\PY{p}{)}
          
          \PY{c}{\PYZsh{}check to make sure we\PYZsq{}re working with the \PYZdq{}entire\PYZdq{} data set}
          \PY{n+nb}{len}\PY{p}{(}\PY{n}{gcFrame}\PY{o}{.}\PY{n}{Detections}\PY{p}{)}
\end{Verbatim}

            \begin{Verbatim}[commandchars=\\\{\}]
{\color{outcolor}Out[{\color{outcolor}130}]:} 16384
\end{Verbatim}
        
    \begin{Verbatim}[commandchars=\\\{\}]
{\color{incolor}In [{\color{incolor}131}]:} \PY{n}{gcFrame}\PY{o}{.}\PY{n}{head}\PY{p}{(}\PY{p}{)}
\end{Verbatim}

            \begin{Verbatim}[commandchars=\\\{\}]
{\color{outcolor}Out[{\color{outcolor}131}]:}              Date/Time  Detections
          0  09/23/2013 17:26:27           5
          1  09/23/2013 17:26:32           9
          2  09/23/2013 17:26:37           8
          3  09/23/2013 17:26:42           7
          4  09/23/2013 17:26:47           9
\end{Verbatim}
        
    \begin{Verbatim}[commandchars=\\\{\}]
{\color{incolor}In [{\color{incolor}135}]:} \PY{c}{\PYZsh{}convert the timestamp to a time date2num can use}
          \PY{c}{\PYZsh{}the \PYZpc{} and punctuation is written exactly how the information appears in the df}
          \PY{n}{gcFrame}\PY{p}{[}\PY{l+s}{\PYZsq{}}\PY{l+s}{Time}\PY{l+s}{\PYZsq{}}\PY{p}{]} \PY{o}{=} \PY{p}{[}\PY{n}{datetime}\PY{o}{.}\PY{n}{strptime}\PY{p}{(}\PY{n}{t}\PY{p}{,} \PY{l+s}{\PYZdq{}}\PY{l+s}{\PYZpc{}}\PY{l+s}{m/}\PY{l+s+si}{\PYZpc{}d}\PY{l+s}{/}\PY{l+s}{\PYZpc{}}\PY{l+s}{Y }\PY{l+s}{\PYZpc{}}\PY{l+s}{H:}\PY{l+s}{\PYZpc{}}\PY{l+s}{M:}\PY{l+s}{\PYZpc{}}\PY{l+s}{S}\PY{l+s}{\PYZdq{}}\PY{p}{)} \PY{k}{for} \PY{n}{t} \PY{o+ow}{in} \PY{n}{gcFrame}\PY{p}{[}\PY{l+s}{\PYZsq{}}\PY{l+s}{Date/Time}\PY{l+s}{\PYZsq{}}\PY{p}{]}\PY{p}{]}
          
          \PY{c}{\PYZsh{}the scatter column is a conversion of the time column to a number we can }
          \PY{c}{\PYZsh{}use to graph the points on the plot of the time series}
          \PY{n}{gcFrame}\PY{p}{[}\PY{l+s}{\PYZsq{}}\PY{l+s}{Scatter}\PY{l+s}{\PYZsq{}}\PY{p}{]} \PY{o}{=} \PY{n}{mdates}\PY{o}{.}\PY{n}{date2num}\PY{p}{(}\PY{n}{gcFrame}\PY{p}{[}\PY{l+s}{\PYZsq{}}\PY{l+s}{Time}\PY{l+s}{\PYZsq{}}\PY{p}{]}\PY{p}{)}
\end{Verbatim}

    \begin{Verbatim}[commandchars=\\\{\}]
{\color{incolor}In [{\color{incolor}136}]:} \PY{n}{gcFrame}\PY{o}{.}\PY{n}{head}\PY{p}{(}\PY{p}{)}
\end{Verbatim}

            \begin{Verbatim}[commandchars=\\\{\}]
{\color{outcolor}Out[{\color{outcolor}136}]:}              Date/Time  Detections                Time        Scatter
          0  09/23/2013 17:26:27           5 2013-09-23 17:26:27  735134.726701
          1  09/23/2013 17:26:32           9 2013-09-23 17:26:32  735134.726759
          2  09/23/2013 17:26:37           8 2013-09-23 17:26:37  735134.726817
          3  09/23/2013 17:26:42           7 2013-09-23 17:26:42  735134.726875
          4  09/23/2013 17:26:47           9 2013-09-23 17:26:47  735134.726933
\end{Verbatim}
        

    \subsubsection{9.2.2 Plotting the Geiger counter time series with points}


    There are many plotting options, including ones specifically for time
series data. I continued to have the best luck with the Panda's
DataFrame and adjusted my plotting accordingly. Here's the code for a
plot of the time series with the individual 16384 points. Change the
line widths, colors, and sizes in the code below to customize your own
time series plot.

    \begin{Verbatim}[commandchars=\\\{\}]
{\color{incolor}In [{\color{incolor}138}]:} \PY{c}{\PYZsh{}open a figure window and color it white}
          \PY{n}{figure} \PY{o}{=} \PY{n}{plt}\PY{o}{.}\PY{n}{figure}\PY{p}{(}\PY{n}{figsize}\PY{o}{=}\PY{p}{(}\PY{l+m+mi}{20}\PY{p}{,}\PY{l+m+mi}{6}\PY{p}{)}\PY{p}{)}
          
          \PY{c}{\PYZsh{}built\PYZhy{}in DataFrame function to plot time as x and theta as y, with custom y limits,}
          \PY{c}{\PYZsh{}line width, and color}
          \PY{n}{TSPlotG} \PY{o}{=} \PY{n}{gcFrame}\PY{o}{.}\PY{n}{plot}\PY{p}{(}\PY{n}{x}\PY{o}{=}\PY{l+s}{\PYZsq{}}\PY{l+s}{Scatter}\PY{l+s}{\PYZsq{}}\PY{p}{,}\PY{n}{y}\PY{o}{=}\PY{l+s}{\PYZsq{}}\PY{l+s}{Detections}\PY{l+s}{\PYZsq{}}\PY{p}{,}\PY{n}{linewidth}\PY{o}{=}\PY{o}{.}\PY{l+m+mi}{1}\PY{p}{,} \PY{n}{ylim}\PY{o}{=}\PY{p}{(}\PY{l+m+mi}{0}\PY{p}{,}\PY{l+m+mi}{23}\PY{p}{)}\PY{p}{,} \PY{n}{c}\PY{o}{=}\PY{l+s}{\PYZsq{}}\PY{l+s}{green}\PY{l+s}{\PYZsq{}}\PY{p}{)}
          
          \PY{n}{TSPlotG}\PY{o}{.}\PY{n}{set\PYZus{}axis\PYZus{}bgcolor}\PY{p}{(}\PY{l+s}{\PYZsq{}}\PY{l+s}{white}\PY{l+s}{\PYZsq{}}\PY{p}{)}   \PY{c}{\PYZsh{}change background to white}
          
          \PY{c}{\PYZsh{}set x and y labels, title, and adjust thier sizes according}
          \PY{n}{TSPlotG}\PY{o}{.}\PY{n}{set\PYZus{}ylabel}\PY{p}{(}\PY{l+s}{\PYZsq{}}\PY{l+s}{\PYZsh{} of Detections in 5 Seconds}\PY{l+s}{\PYZsq{}}\PY{p}{,} \PY{n}{fontsize}\PY{o}{=}\PY{l+m+mi}{16}\PY{p}{)}
          \PY{n}{TSPlotG}\PY{o}{.}\PY{n}{set\PYZus{}xlabel}\PY{p}{(}\PY{l+s}{\PYZsq{}}\PY{l+s}{Date/Time}\PY{l+s}{\PYZsq{}}\PY{p}{,}\PY{n}{fontsize}\PY{o}{=}\PY{l+m+mi}{16}\PY{p}{)}
          \PY{n}{TSPlotG}\PY{o}{.}\PY{n}{set\PYZus{}title}\PY{p}{(}\PY{l+s}{\PYZsq{}}\PY{l+s}{Geiger Counter Source Radioactivity Time Series}\PY{l+s}{\PYZsq{}}\PY{p}{,}\PY{n}{fontsize}\PY{o}{=}\PY{l+m+mi}{20}\PY{p}{)}
          \PY{n}{plt}\PY{o}{.}\PY{n}{tick\PYZus{}params}\PY{p}{(}\PY{n}{axis}\PY{o}{=}\PY{l+s}{\PYZsq{}}\PY{l+s}{both}\PY{l+s}{\PYZsq{}}\PY{p}{,} \PY{n}{labelsize}\PY{o}{=}\PY{l+m+mi}{15}\PY{p}{)}   \PY{c}{\PYZsh{}change the font size of the axes ticks}
          
          \PY{n}{xlabels} \PY{o}{=} \PY{p}{[}\PY{p}{]}
          \PY{n}{xticks} \PY{o}{=} \PY{p}{[}\PY{p}{]}
          \PY{k}{for} \PY{n}{i} \PY{o+ow}{in} \PY{n+nb}{range}\PY{p}{(}\PY{l+m+mi}{40}\PY{p}{)}\PY{p}{:}
              \PY{c}{\PYZsh{}populate a list of 41 date/times with even time intervals over our 16834 points}
              \PY{n}{xlabels}\PY{o}{.}\PY{n}{append}\PY{p}{(}\PY{n}{gcFrame}\PY{p}{[}\PY{l+s}{\PYZsq{}}\PY{l+s}{Date/Time}\PY{l+s}{\PYZsq{}}\PY{p}{]}\PY{p}{[}\PY{n}{i}\PY{o}{*}\PY{l+m+mi}{420}\PY{p}{]}\PY{p}{)}
              
              \PY{c}{\PYZsh{}an array to place the date/time labels at the corresponding x value}
              \PY{n}{xticks}\PY{o}{.}\PY{n}{append}\PY{p}{(}\PY{n}{gcFrame}\PY{p}{[}\PY{l+s}{\PYZsq{}}\PY{l+s}{Scatter}\PY{l+s}{\PYZsq{}}\PY{p}{]}\PY{p}{[}\PY{n}{i}\PY{o}{*}\PY{l+m+mi}{420}\PY{p}{]}\PY{p}{)}     
          
          \PY{c}{\PYZsh{}place the date/time labels and rotate them}
          \PY{n}{TSPlotG}\PY{o}{.}\PY{n}{set\PYZus{}xticklabels}\PY{p}{(}\PY{n}{xlabels}\PY{p}{,} \PY{n}{rotation}\PY{o}{=}\PY{l+m+mi}{90}\PY{p}{,} \PY{n}{fontsize}\PY{o}{=}\PY{l+m+mi}{15}\PY{p}{)}    
          \PY{n}{TSPlotG}\PY{o}{.}\PY{n}{set\PYZus{}xticks}\PY{p}{(}\PY{n}{xticks}\PY{p}{)}
          
          \PY{c}{\PYZsh{}add the points to the graph}
          \PY{n}{plt}\PY{o}{.}\PY{n}{scatter}\PY{p}{(}\PY{n}{gcFrame}\PY{p}{[}\PY{l+s}{\PYZsq{}}\PY{l+s}{Scatter}\PY{l+s}{\PYZsq{}}\PY{p}{]}\PY{p}{,}\PY{n}{gcFrame}\PY{p}{[}\PY{l+s}{\PYZsq{}}\PY{l+s}{Detections}\PY{l+s}{\PYZsq{}}\PY{p}{]}\PY{p}{,} \PY{n}{s}\PY{o}{=}\PY{l+m+mf}{2.5}\PY{p}{,} \PY{n}{color}\PY{o}{=}\PY{l+s}{\PYZsq{}}\PY{l+s}{black}\PY{l+s}{\PYZsq{}}\PY{p}{)}\PY{p}{;}
\end{Verbatim}

    \begin{center}
    \adjustimage{max size={0.9\linewidth}{0.9\paperheight}}{SeniorProjectPDF_files/SeniorProjectPDF_302_0.png}
    \end{center}
    { \hspace*{\fill} \\}
    

    \subsubsection{9.2.3 Moving Averages with Geiger Counter Data}


    When we looked at the Pendulum data, we were worried about a constant
Δt. I provided an, arguably, uncessary way to evalute the change from
one ti to the next. If the time intervals are not equally spaced, you
can create your own intervals in specified n-sized windows using a
smoother.

    Below is the simple moving average, or as Pandas calls it the
``roling\_mean''.

    \begin{Verbatim}[commandchars=\\\{\}]
{\color{incolor}In [{\color{incolor}141}]:} \PY{c}{\PYZsh{}pandas has a built\PYZhy{}in rolling means function}
          \PY{n}{gcFrame}\PY{p}{[}\PY{l+s}{\PYZsq{}}\PY{l+s}{SMA}\PY{l+s}{\PYZsq{}}\PY{p}{]} \PY{o}{=} \PY{n}{pd}\PY{o}{.}\PY{n}{rolling\PYZus{}mean}\PY{p}{(}\PY{n}{gcFrame}\PY{o}{.}\PY{n}{Detections}\PY{p}{,}\PY{n}{window}\PY{o}{=}\PY{l+m+mi}{24}\PY{p}{)}
\end{Verbatim}

    Plot the rolling mean over the original geiger counter data.

    \begin{Verbatim}[commandchars=\\\{\}]
{\color{incolor}In [{\color{incolor}143}]:} \PY{c}{\PYZsh{}open a figure window and color it white}
          \PY{n}{figure} \PY{o}{=} \PY{n}{plt}\PY{o}{.}\PY{n}{figure}\PY{p}{(}\PY{n}{figsize}\PY{o}{=}\PY{p}{(}\PY{l+m+mi}{20}\PY{p}{,}\PY{l+m+mi}{8}\PY{p}{)}\PY{p}{)}
          
          \PY{c}{\PYZsh{}built\PYZhy{}in DataFrame function to plot time as x and theta as y, with custom y limits,}
          \PY{c}{\PYZsh{}line width, and color}
          \PY{n}{TSPlotG2} \PY{o}{=} \PY{n}{gcFrame}\PY{o}{.}\PY{n}{plot}\PY{p}{(}\PY{n}{x}\PY{o}{=}\PY{l+s}{\PYZsq{}}\PY{l+s}{Scatter}\PY{l+s}{\PYZsq{}}\PY{p}{,}\PY{n}{y}\PY{o}{=}\PY{l+s}{\PYZsq{}}\PY{l+s}{Detections}\PY{l+s}{\PYZsq{}}\PY{p}{,}\PY{n}{linewidth}\PY{o}{=}\PY{l+m+mi}{1}\PY{p}{,} \PY{n}{ylim}\PY{o}{=}\PY{p}{(}\PY{l+m+mi}{0}\PY{p}{,}\PY{l+m+mi}{23}\PY{p}{)}\PY{p}{,} \PY{n}{c}\PY{o}{=}\PY{l+s}{\PYZsq{}}\PY{l+s}{black}\PY{l+s}{\PYZsq{}}\PY{p}{)}
          
          \PY{n}{Line2} \PY{o}{=} \PY{n}{gcFrame}\PY{o}{.}\PY{n}{plot}\PY{p}{(}\PY{n}{x}\PY{o}{=}\PY{l+s}{\PYZsq{}}\PY{l+s}{Scatter}\PY{l+s}{\PYZsq{}}\PY{p}{,}\PY{n}{y}\PY{o}{=}\PY{l+s}{\PYZsq{}}\PY{l+s}{SMA}\PY{l+s}{\PYZsq{}}\PY{p}{,}\PY{n}{linewidth}\PY{o}{=}\PY{l+m+mi}{1}\PY{p}{,} \PY{n}{c}\PY{o}{=}\PY{l+s}{\PYZsq{}}\PY{l+s}{red}\PY{l+s}{\PYZsq{}}\PY{p}{)}
          
          \PY{c}{\PYZsh{}set x and y labels, title, and adjust thier sizes according}
          \PY{n}{TSPlotG2}\PY{o}{.}\PY{n}{set\PYZus{}ylabel}\PY{p}{(}\PY{l+s}{\PYZsq{}}\PY{l+s}{\PYZsh{} of Detections in 5 Seconds}\PY{l+s}{\PYZsq{}}\PY{p}{,} \PY{n}{fontsize}\PY{o}{=}\PY{l+m+mi}{16}\PY{p}{)}
          \PY{n}{TSPlotG2}\PY{o}{.}\PY{n}{set\PYZus{}xlabel}\PY{p}{(}\PY{l+s}{\PYZsq{}}\PY{l+s}{Date/Time}\PY{l+s}{\PYZsq{}}\PY{p}{,}\PY{n}{fontsize}\PY{o}{=}\PY{l+m+mi}{16}\PY{p}{)}
          \PY{n}{TSPlotG2}\PY{o}{.}\PY{n}{set\PYZus{}title}\PY{p}{(}\PY{l+s}{\PYZsq{}}\PY{l+s}{Geiger Counter Source Radioactivity Time Series}\PY{l+s}{\PYZsq{}}\PY{p}{,}\PY{n}{fontsize}\PY{o}{=}\PY{l+m+mi}{20}\PY{p}{)}
          \PY{n}{plt}\PY{o}{.}\PY{n}{tick\PYZus{}params}\PY{p}{(}\PY{n}{axis}\PY{o}{=}\PY{l+s}{\PYZsq{}}\PY{l+s}{both}\PY{l+s}{\PYZsq{}}\PY{p}{,} \PY{n}{labelsize}\PY{o}{=}\PY{l+m+mi}{15}\PY{p}{)}   \PY{c}{\PYZsh{}change the font size of the axes ticks}
          
          \PY{n}{xlabels} \PY{o}{=} \PY{p}{[}\PY{p}{]}
          \PY{n}{xticks} \PY{o}{=} \PY{p}{[}\PY{p}{]}
          \PY{k}{for} \PY{n}{i} \PY{o+ow}{in} \PY{n+nb}{range}\PY{p}{(}\PY{l+m+mi}{40}\PY{p}{)}\PY{p}{:}
              \PY{c}{\PYZsh{}populate a list of 41 date/times with even time intervals over our 16834 points}
              \PY{n}{xlabels}\PY{o}{.}\PY{n}{append}\PY{p}{(}\PY{n}{gcFrame}\PY{p}{[}\PY{l+s}{\PYZsq{}}\PY{l+s}{Date/Time}\PY{l+s}{\PYZsq{}}\PY{p}{]}\PY{p}{[}\PY{n}{i}\PY{o}{*}\PY{l+m+mi}{420}\PY{p}{]}\PY{p}{)}
              
              \PY{c}{\PYZsh{}an array to place the date/time labels at the corresponding x value}
              \PY{n}{xticks}\PY{o}{.}\PY{n}{append}\PY{p}{(}\PY{n}{gcFrame}\PY{p}{[}\PY{l+s}{\PYZsq{}}\PY{l+s}{Scatter}\PY{l+s}{\PYZsq{}}\PY{p}{]}\PY{p}{[}\PY{n}{i}\PY{o}{*}\PY{l+m+mi}{420}\PY{p}{]}\PY{p}{)}
          
          \PY{c}{\PYZsh{}place the date/time labels and rotate them}
          \PY{n}{TSPlotG2}\PY{o}{.}\PY{n}{set\PYZus{}xticklabels}\PY{p}{(}\PY{n}{xlabels}\PY{p}{,} \PY{n}{rotation}\PY{o}{=}\PY{l+m+mi}{90}\PY{p}{,} \PY{n}{fontsize}\PY{o}{=}\PY{l+m+mi}{15}\PY{p}{)} 
          \PY{n}{TSPlotG2}\PY{o}{.}\PY{n}{set\PYZus{}xticks}\PY{p}{(}\PY{n}{xticks}\PY{p}{)}
          
          \PY{c}{\PYZsh{}add the points to the graph}
          \PY{n}{Scat} \PY{o}{=} \PY{n}{plt}\PY{o}{.}\PY{n}{scatter}\PY{p}{(}\PY{n}{gcFrame}\PY{p}{[}\PY{l+s}{\PYZsq{}}\PY{l+s}{Scatter}\PY{l+s}{\PYZsq{}}\PY{p}{]}\PY{p}{,}\PY{n}{gcFrame}\PY{p}{[}\PY{l+s}{\PYZsq{}}\PY{l+s}{Detections}\PY{l+s}{\PYZsq{}}\PY{p}{]}\PY{p}{,} \PY{n}{s}\PY{o}{=}\PY{l+m+mf}{2.5}\PY{p}{,} \PY{n}{color}\PY{o}{=}\PY{l+s}{\PYZsq{}}\PY{l+s}{black}\PY{l+s}{\PYZsq{}}\PY{p}{)} 
          \PY{n}{TSPlotG2}\PY{o}{.}\PY{n}{legend}\PY{p}{(}\PY{p}{(}\PY{l+s}{\PYZsq{}}\PY{l+s}{Fit Data}\PY{l+s}{\PYZsq{}}\PY{p}{,}\PY{l+s}{\PYZsq{}}\PY{l+s}{SMA}\PY{l+s}{\PYZsq{}}\PY{p}{,}\PY{l+s}{\PYZsq{}}\PY{l+s}{Data Points (time,detections/5sec)}\PY{l+s}{\PYZsq{}}\PY{p}{)}\PY{p}{,} \PY{n}{loc}\PY{o}{=}\PY{l+m+mi}{1}\PY{p}{)}\PY{p}{;}
\end{Verbatim}

    \begin{center}
    \adjustimage{max size={0.9\linewidth}{0.9\paperheight}}{SeniorProjectPDF_files/SeniorProjectPDF_308_0.png}
    \end{center}
    { \hspace*{\fill} \\}
    
    I have heard the centered moving average is also built-in to Pandas, but
it was just as fast to write my own function for it.

    \begin{Verbatim}[commandchars=\\\{\}]
{\color{incolor}In [{\color{incolor}144}]:} \PY{c}{\PYZsh{}function for CMA since it wasn\PYZsq{}t built into pandas}
          \PY{k}{def} \PY{n+nf}{CMA}\PY{p}{(}\PY{n}{vec}\PY{p}{,} \PY{n}{n}\PY{p}{)}\PY{p}{:}
              \PY{n}{c} \PY{o}{=} \PY{n}{n}\PY{o}{/}\PY{l+m+mi}{2}
              \PY{n}{cma} \PY{o}{=} \PY{n}{np}\PY{o}{.}\PY{n}{zeros}\PY{p}{(}\PY{p}{[}\PY{l+m+mi}{16384}\PY{p}{,}\PY{l+m+mi}{1}\PY{p}{]}\PY{p}{)}  \PY{c}{\PYZsh{}created an empty array to store values}
              \PY{k}{for} \PY{n}{i} \PY{o+ow}{in} \PY{n+nb}{range}\PY{p}{(}\PY{n}{c}\PY{p}{,} \PY{n+nb}{len}\PY{p}{(}\PY{n}{vec}\PY{p}{)}\PY{o}{\PYZhy{}}\PY{n}{c}\PY{p}{)}\PY{p}{:}    \PY{c}{\PYZsh{}look at values from n/2 to 16384\PYZhy{}n/2}
                  \PY{n}{window} \PY{o}{=} \PY{n}{vec}\PY{p}{[}\PY{n}{i}\PY{o}{\PYZhy{}}\PY{n}{c}\PY{p}{:}\PY{n}{i}\PY{o}{+}\PY{n}{c}\PY{o}{+}\PY{l+m+mi}{1}\PY{p}{]}       \PY{c}{\PYZsh{}taking average over these values}
                  \PY{n}{cma}\PY{p}{[}\PY{n}{i}\PY{p}{]} \PY{o}{=} \PY{p}{(}\PY{n}{np}\PY{o}{.}\PY{n}{mean}\PY{p}{(}\PY{n}{window}\PY{p}{)}\PY{p}{)}    \PY{c}{\PYZsh{}store average in array}
              \PY{n}{cma}\PY{p}{[}\PY{p}{:}\PY{n}{c}\PY{p}{]} \PY{o}{=} \PY{n+nb+bp}{None}                    \PY{c}{\PYZsh{}fill first n/2 values with None}
              \PY{n}{cma}\PY{p}{[}\PY{p}{(}\PY{n+nb}{len}\PY{p}{(}\PY{n}{vec}\PY{p}{)}\PY{o}{\PYZhy{}}\PY{n}{c}\PY{p}{)}\PY{p}{:}\PY{p}{]} \PY{o}{=} \PY{n+nb+bp}{None}         \PY{c}{\PYZsh{}fill last n/2 values with None}
              \PY{k}{return} \PY{n}{cma}
\end{Verbatim}

    \begin{Verbatim}[commandchars=\\\{\}]
{\color{incolor}In [{\color{incolor}145}]:} \PY{n}{gcFrame}\PY{p}{[}\PY{l+s}{\PYZsq{}}\PY{l+s}{CMA}\PY{l+s}{\PYZsq{}}\PY{p}{]} \PY{o}{=} \PY{n}{CMA}\PY{p}{(}\PY{n}{gcFrame}\PY{o}{.}\PY{n}{Detections}\PY{p}{,}\PY{l+m+mi}{24}\PY{p}{)}   \PY{c}{\PYZsh{}add column to the data frame}
\end{Verbatim}

    This time, let's look at only a section of the data. For this, we will
slice the data from 7am to 8am on September 24th.

    \begin{Verbatim}[commandchars=\\\{\}]
{\color{incolor}In [{\color{incolor}147}]:} \PY{k}{print}\PY{p}{(}\PY{n}{gcFrame}\PY{p}{[}\PY{l+s}{\PYZsq{}}\PY{l+s}{Date/Time}\PY{l+s}{\PYZsq{}}\PY{p}{]}\PY{p}{[}\PY{l+m+mi}{9763}\PY{p}{]}\PY{p}{,}\PY{n}{gcFrame}\PY{p}{[}\PY{l+s}{\PYZsq{}}\PY{l+s}{Date/Time}\PY{l+s}{\PYZsq{}}\PY{p}{]}\PY{p}{[}\PY{l+m+mi}{10483}\PY{p}{]}\PY{p}{)}
\end{Verbatim}

    \begin{Verbatim}[commandchars=\\\{\}]
('09/24/2013 7:00:02', '09/24/2013 8:00:02')
    \end{Verbatim}

    With the indicies above, i'm creating a new data frame with all the
varibles I want from 7am to 8am.

    \begin{Verbatim}[commandchars=\\\{\}]
{\color{incolor}In [{\color{incolor}148}]:} \PY{c}{\PYZsh{}slice the data}
          \PY{n}{sliceFrame} \PY{o}{=} \PY{n}{pd}\PY{o}{.}\PY{n}{DataFrame}\PY{p}{(}\PY{n}{gcFrame}\PY{o}{.}\PY{n}{CMA}\PY{p}{[}\PY{l+m+mi}{9763}\PY{p}{:}\PY{l+m+mi}{10484}\PY{p}{]}\PY{p}{,} \PY{n}{columns}\PY{o}{=}\PY{p}{[}\PY{l+s}{\PYZsq{}}\PY{l+s}{CMA}\PY{l+s}{\PYZsq{}}\PY{p}{]}\PY{p}{)}
          \PY{n}{sliceFrame}\PY{p}{[}\PY{l+s}{\PYZsq{}}\PY{l+s}{SMA}\PY{l+s}{\PYZsq{}}\PY{p}{]} \PY{o}{=} \PY{n}{gcFrame}\PY{o}{.}\PY{n}{SMA}\PY{p}{[}\PY{l+m+mi}{9763}\PY{p}{:}\PY{l+m+mi}{10484}\PY{p}{]}
          \PY{n}{sliceFrame}\PY{p}{[}\PY{l+s}{\PYZsq{}}\PY{l+s}{Detections}\PY{l+s}{\PYZsq{}}\PY{p}{]} \PY{o}{=} \PY{n}{gcFrame}\PY{o}{.}\PY{n}{Detections}\PY{p}{[}\PY{l+m+mi}{9763}\PY{p}{:}\PY{l+m+mi}{10484}\PY{p}{]}
          \PY{n}{sliceFrame}\PY{p}{[}\PY{l+s}{\PYZsq{}}\PY{l+s}{Scatter}\PY{l+s}{\PYZsq{}}\PY{p}{]} \PY{o}{=} \PY{n}{gcFrame}\PY{o}{.}\PY{n}{Scatter}\PY{p}{[}\PY{l+m+mi}{9763}\PY{p}{:}\PY{l+m+mi}{10484}\PY{p}{]}
          \PY{n}{sliceFrame}\PY{p}{[}\PY{l+s}{\PYZsq{}}\PY{l+s}{Date/Time}\PY{l+s}{\PYZsq{}}\PY{p}{]} \PY{o}{=} \PY{n}{gcFrame}\PY{p}{[}\PY{l+s}{\PYZsq{}}\PY{l+s}{Date/Time}\PY{l+s}{\PYZsq{}}\PY{p}{]}\PY{p}{[}\PY{l+m+mi}{9763}\PY{p}{:}\PY{l+m+mi}{10484}\PY{p}{]}
          \PY{n+nb}{len}\PY{p}{(}\PY{n}{sliceFrame}\PY{o}{.}\PY{n}{CMA}\PY{p}{)}
\end{Verbatim}

            \begin{Verbatim}[commandchars=\\\{\}]
{\color{outcolor}Out[{\color{outcolor}148}]:} 721
\end{Verbatim}
        
    Now there are only 721 observations. The way I wrote the axis function,
it will need to be updated. There are time locators that will determine
the spacing automatically, but I enjoy functions. If I had to do this
more, I would write a general fuction for determining the time ticks for
the x axis. After adjusting the x axis fuction, we'll see how the SMA
and CMA compare in the sliced time window.

    \begin{Verbatim}[commandchars=\\\{\}]
{\color{incolor}In [{\color{incolor}153}]:} \PY{c}{\PYZsh{}open a figure window and color it white}
          \PY{n}{figure} \PY{o}{=} \PY{n}{plt}\PY{o}{.}\PY{n}{figure}\PY{p}{(}\PY{n}{figsize}\PY{o}{=}\PY{p}{(}\PY{l+m+mi}{20}\PY{p}{,}\PY{l+m+mi}{8}\PY{p}{)}\PY{p}{)}
          
          
          \PY{c}{\PYZsh{}built\PYZhy{}in DataFrame function to plot time as x and theta as y, with custom y limits, }
          \PY{c}{\PYZsh{}line width, and color}
          \PY{n}{TSPlotG3} \PY{o}{=} \PY{n}{sliceFrame}\PY{o}{.}\PY{n}{plot}\PY{p}{(}\PY{n}{x}\PY{o}{=}\PY{l+s}{\PYZsq{}}\PY{l+s}{Scatter}\PY{l+s}{\PYZsq{}}\PY{p}{,}\PY{n}{y}\PY{o}{=}\PY{l+s}{\PYZsq{}}\PY{l+s}{Detections}\PY{l+s}{\PYZsq{}}\PY{p}{,}\PY{n}{linewidth}\PY{o}{=}\PY{l+m+mi}{2}\PY{p}{,} \PY{n}{ylim}\PY{o}{=}\PY{p}{(}\PY{l+m+mi}{0}\PY{p}{,}\PY{l+m+mi}{17}\PY{p}{)}\PY{p}{,} \PY{n}{c}\PY{o}{=}\PY{l+s}{\PYZsq{}}\PY{l+s}{grey}\PY{l+s}{\PYZsq{}}\PY{p}{)}
          
          \PY{c}{\PYZsh{}set x and y labels, title, and adjust thier sizes according}
          \PY{n}{TSPlotG3}\PY{o}{.}\PY{n}{set\PYZus{}ylabel}\PY{p}{(}\PY{l+s}{\PYZsq{}}\PY{l+s}{\PYZsh{} of Detections in 5 Seconds}\PY{l+s}{\PYZsq{}}\PY{p}{,} \PY{n}{fontsize}\PY{o}{=}\PY{l+m+mi}{16}\PY{p}{)}
          \PY{n}{TSPlotG3}\PY{o}{.}\PY{n}{set\PYZus{}title}\PY{p}{(}\PY{l+s}{\PYZsq{}}\PY{l+s}{Geiger Counter Source Radioactivity Time Series}\PY{l+s}{\PYZsq{}}\PY{p}{,}\PY{n}{fontsize}\PY{o}{=}\PY{l+m+mi}{20}\PY{p}{)}
          \PY{n}{plt}\PY{o}{.}\PY{n}{tick\PYZus{}params}\PY{p}{(}\PY{n}{axis}\PY{o}{=}\PY{l+s}{\PYZsq{}}\PY{l+s}{both}\PY{l+s}{\PYZsq{}}\PY{p}{,} \PY{n}{labelsize}\PY{o}{=}\PY{l+m+mi}{15}\PY{p}{)}   \PY{c}{\PYZsh{}change the font size of the axes ticks}
          
          \PY{n}{xlabels} \PY{o}{=} \PY{p}{[}\PY{p}{]}
          \PY{n}{xticks} \PY{o}{=} \PY{p}{[}\PY{p}{]}
          \PY{k}{for} \PY{n}{i} \PY{o+ow}{in} \PY{n+nb}{range}\PY{p}{(}\PY{l+m+mi}{25}\PY{p}{)}\PY{p}{:}
              \PY{c}{\PYZsh{}populate a list of 25 date/times with even time intervals over our 721 points}
              \PY{n}{xlabels}\PY{o}{.}\PY{n}{append}\PY{p}{(}\PY{n}{sliceFrame}\PY{p}{[}\PY{l+s}{\PYZsq{}}\PY{l+s}{Date/Time}\PY{l+s}{\PYZsq{}}\PY{p}{]}\PY{p}{[}\PY{l+m+mi}{9763}\PY{o}{+}\PY{n}{i}\PY{o}{*}\PY{l+m+mi}{30}\PY{p}{]}\PY{p}{)}  
              
              \PY{c}{\PYZsh{}an array to place the date/time labels at the corresponding x value}
              \PY{n}{xticks}\PY{o}{.}\PY{n}{append}\PY{p}{(}\PY{n}{sliceFrame}\PY{p}{[}\PY{l+s}{\PYZsq{}}\PY{l+s}{Scatter}\PY{l+s}{\PYZsq{}}\PY{p}{]}\PY{p}{[}\PY{l+m+mi}{9763}\PY{o}{+}\PY{n}{i}\PY{o}{*}\PY{l+m+mi}{30}\PY{p}{]}\PY{p}{)}     \PY{c}{\PYZsh{}an array to place the date/time labels at the corresponding x value}
          
          \PY{n}{TSPlotG3}\PY{o}{.}\PY{n}{set\PYZus{}xticklabels}\PY{p}{(}\PY{n}{xlabels}\PY{p}{,} \PY{n}{rotation}\PY{o}{=}\PY{l+m+mi}{90}\PY{p}{,} \PY{n}{fontsize}\PY{o}{=}\PY{l+m+mi}{15}\PY{p}{)}    \PY{c}{\PYZsh{}place the date/time labels and rotate them}
          \PY{n}{TSPlotG3}\PY{o}{.}\PY{n}{set\PYZus{}xticks}\PY{p}{(}\PY{n}{xticks}\PY{p}{)}
          \PY{n}{Line2} \PY{o}{=} \PY{n}{sliceFrame}\PY{o}{.}\PY{n}{plot}\PY{p}{(}\PY{n}{x}\PY{o}{=}\PY{l+s}{\PYZsq{}}\PY{l+s}{Scatter}\PY{l+s}{\PYZsq{}}\PY{p}{,}\PY{n}{y}\PY{o}{=}\PY{l+s}{\PYZsq{}}\PY{l+s}{SMA}\PY{l+s}{\PYZsq{}}\PY{p}{,}\PY{n}{linewidth}\PY{o}{=}\PY{l+m+mi}{5}\PY{p}{,} \PY{n}{c}\PY{o}{=}\PY{l+s}{\PYZsq{}}\PY{l+s}{red}\PY{l+s}{\PYZsq{}}\PY{p}{)}
          \PY{n}{Line3} \PY{o}{=} \PY{n}{sliceFrame}\PY{o}{.}\PY{n}{plot}\PY{p}{(}\PY{n}{x}\PY{o}{=}\PY{l+s}{\PYZsq{}}\PY{l+s}{Scatter}\PY{l+s}{\PYZsq{}}\PY{p}{,}\PY{n}{y}\PY{o}{=}\PY{l+s}{\PYZsq{}}\PY{l+s}{CMA}\PY{l+s}{\PYZsq{}}\PY{p}{,}\PY{n}{linewidth}\PY{o}{=}\PY{l+m+mi}{5}\PY{p}{,} \PY{n}{c}\PY{o}{=}\PY{l+s}{\PYZsq{}}\PY{l+s}{blue}\PY{l+s}{\PYZsq{}}\PY{p}{)}
          \PY{n}{TSPlotG3}\PY{o}{.}\PY{n}{set\PYZus{}xlabel}\PY{p}{(}\PY{l+s}{\PYZsq{}}\PY{l+s}{Date/Time}\PY{l+s}{\PYZsq{}}\PY{p}{,}\PY{n}{fontsize}\PY{o}{=}\PY{l+m+mi}{16}\PY{p}{)}
          \PY{n}{TSPlotG3}\PY{o}{.}\PY{n}{legend}\PY{p}{(}\PY{p}{(}\PY{l+s}{\PYZsq{}}\PY{l+s}{Fit Data}\PY{l+s}{\PYZsq{}}\PY{p}{,}\PY{l+s}{\PYZsq{}}\PY{l+s}{SMA}\PY{l+s}{\PYZsq{}}\PY{p}{,}\PY{l+s}{\PYZsq{}}\PY{l+s}{CMA}\PY{l+s}{\PYZsq{}}\PY{p}{)}\PY{p}{,} \PY{n}{loc}\PY{o}{=}\PY{l+m+mi}{1}\PY{p}{)}\PY{p}{;}
\end{Verbatim}

    \begin{center}
    \adjustimage{max size={0.9\linewidth}{0.9\paperheight}}{SeniorProjectPDF_files/SeniorProjectPDF_317_0.png}
    \end{center}
    { \hspace*{\fill} \\}
    

    \section{Chapter 10 Formatting and Coverting IPython Notebooks}


    \begin{Verbatim}[commandchars=\\\{\}]
{\color{incolor}In [{\color{incolor}}]:} 
\end{Verbatim}


    % Add a bibliography block to the postdoc
    
    
    
    \end{document}
